%----------------------------------------------------------------------------
% ----- File:        intro.tex 
% ----- Author:      Rainer Menzner (Rainer.Menzner@web.de)
% ----- Date:        2001-06-03
% ----- Description: This file is part of the t1lib-documentation.
% ----- Copyright:   t1lib is copyrighted (c) Rainer Menzner, 1996-2001.
%                    As of version 0.5, t1lib is distributed under the
%                    GNU General Public Library License. The
%                    conditions can be found in the files LICENSE and
%                    LGPL, which should reside in the toplevel
%                    directory of the distribution.  Please note that 
%                    there are parts of t1lib that are subject to
%                    other licenses:
%                    The parseAFM-package is copyrighted by Adobe Systems
%                    Inc.
%                    The type1 rasterizer is copyrighted by IBM and the
%                    X11-consortium.
% ----- Warranties:  Of course, there's NO WARRANTY OF ANY KIND :-)
% ----- Credits:     I want to thank IBM and the X11-consortium for making
%                    their rasterizer freely available.
%                    Also thanks to Piet Tutelaers for his ps2pk, from
%                    which I took the rasterizer sources in a format
%                    independent from X11.
%                    Thanks to all people who make free software living!
%----------------------------------------------------------------------------

\newpage
\section{Introduction}
\subsection{What Does \tonelib\ Do?}
\tonelib\ is a library written in the C programming language 
allowing a programmer to generate bitmaps from Adobe (TM) Type 1 fonts
quite easily. These bitmaps are returned in a data structure
with type {\ttfamily GLYPH}. This special
{\ttfamily GLYPH}-type is also used in the X11 window system to
describe character bitmaps. It contains the bitmap data as well as
some metric information. But \tonelib\ is in itself entirely independent
of the X11-system or any other graphical user interface.

Given that the X11-system is probably the most frequently used
window system in the UNIX-world, and furthermore assuming that most
graphical applications run under window systems, it appears that
\tonelib\ implements functionality already provided by the
graphical user interface,
the X11-system. Thus the question arises: Why not use X11 directly for
rastering characters? Well, the answer is quite simple; the X11 Font
machinery appears to be too {\em static} in order to use it for
certain purposes. Moreover X11-calculations and positioning of
characters are based on bitmap
dimensions and are thus subject to error accumulation.

Here is a list of features which are supported in the current release
of \tonelib.
\begin{itemize}
\item Rasterizing is done as characters get requested. The X11-server,
  in contrast,
  rasters a font completely at a given size
  when it is loaded. {\em Rastering on demand} saves time when the
  font is loaded the first time and saves memory since often only the
  alphabetic letters and a few other characters are needed.
\item The encoding mechanism of PostScript is fully supported. The
  user may use the fonts' internal or some other encoding.
  Additional encoding files can be loaded at runtime and fonts can be
  reencoded at runtime. The syntax of encoding files is
  straight-forward and simple.
\item The library makes use of Adobe Font Metric data in the form of
  AFM files. This may seem a disadvantage, but in order
  to make decent typesetting possible, some more information than that
  contained in Type 1 font files is needed. Besides, Adobe makes
  these AFM files freely available on their ftp-server for all
  registered Type 1 fonts.
\item In case AFM files are missing, \tonelib\ is able to generate
  metrics information in charspace accuracy by rastering each
  character at 1000~bp.
\item In addition to generation of character bitmaps, there is a way
  to directly raster strings of any length in a given font. 
  The space-width may explicitly be corrected by the user. This
  may be needed by word processing applications.
\item Strings may be rastered 
  optionally using pairwise kerning information from
  the AFM file. Kerning is an important feature of good
  quality typesetting.
  Kerning information can also be requested by the user without
  rasterization. 
\item Ligature information is made available to the user in an
  efficient way. Use of Ligatures is another characteristic of good
  typesetting. As to my 
  knowledge, only \TeX\ and all related macro-packages are able to
  handle ligatures in a natural and efficient way. 
\item Rotation and arbitrary transformation on the fly is supported. 
\item \tonelib\ supports antialiasing. In this case, a pixel is
  represented as a byte, word or double word. Antialiasing is
  implemented by subsampling with factor 2 or 4 alternatively.
  If you use the X11-interface introduced in \tonelib\ V.\ 0.3-beta, 
  even colored antialiasing between any pair of colors is provided in a
  completely transparent way.
\item In addition to transformation on the fly, two transformation
  types---{\em slanting} and {\em extending}---are possible on the fontlevel
  including bitmap caching. Horizontal expansion of fonts is fully supported
  and since version 0.3-beta also {\em slanting} of fonts is nearly fully
  supported.  For restrictions and drawbacks of slanting fonts see
  \ref{transformations} on page \pageref{transformations}.
\item Paths, the library searches for the different needed file types
  are specified at runtime by means of a
  configuration file. They may thus be changed without needing to
  recompile the application. For example, the directories of the
  X11-system's Type 1 font files may be specified there in order to use
  these fonts with the library. A user may have his own configuration file and
  as a fallback/default there is a system wide configuration file. This should
  be setup when the library is installed.
\item Since of version 0.3-beta a special set of functions is provided which
  implements a more comfortable X11-interface. This is due to the fact that
  X11 is the only standard window system in the UNIX world. However, as before
  \tonelib\ may be compiled and used without even having X11 installed.
\item Some decorations like {\em underlining} are supported by simply setting
  a flag for the rastering functions.
\item {\em Right To Left} typesetting is supported.
\item Font subsetting is easily achieved using a high level function. This
  makes it easier for application to efficiently export Postscript files.
\item Composite character information can be retrieved. Moreover, \tonelib\
  transparently handles and realizes composite character definitions from AFM
  files.
\end{itemize}
There are also some problems and features not yet
implemented, but likely to be implemented in the future.
The main problem up to now is:
\begin{itemize}
\item The font cache isn't a font cache really. At this time all
  bitmaps are saved by always allocating more memory from the
  system. No automatic removal of bitmaps no longer needed is 
  done. However, the user has the possibility of explicitly
  removing data, if he thinks it is not needed anymore.
\end{itemize}

\subsection{Copyrights and Credits}
There are some copyrights on parts of the library and there are some
programmers (or corporations) which I want to give credit. 
The library uses:
\begin{itemize}
\item[--] all internal parts of the X11-rasterizer donated to
  the X11-project by IBM. This rasterizer does the {\em real} hard
  work of scan-conversion.
\item[--] the modifications to the rasterizer done by Piet
  Tutelaers in his ps2pk-package. The main purpose was decoupling the 
  sources from the X11-system sources.
\item[--] the \verb+parse_AFM+ software which was made freely available
  by Adobe. This is used to parse the AFM files (what
  a surprise) and to generate the data structures the information is
  saved in.
\end{itemize}

Raph Levien (\verb+raph@acm.org+) contributed an algorithm for 
sampling down non-antialiased bitmaps to antialiased bitmaps in a very
efficient manor. This makes antialiasing a lot faster.

Fred L.\ Drake, Jr. (\verb+fdrake@acm.org+) wrote a Python interface to
\tonelib, which is distributed with \tonelib. This wrapper is called
\verb+t1python+ and allows Python-programmers to use Type 1 fonts. I
can not tell anything more on this topic since I do not know the
Python language. All questions concerning the Python interface should
thus be addressed to Fred L.\ Drake, Jr.

Evgeny Stambulchik (\verb+fnevgeny@plasma-gate.weizmann.ac.il+), maintainer of
the \verb+grace+-project---a descendent of \verb+xmgr+, never gets tired of
finding and reporting (and fixing) bugs in \tonelib. Other members of this
project spent time in porting \tonelib\ to further systems:
\begin{itemize}
\item Ed Vigmond (\verb+vigmonde@IGB.UMontreal.CA+): IRIX-port,
\item John Hasstedt (\verb+John.Hasstedt@sunysb.edu+): VMS-port,
\item Alexander Mai (\verb+st002279@hrzpub.tu-darmstadt.de+): OS/2-port.
\end{itemize}

Hirotsugo Kakugawa (\verb+h.kakugawa@computer.org+) added support for GNU
\verb+libtool+ to \tonelib.

David Huggins-Daines (\verb+bn711@freenet.carleton.ca+) spent effort in
finding memory leaks and maintains a Debian-package of \tonelib.


\begin{center}
  \Large Thanks to all these people \\
  and to all those contributors not mentioned here! 
\end{center}

\subsection{Motivation}
The idea of writing this library was due to the SciTeXt-project which
was founded in 1996. They needed a font raster system with some more
functionality than X11 provided. You may find that some things (for
example the format of the font database file) is really closely
related to SciTeXt. In February '{}97 the SciTeXt developers decided to
migrate from C to Java and did not need a font rasterizer any more. Since I
thought there could be other applications for this library I continued the
work on it and voila. 

I have removed the history description which followed at this place in some
previous versions. The history can be roughly viewed in the file \verb+Changes+
which is located in the toplevel directory of the distribution.


\subsection{How to Reach the Author/How to Get \tonelib}
If you have questions, comments or bug reports, you can reach me by
eMail. The address is 
\begin{center}
\verb+Rainer.Menzner@web.de+
\end{center}
If you have bug-reports, it would be best if you could reproduce the error
using the test program \verb+xglyph+ (see \ref{xglyph}). 

\tonelib\ is available by anonymous ftp as\\[5mm]
\verb+ftp://sunsite.unc.edu/pub/Linux/libs/graphics/t1lib-+$x.y[.z]$\verb+.tar.gz+
\\[5mm]
Here, $x.y.z$ is the version identifier and the brackets around $z$ indicate
that this entry is optional.

%%% Local Variables: 
%%% mode: latex
%%% TeX-master: "t1lib_doc"
%%% End: 
