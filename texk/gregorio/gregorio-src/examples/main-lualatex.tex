% !TEX program = LuaLaTeX+se

% This is a simple template for a LuaLaTeX document using gregorio scores.

\documentclass[11pt]{article} % use larger type; default would be 10pt

% usual packages loading:
\usepackage{fontspec}
\usepackage{graphicx} % support the \includegraphics command and options
\usepackage{geometry} % See geometry.pdf to learn the layout options. There are lots.
\geometry{a4paper} % or letterpaper (US) or a5paper or....
\usepackage{gregoriotex} % for gregorio score inclusion
\usepackage{fullpage} % to reduce the margins
\usepackage{libertine} % Decent (free) font for Gregorian, but should be changed if you have high standards

\begin{document}

% The title:
\begin{center}\begin{huge}\textsc{Populus Sion}\end{huge}\end{center}

% Here we set the space around the initial.
% Please report to http://gregorio-project.github.io/gregoriotex/details.html for more details and options
\grechangedim{beforeinitialshift}{2.2mm}{scalable}
\grechangedim{afterinitialshift}{2.2mm}{scalable}

% Here we set the initial font. Change 43 if you want a bigger initial.
\grechangestyle{initial}{\fontsize{43}{43}\selectfont}%

% We set red lines here, comment it if you want black ones.
\gresetlinecolor{gregoriocolor}

% We set VII above the initial.
\grechangestyle{annotation}{\small\bfseries}
\greannotation{Intr.}
\greannotation{\textsc{vii}}
% We type a text in the top right corner of the score:
\grecommentary{Cf. Is. 30, 19 . 30 ; Ps. 79}

% and finally we include the scores. The file must be in the same directory as this one.
\gregorioscore[a]{PopulusSion}

\bigskip
\begin{center}\begin{huge}\textsc{Factus Est}\end{huge}\end{center}

\gregorioscore[a]{FactusEst}

\end{document}
