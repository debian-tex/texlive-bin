% eptex

% \Uchar <chr_code>
%  0--255:常に欧文文字トークン
%  256以上の,内部コードで許される値:常に和文文字トークン
% \Ucharcat <chr_code> <catcode>
% <chr_code> in [0,128): 欧文文字トークンを生成.<catcode> in {1..4, 6..8, 10..13}
% <chr_code> in [128,256)
%    e-pTeX の場合:欧文文字トークンを生成.<catcode> in {1..4, 6..8, 10..13}
%    e-upTeX の場合:欧文/和文文字トークンを生成.<catcode> in {1..4, 6..8, 10..13, 16..19}
% <chr_code> >=256: 和文文字トークンを生成.
%    e-pTeX の場合: <catcode> in {16..18}
%    e-upTeX の場合:<catcode> in {16..19}


% e-upTeX:


\let\bg={ \let\eg=}
{\catcode`\ =9\relax
\gdef\KCAT{%
  \immediate\write17{%
    [\expandafter\string\x\space
    \expandafter\ifcat\x$   math\space shift\fi
    \expandafter\ifcat\x&   alignment\fi
    \expandafter\ifcat\x^   superscript\fi
    \expandafter\ifcat\x_   subscript\fi
    \expandafter\ifcat\x\space space\fi
    \expandafter\ifcat\x a  letter\fi
    \expandafter\ifcat\x 1  other\space char\fi
    \expandafter\ifcat\x ~  active\fi
    \expandafter\ifcat\x 空 kanji\fi
    \expandafter\ifcat\x ア kana\fi
    \expandafter\ifcat\x { other\space kchar\fi
    \ifdefined\ucs
      \expandafter\ifcat\x 한 hangul\fi
    \fi]}%
}}
\let\sharp=#
\font\x=ec-lmtt10 \x
\scrollmode

\edef\x{\Uchar`\{}\KCAT
\edef\x{\Uchar`\}}\KCAT
\edef\x{\Uchar`\$}\KCAT
\edef\x{\Uchar`\&}\KCAT
\edef\x{\Uchar`\#}\KCAT
\edef\x{\Uchar`\^}\KCAT
\edef\x{\Uchar`\_}\KCAT
\edef\x{\Uchar`\ }\KCAT
\edef\x{\Uchar`\a}\KCAT
\edef\x{\Uchar`\1}\KCAT
\edef\x{\Uchar`~}\KCAT
\edef\x{\Uchar`漢}\KCAT
\edef\x{\Uchar`あ}\KCAT
\edef\x{\Uchar`)}\KCAT

\ifdefined\ucs %======
\immediate\write0{■\string\Uchar\space and \string\kcatcode}
{\kcatcode"03B1=15 \kcatcode"FF=15
\edef\x{\Uchar"FF}\KCAT%"
\edef\x{\Uchar"03B1}\KCAT%"
}

{\kcatcode"03B1=17 \kcatcode"FF=17
\edef\x{\Uchar"FF}\KCAT%"
\edef\x{\Uchar"03B1}\KCAT%"
}
\fi %======

\immediate\write0{■\string\Ucharcat.}

\edef\x{\Ucharcat`\# 0}\KCAT % error "! Invalid code"
\edef\x{\Ucharcat`\# 3}\KCAT
\edef\x{\Ucharcat`\# 4}\KCAT
\edef\x{\Ucharcat`\# 5}\KCAT % error "! Invalid code"
\edef\x{\Ucharcat`\# 7}\KCAT
\edef\x{\Ucharcat`\# 8}\KCAT
\edef\x{\Ucharcat`\# 9}\KCAT % error "! Invalid code"
\edef\x{\Ucharcat`\# 10}\KCAT
\edef\x{\Ucharcat`\# 11}\KCAT
\edef\x{\Ucharcat`\# 12}\KCAT
\edef\x{\unexpanded\expandafter{\Ucharcat`\# 13}}
\message{\expandafter\meaning\unexpanded\expandafter{\x}} % undefined
\edef\x{\Ucharcat`\# 14}\KCAT % error "! Invalid code"
\edef\x{\Ucharcat`\# 15}\KCAT % error "! Invalid code"

\edef\x{\Ucharcat`\# 16} % error "! Invalid code (16)"
\KCAT

\ifdefined\ucs %======
\edef\x{\Ucharcat`漢 3} % error "! Invalid code (3)"
\KCAT

\edef\x{\Ucharcat`$ 16}\KCAT
\edef\x{\Ucharcat`: 17}\KCAT
\edef\x{\Ucharcat`あ 18}\KCAT
\edef\x{\Ucharcat`漢 19}\KCAT

{\kcatcode"03B1=15 %"
\edef\x{\Ucharcat"03B1 12}\KCAT%" error "! Invalid code (12)"
\edef\x{\Ucharcat"03B1 17}\KCAT%"
}

{\kcatcode"03B1=16 %"
\edef\x{\Ucharcat"03B1 12}\KCAT%" error "! Invalid code (12)"
\edef\x{\Ucharcat"03B1 17}\KCAT%"
}

{\kcatcode"FF=15 %"
\edef\x{\Ucharcat"FF 12}\KCAT%"
\edef\x{\Ucharcat"FF 17}\KCAT%"
}

{\kcatcode"FF=16 %"
\edef\x{\Ucharcat"FF 12}\KCAT%"
\edef\x{\Ucharcat"FF 17}\KCAT%"
}
\else

\edef\x{\Ucharcat`漢 3} % error "! Bad character code" in eptex

\fi % ======


\end

