% !TEX program = LuaLaTeX+se
\documentclass[12pt,a4paper]{article}
\usepackage[titletoc,toc,title]{appendix}
\usepackage{fontspec}
\defaultfontfeatures{Ligatures=TeX}
\setmainfont[
		BoldFont=LinLibertineOZ,
		BoldItalicFont=LinLibertineOZI,
		SmallCapsFont=LinLibertineO,
		SmallCapsFeatures={Letters=SmallCaps},
]{LinuxLibertineO}
\setsansfont[Extension=.otf,
		BoldFont=LinBiolinum_RB,
		ItalicFont=LinBiolinum_RI,
		BoldItalicFont=LinBiolinum_RB,% fake
		SmallCapsFont=LinBiolinum_R,
		SmallCapsFeatures={Letters=SmallCaps},
		]{LinBiolinum_R}
\setmonofont[Extension=.otf,
		ItalicFont=Inconsolatazi4-Regular,
		BoldFont=Inconsolatazi4-Bold,
		BoldItalicFont=Inconsolatazi4-Bold,
		AutoFakeSlant,
		ItalicFeatures={FakeSlant},
		BoldItalicFeatures={FakeSlant}
		]{Inconsolatazi4-Regular}
\usepackage{enumitem}
\usepackage{xspace}
\usepackage{multicol}
\usepackage{units}
\usepackage{framed}
\usepackage{url}
\usepackage{tabulary}
\usepackage{tabularx}
\usepackage{adjustbox}
\usepackage{xparse}

\usepackage{makeidx}
\makeindex

\usepackage[table]{xcolor}
\definecolor{lightgray}{gray}{0.9}
\definecolor{green}{HTML}{0c700c}
\definecolor{myred}{HTML}{FF3333}

\usepackage{minted} % must be after xcolor
\makeatletter
\@ifpackagelater{minted}{2013/12/21}{%
	\newminted{latex}{autogobble,bgcolor=lightgray}%
}{%
	\newminted{latex}{gobble=2,bgcolor=lightgray}%
}%
\makeatother
%\newminted{gabc}{bgcolor=lightgray} % can this be done?
\newenvironment{gabccode}{\tt}{}


\usepackage[allowdeprecated=false]{gregoriotex}

\usepackage{carolmin}
\usepackage{mflogo}

\usepackage{hyperref}
\hypersetup{colorlinks, citecolor=black, filecolor=black, linkcolor=green, urlcolor=green}

\usepackage{longtable}
\usepackage{multirow}
\usepackage{pdflscape}
\usepackage{hhline}
\usepackage{listings}
\usepackage{lstautogobble}
\directlua{dofile('GregorioRef.lua')}

\newcommand*{\eg}{e.g.\@\xspace}
\newcommand*{\nb}{n.b.\@\xspace}
\newcommand*{\ie}{i.e.\@\xspace}
\newcommand*{\etc}{etc.\@\xspace}

\newif\ifbreakable
\let\oldsubsection\subsection
\def\subsection{\filbreak\breakablefalse\oldsubsection}
\let\oldsubsubsection\subsubsection
\def\subsubsection{\filbreak\breakablefalse\oldsubsubsection}

\newcommand{\macroname}[3]{%
	\vspace{3.25ex plus 1ex minus .2ex}%
	\ifbreakable%
		\filbreak%
	\else%
		\breakabletrue%
	\fi%
	\makebox[\linewidth]{\ttfamily\bfseries #1#2%
	\hspace{\fill}\normalfont\itshape #3}%
	\vspace{1.5ex plus .2ex}%
	\index{#1}}
	%\addcontentsline{toc}{subsubsection}{#1}}

\newcommand{\optional}[1]{{\itshape #1}}

\lstset{backgroundcolor=\color{lightgray},
				basicstyle=\small\ttfamily,
				numbers=left,
				numberstyle=\footnotesize,
				stepnumber=1,
				numbersep=5pt}

% for the character tables
\font\greciliae = {name:greciliae} at 1000000 sp\relax
\font\gregorio = {name:gregorio} at 1000000 sp\relax
\font\parmesan = {name:parmesan} at 1000000 sp\relax
\font\greciliaeOp = {name:greciliae-op} at 1000000 sp\relax
\font\gregorioOp = {name:gregorio-op} at 1000000 sp\relax
\font\parmesanOp = {name:parmesan-op} at 1000000 sp\relax
\font\greextra = {name:greextra} at 12 pt\relax
\newcommand{\excluded}[1]{{\tiny\itshape #1}}

\newenvironment{argtable}{%
	\bigskip\rowcolors{1}{lightgray}{lightgray}
	\tabularx{\textwidth}{clX}
		Arg & Value & Description \\
		\hline}%
	{\endtabularx\bigskip}

\makeatletter%
\NewDocumentEnvironment{gdimension}{m}{\macroname{#1}{}{gsp-default.tex}}{%

	\gre@rubberpermit{#1}%
	\ifgre@rubber%
		Default: \directlua{GregorioRef.emit_dimension("\luaescapestring{\csname gre@space@skip@#1\endcsname}")}
	\else%
		Default: \directlua{GregorioRef.emit_dimension("\luaescapestring{\csname gre@space@dimen@#1\endcsname}")}
	\fi%
}

\newcommand{\writemode}[3]{%
	\gre@style@modeline #1\endgre@style@modeline %
	\gre@style@modemodifier #2\endgre@style@modemodifier %
	\gre@style@modedifferentia #3\endgre@style@modedifferentia %
}
\makeatother

\setlength{\parindent}{0mm} % Default is 15pt

\begin{document}

\begin{titlepage}
	\begin{center}
		\Huge
		\textcolor{myred}{Gregorio} and \textcolor{myred}{Gregorio\TeX}:

		Tools for gregorian score engraving.

		\vspace{1cm}

		\large Version \textbf{4.1.0-rc1}, 18 February 2016 %% PARSE_VERSION_DATE

		\vspace{1.5cm}
	\end{center}
	\gresetlinecolor{gregoriocolor}
	\grechangestyle{lowchoralsign}{\cminfamily\small}%
	\grechangestyle{highchoralsign}{\cminfamily\small}%
	\def\GreStar{\greheightstar}%
	\greannotation{\scriptsize Comm.}%
	\greannotation{\scriptsize VII}%
	\begingroup%
		\color{black!60}%
		\setmainfont[SmallCapsFont=AlegreyaSC]{Alegreya}%
		\newlength{\mini}%
		\setlength{\mini}{\hsize}%
		\addtolength{\mini}{-4cm}%
		\setlength{\fboxsep}{5mm}%
		\hfill\fbox{\parbox{\mini}{\gregorioscore[f]{factus}}}\hfill%
	\endgroup
	\begin{center}
		\vspace{1.5cm}%
		\href{http://gregorio-project.github.io/}{Homepage}

		Source code available on
		\href{http://github.com/gregorio-project/gregorio}{GitHub}.
	\end{center}

	\vspace{2cm}
\end{titlepage}

\cleardoublepage

\tableofcontents

\setlength{\parskip}{\bigskipamount}
\cleardoublepage

\section{Gregorio\TeX{} Macros}
The following sections document the macros available in the Gregorio\TeX{} package. The format is as follows:

\macroname{MacroName}{\{Args\}}{Source File}
Description of macro.

\begin{argtable}
	Arg \# & Data type & Description of argument\\
	Arg \# & \texttt{keyword} & Description of the setting the keyword corresponds to
\end{argtable}

The source file where the macro is defined is included for developers
who wish to consult it.

Some of the macros intended for inclusion in the main.tex file by the user
include usage examples.

Macros are divided into three groups:
\begin{description}
\item[User Commands] These macros are meant to be used by your average user in their \TeX\ files in order to fine tune the appearance of their scores.  They should have names which consist solely of lowercase letters and be prefixed with the \texttt{gre}.  Where the name clearly identifies the function as belonging to Gregorio\TeX, the prefix may be omitted.
\item[Gregorio Controls] These macros are written by the command line tool to gtex files and should not appear outside of gtex files.  They should have names which are in CamelCase and be prefixed with \texttt{Gre}.
\item[Gregorio\TeX\ internals] These macros are used by Gregorio\TeX\ to process and typeset a score and should not appear in a user's document anywhere (not even in gtex files).  They should have names which are all lowercase and be prefixed with \makeatletter\texttt{gre@}\makeatother.
\end{description}


% !TEX root = GregorioRef.tex
% !TEX program = LuaLaTeX+se
%
% Copyright (C) 2006-2017 The Gregorio Project (see CONTRIBUTORS.md)
%
% This file is part of Gregorio.
%
% Gregorio is free software: you can redistribute it and/or modify
% it under the terms of the GNU General Public License as published by
% the Free Software Foundation, either version 3 of the License, or
% (at your option) any later version.
%
% Gregorio is distributed in the hope that it will be useful,
% but WITHOUT ANY WARRANTY; without even the implied warranty of
% MERCHANTABILITY or FITNESS FOR A PARTICULAR PURPOSE.  See the
% GNU General Public License for more details.
%
% You should have received a copy of the GNU General Public License
% along with Gregorio.  If not, see <http://www.gnu.org/licenses/>.
%
\section{User Controls}

These functions are available to the user to customize elements of the
score which cannot be controlled from the gabc file. They can be added
to any \verb=.tex= file. Do not add them to any \verb=.gtex= or
\verb=.gabc= file.

\subsection{Using the Package}

To use the Gregorio\TeX\ package in a \LaTeX\ document, include \verb=\usepackage{gregoriotex}=
in the document preamble. This macro has the following form:

\macroname{\textbackslash usepackage}{[\optional{(options)}]\{gregoriotex\}}{gregoriotex.sty}

The optional arguments are:

\bigskip\rowcolors{1}{lightgray}{lightgray}
\begin{tabular}{lp{10cm plus .5cm}}
	Argument & Description \\
	\hline
	\texttt{debug} & Debug messages will be printed to the output log.  Can also be specified as \verb:debug={<types>}:, in which case only messages of the categories (see \nameref{DebugCategory}) listed will be printed to the output log.\\
	\hline
	\texttt{nevercompile} & Default. The classic behavior of Gregorio\TeX. The user is %
		responsible for compiling gabc scores into gtex files.\\
	\texttt{autocompile} & Gregorio\TeX\ will automatically compile gtex files from gabc %
		files when necessary. If the gabc has been modified, or the %
		gtex has an outdated version, or the gtex file does not exist, %
		THEN Gregorio\TeX\ will compile a new gtex file.\\
	\texttt{forcecompile} & Gregorio\TeX\ will compile all scores from their gabc files.\\
	\hline
	\texttt{allowdeprecated=false} & Force all deprecated commands to raise a package error %
		rather than a warning. This allows the user to ensure that their file is %
		compliant with the current version of Gregorio\TeX.\\
\end{tabular}\bigskip

\textbf{Note:} \verb=nevercompile=, \verb=autocompile=, and
\verb=forcecompile= conflict with eachother. Only one should be
specified in the options list.\bigskip

To use the package in a Plain \TeX\ document, include \verb=\input gregoriotex.tex= near the top of the document (the area which would correspond to the preamble in \LaTeX).

To use the \texttt{debug} option in Plain \TeX, you'll need to define \verb=\gre@debug= manually as a string listing the kinds of messages you want printed (or as \verb=all= if you want all messages printed).

To use the \texttt{allowdeprecated=false} option, you'll need \verb=\gre@allowdeprecatedfalse=.

The compilation options can be set using \verb=\gresetcompilegabc= (see below).

\textbf{Important:} Gregorio\TeX{} may require up to two passes (runs of
\texttt{lualatex} or \texttt{luatex}) to compute the line heights correctly.  If a second
pass is required, Gregorio\TeX{} will emit the following
warning:\par\medskip

\begin{scriptsize}
\begin{latexcode}
Module gregoriotex warning: Line heights or variable brace lengths may have changed. Rerun to fix.
\end{latexcode}
\end{scriptsize}

Gregorio\TeX{} two-pass processing is compatible with \texttt{latexmk}.

If you only need the special symbols which Gregorio\TeX\ contains, and not the ability to include scores or musical glyphs, then you can load \texttt{gregoriosyms} instead of \texttt{gregoriotex}.  It supports all of the above options except those specifically related to scores.  \textbf{You should not try to load both packages}.


\subsection{Commands}

Once you've included the package in your document (as explained above) the following commands allow you to insert scores and manipulate the way they appear in the document.

\subsubsection{Including scores}

\macroname{\textbackslash gregorioscore}{[\optional{\#1}]\{\#2\}}{gregoriotex-main.tex}
Macro for including scores.  Works on both gabc and tex files.

\begin{argtable}
	\#1 & \texttt{n} & Optional. \#2 will be included as is. \\
			& \texttt{a} & Optional. Gregorio\TeX\ will automatically compile gabc files if necessary.\\
			& \texttt{f} & Optional. Forces Gregorio\TeX\ to compile the gabc file.\\
	\#2 & string & Relative or absolute path to the score.\\
\end{argtable}

Example:\par\medskip
\begin{latexcode}
	\gregorioscore[n]{TecumPrincipium.gtex}
	\gregorioscore{Chant/VirgoVirginum.gabc}
	\gregorioscore{/home/user/chant/AdTeLevavi}
	\gregorioscore[a]{AveMaria}

	%The following lines include the same score:
	\gregorioscore{Christus}
	\gregorioscore{Christus.gtex}
	\gregorioscore{./Christus}
	\gregorioscore{./Christus.gabc}

	%With the optional arg [f], #2 must be a file usable by \TeX.
	\gregorioscore[f]{TecumPrincipium.gabc} % Wrong
\end{latexcode}

\textbf{Important:} For the sake of clarity it is recommended that the
file extension be omitted from \texttt{\#2} unless using the \texttt{nevercompile} option. When the \texttt{nevercompile}
option is in effect (either via package option
\texttt{[nevercompile]}, or \verb=\gresetcompilegabc{never}=, or
\verb=\gregorioscore[n]=) \#2 must be a \TeX\ file that exists and
the file extension (normally gtex) must be given.

When called with the optional argument \texttt{[a]} Gregorio\TeX\ will
automatically generate a \texttt{gtex} file in this format:
\texttt{\textit{scorename}-x\_x\_x.gtex} where \texttt{x\_x\_x} is the
gregorio version. This resulting file is not intended to be modified
by the user and will be removed when the gabc file is recompiled. The
rules that Gregorio\TeX\ uses to determine if a gabc file needs to be
compiled are:

\begin{itemize}
\item If a gtex file does not exist.
\item If the modification time of the gabc file is newer than its
	corresponding gtex file.
\item If the version of the gtex file is outdated.
\end{itemize}

When called with the optional argument \texttt{[n]} Gregorio\TeX\ will
include the score without doing anything else. This is the same as the
old behavior of Gregorio\TeX\ and therefore the default behavior.

When called with the optional argument \texttt{[f]} Gregorio\TeX\ will
compile the gabc file into a gtex file. This is similar to
\texttt{[a]} except the gabc is compiled every time.

\macroname{\textbackslash gresetcompilegabc}{\{\#1\}}{gregoriotex-main.tex}
A macro to change the behavior of the way Gregorio\TeX\ includes scores.  This is similar to using the package options \verb=[forcecompile]=, \verb=[autocompile]=, and \verb=[nevercompile]=, but does not necessarly apply to the entire document.


\begin{argtable}
	\#1 & \texttt{force} & all later calls of \verb=\gregorioscore= will compile the gabc
file into a gtex file.\\
	& \texttt{auto} & all later calls of \verb=\gregorioscore= will use Gregorio\TeX's
automatic compilation of gabc files.\\
	& \texttt{never} & all later calls of \verb=\gregorioscore= will include the score
without doing anything else.
\end{argtable}

\medskip This macro can be combined in the same document with different arguments to
switch between different behaviors: \par\medskip
\begin{latexcode}
	\usepackage{gregoriotex} % [nevercompile] is the default.
	----
	\gregorioscore{TecumPrincipium} % gabc never compiled.
	\gregorioscore[f]{TecumPrincipium} % gabc always compiled.
	\gregorioscore[a]{TecumPrincipium} % gabc auto compiled.

	\gresetcompilegabc{auto}
	\gregorioscore{TecumPrincipium} % gabc auto compiled.
	\gregorioscore[n]{TecumPrincipium} % gabc never compiled.
	\gregorioscore[f]{TecumPrincipium} % gabc always compiled.

	\gresetcompilegabc{force}
	\gregorioscore{TecumPrincipium} % gabc always compiled.
	\gregorioscore[n]{TecumPrincipium} % gabc never compiled.
	\gregorioscore[a]{TecumPrincipium} % gabc auto compiled.
\end{latexcode}

\macroname{\textbackslash gabcsnippet}{\{\#1\}}{gregoriotex-main.tex}
Converts the gabc notation specified in \texttt{\#1} to Gregorio\TeX\ and
includes it directly in the document.

\begin{argtable}
	\#1 & string & The gabc to insert into the document.\\
\end{argtable}

\medskip For example:\par\medskip
\begin{latexcode}
	\gabcsnippet{(c3) Al(eg~)le(gv.fhg)lu(efe___)ia(e.) (::)}
\end{latexcode}


\subsubsection{Point-and-click}

Gregorio can add Lilypond-like point-and-click links into the output PDF
file.  The URLs added to the PDF conform with Lilypond and will use the
Lilypond scripts if they are enabled on your system.  To configure your
system for this feature, please see the documentation for Lilypond since
they established the feature.\bigskip

In addition to switching this feature on in \TeX{}, you must also pass the
\texttt{-p} option to \texttt{gregorio} when converting your gabc files to
Gregorio\TeX{} for inclusion.  This will automatically be done for auto- and
force-compiled scores, but if an auto-compiled score was compiled without the
option, Gregorio\TeX{} will not realize it has changed to recompile it.  In
this case, remove the corresponding \texttt{.gtex} file to force it to
recompile.\bigskip

\textbf{Important:} As with LilyPond, you should always turn off
point-and-click before producing gtex and/or PDF files for distribution.
This feature embeds absolute filenames from your computer as links in
the PDF, which can pose a security risk.  This is why this feature is
disabled by default.

\macroname{\textbackslash gresetpointandclick}{\{\#1\}}{gregoriotex-syllable.tex}
Macro to enable or disable the point-and-click feature.

\begin{argtable}
	\#1 & \texttt{on}  & Enable point-and-click link generation.\\
	& \texttt{off} & Disable point-and-click link generation (default).
\end{argtable}

This feature may be switched on and off as desired between scores.


\subsubsection{Overall Size}
While the default size for Gregorio scores is designed to approximate that found in most liturgical books, Gregorio\TeX\ also provides mechanisms for changing the size of your scores for use in any application.

\macroname{\textbackslash grechangestaffsize}{\{\#1\}}{gregoriotex-main.tex}
Macro to adjust the size of the staff.

\begin{argtable}
	\#1 & integer & The size of the staff lines.  Default value is 17.  Higher numbers yield larger staves.
\end{argtable}

\macroname{\textbackslash grechangestafflinethickness}{\{\#1\}}{gregoriotex-spaces.tex}
Macro to adjust the thickness of the staff lines.

\begin{argtable}
	\#1 & integer & The thickness of the staff lines.  The default value is same as staff size.
\end{argtable}


\subsubsection{Fine Tuning Dimensions}
In addition to providing control over the overall size of your scores, Gregorio\TeX\ allows you to fine tune the spacings around and between the various elements using the following functions.

\macroname{\textbackslash grecreatedim}{\{\#1\}\{\#2\}\{\#3\}}{gregoriotex-spaces.tex}
Macro to create one of Gregorio\TeX’s distances.  Used to initialize distances in a space configuration file.  For an example of such a file, please see \textit{gsp-default.tex}, which contains the default spacing configuration for Gregorio\TeX.

\begin{argtable}
	\#1 & string & The name of the distance to be changed.  See \nameref{distances} below.\\
	\#2 & string & The distance in string format.  \textbf{Note:} You cannot use a length register for this argument.  You \emph{must} use a string because of the way that Gregorio\TeX\ handles spaces.\\
	\#3 & \texttt{fixed} & Distance will not scale when staff size is changed.\\
	& \texttt{scalable} & Distance will scale when staff size is changed.\\
	& \texttt{inherited} & Distance will inherit its value from another distance.  When this argument is used, then \#2 should be the name of another Gregorio\TeX\ distance.
\end{argtable}

\macroname{\textbackslash grechangedim}{\{\#1\}\{\#2\}\{\#3\}}{gregoriotex-spaces.tex}
Macro to change one of Gregorio\TeX’s distances.  This function will check to make sure the distance you are trying to change exists first.

\begin{argtable}
	\#1 & string & The name of the distance to be changed.  See \nameref{distances} below.\\
	\#2 & string & The distance in string format.  \textbf{Note:} You cannot use a length register for this argument.  You \emph{must} use a string because of the way that Gregorio\TeX\ handles spaces.\\
	\#3 & \texttt{fixed} & Distance will not scale when staff size is changed.\\
	& \texttt{scalable} & Distance will scale when staff size is changed.\\
	& \texttt{inherited} & Distance will inherit its value from another distance.  When this argument is used, then \#2 should be the name of another Gregorio\TeX\ distance.
\end{argtable}

\macroname{\textbackslash grechangenextscorelinedim}{\{\#1\}\{\#2\}\{\#3\}\{\#4\}}{gregoriotex-spaces.tex}
Changes one of Gregorio\TeX’s distances for a given line in the next included score.  This works with \texttt{spaceabovelines}, \texttt{spacebeneathtext}, and \texttt{spacelinestext}.

\begin{argtable}
	\#1 & list of integers & A comma-separated list of line numbers in the next
													 score to be adjusted.\\
	\#2 & string & The name of the distance to be changed.  See
								 \nameref{distances} below.\\
	\#3 & string & The distance in string format.  \textbf{Note:} You cannot use
								 a length register for this argument.  You \emph{must} use a
								 string because of the way that Gregorio\TeX\ handles spaces.\\
	\#4 & \texttt{fixed} & Distance will not scale when staff size is changed.\\
	& \texttt{scalable} & Distance will scale when staff size is changed.\\
	& \texttt{inherited} & Distance will inherit its value from another
												 distance.  When this argument is used, then \#3 should
												 be the name of another Gregorio\TeX\ distance.
\end{argtable}

\macroname{\textbackslash grescaledim}{\{\#1\}\{\#2\}}{gregoriotex-spaces.tex}
Macro to turn on or off scaling with the staff size for a particular distance.

\begin{argtable}
	\#1 & string & The name of the distance for which scaling is to changed.  See \nameref{distances} below.\\
	\#2 & \texttt{yes}/\texttt{true}/\texttt{on}/\texttt{scalable} & Choose just one of the given keywords.  Scale the distance when changing the size of the staff.\\
	& string not in list above & Do not scale the distance when changing the size of the staff.
\end{argtable}

\textbf{Nota bene:} This macro also can be used to change whether or not the staff line thickness scales with the staff size by specifying \texttt{stafflinefactor} for the first argument.

\macroname{\textbackslash grechangecount}{\{\#1\}\{\#2\}}{gregoriotex-spaces.tex}
Macro to change one of Gregorio\TeX’s counts or penalities (numeric values).

\begin{argtable}
	\#1 & string & The name of the count to be changed.  See \nameref{counts} and \nameref{penalties} below.\\
	\#2 & integer & The new value.\\
\end{argtable}

\macroname{\textbackslash grechangenextscorelinecount}{\{\#1\}\{\#2\}\{\#3\}}{gregoriotex-spaces.tex}
Changes one of Gregorio\TeX’s counts or penalties for a given line in the next included score.

\begin{argtable}
	\#1 & list of integers & A comma-separated list of line numbers in the next
													 score to be adjusted.\\
	\#2 & string & The name of the count to be changed.  See \nameref{counts} and
								 \nameref{penalties} below.\\
	\#3 & integer & The new value.\\
\end{argtable}

\macroname{\textbackslash greloadspaceconf}{\{\#1\}}{gregoriotex-spaces.tex}
Macro to load a space configuration file.  Space configuration file names have the format \verb=gsp-identifier.tex= and must be in the same directory as your project or in your texmf directory.

\begin{argtable}
	\#1 & string & The identifier of the space configuration file.
\end{argtable}

Example:\par\medskip
\begin{latexcode}
	% loads gsp-default.tex, the default configuration file
	\greloadspaceconf{default}
	% loads a custom configuration called gsp-myspaces.tex
	\greloadspaceconf{myspaces}
\end{latexcode}

\macroname{\textbackslash greconffactor}{}{gsp-default.tex}
A count which indicates the staff size that a space configuration file is designed for.  Each space configuration file must have this value set as Gregorio\TeX\ will compare it to the current staff size to determine if the configuration file being loaded needs to be rescaled.

\macroname{\textbackslash gresetlineheightexpansion}{\{\#1\}}{gregoriotex-main.tex}
Macro to configure line height expansion behavior when notes appear
above or below the staff lines.

\begin{argtable}
		\#1 & \texttt{variable} & Expand lines within a score independently
															of each other \\
				& \texttt{uniform}  & Expand all lines within a score uniformly
\end{argtable}

By default, Gregorio\TeX{} uses \texttt{variable} line expansion.  This
produces output similar to modern liturgical books.  However, this
feature imposes a slight performance impact and typically requires a
second pass (run of \texttt{lualatex}) to get the heights right.\bigskip

The older behavior of Gregorio\TeX{}, \texttt{uniform} line expansion,
does not have this performance impact.  However, the extra space it adds
below the staff lines may look out-of-place in a section where there are
no notes below the staff lines.\bigskip

This behavior may be switched as needed within a \TeX{} document and
affects all the scores which follow.  However, if \texttt{variable} line
expansion is enabled anywhere in the document, the second pass will be
necessary.

\begin{small}
\begin{framed}
		\textit{For experts only:}\bigskip

		It is possible to suppress the line height computation and just use
		whatever has been computed from the previous run.  If you are sure
		that the score line heights haven't changed from the previous run,
		define the \verb=\greskipheightcomputation= control sequence before
		including the Gregorio\TeX{} package.  This will save a little bit
		of time per run.
\end{framed}
\end{small}

\macroname{\textbackslash gresetledgerlineheuristic}{\{\#1\}}{gregoriotex-spaces.tex}
Macro which enables or disables ledger line heuristics.  Currently, ledger
line heuristics allow Gregorio to reduce the space between a note and a
horizontal episema that surround a line on which a ledger line may appear
when the ledger line \textit{does not} appear.

\begin{argtable}
		\#1 & \texttt{enable}  & Ledger line heuristics will be used in placing
														 the horizontal episema \\
				& \texttt{disable} & Ledger line heuristics will not be used in
														 placing the horizontal episema \\
\end{argtable}

Because of the complexity of computing distances exactly, the heuristic may
guess incorrectly, causing the horizontal episema to be placed incorrectly.
This may be overridden on a note-by-note basic by using the
\texttt{[hl:\textit{n}]} and \texttt{[ll:\textit{n}]} gabc directives.  The
\texttt{hl} directive sets an explicit high ledger line (above the staff),
and the \texttt{ll} directive sets an explicit low ledger line (below the
staff).  The \texttt{\textit{n}} should be set to indicate whether the
system should act as if the ledger line exists (\texttt{1}) or not
(\texttt{0}).


\subsubsection{Staff Lines}

\macroname{\textbackslash gresetlinecolor}{\{\#1\}}{gregoriotex.sty \textup{and} gregoriotex.tex}
Macro for changing the color of the staff lines.   The two most common colors you're going to want to use are \texttt{gregoriocolor} (see \nameref{colors}) and \texttt{black} (the default).

\begin{argtable}
	\#1 & color name & The color of the staff lines
\end{argtable}

\macroname{\textbackslash gresetlines}{\{\#1\}}{gregoriotex-main.tex}
Macro for setting whether the staff lines should be rendered or not.

\begin{argtable}
	\#1 & \texttt{visible} & The staff lines should be printed (default)\\
	& \texttt{invisible} & The staff lines should not be printed
\end{argtable}

\macroname{\textbackslash gresetlinesbehindpunctumcavum}{\{\#1\}}{gregoriotex-signs.tex}
Macro for setting whether the staff lines behind a punctum cavum should be shown or not.

\begin{argtable}
	\#1 & \texttt{visible} & The staff lines behind a punctum cavum should be printed (Plain \TeX\ default)\\
	& \texttt{invisible} & The staff lines behind a punctum cavum should not be printed (\LaTeX\ default)
\end{argtable}

\macroname{\textbackslash gresetlinesbehindalteration}{\{\#1\}}{gregoriotex-signs.tex}
Macro for setting whether the staff lines behind an alternation (\ie, an accidental) should be shown or not.

\begin{argtable}
	\#1 & \texttt{visible} & The staff lines behind an alteration should be printed (Plain \TeX\ default)\\
	& \texttt{invisible} & The staff lines behind an alteration should not be printed (\LaTeX\ default)
\end{argtable}


\subsubsection{Score Font}
Gregorio\TeX\ currently supports 3 different fonts for the glyphs in a score (neumes, clefs, alterations, \etc): Greciliae (a customized version of Caeciliae by Fr.\ Matthew Spencer, OSJ), Gregorio, and Grana Padano (née Parmesan, developed for Lilypond by Juergen Reuter).

\macroname{\textbackslash gresetgregoriofont}{[\optional{\#1}]\{\#2\}}{gregoriotex-main.tex}
Set the font used for the neumes.  The optional argument \texttt{[\#1]}
may be used to specify an alternate font/rule set.  Currently, the only
available alternate font/rule set is \texttt{op} for Dominican neumes.

Note that the font will be looked up by name through luaotfload, see the documentation of luaotfload for what it implies.

\begin{argtable}
	\#1 & \textit{(omitted)} & Use the normal font and rule set (default).\\
	& \texttt{op} & Use the alternate Dominican font/rule set.\\
	\#2 & \texttt{greciliae} & Use the Greciliae font (default).\\
	& \texttt{gregorio} & Use the Gregorio font.\\
	& \texttt{granapadano} & Use the Grana Padano font.\\
\end{argtable}

\macroname{\textbackslash gresetgregoriofontscaled}{[\optional{\#1}]\{\#2\}\{\#3\}}{gregoriotex-main.tex}

This function is the same as above, with a third argument to scale the font. The fonts shipped with Gregorio do not need to use this function, but some custom fonts do. Note that you cannot use this to scale glyphs up or down, as they would not be placed correctly on the staff.

The two first arguments are the same as \texttt{\textbackslash gresetgregoriofont}. The third argument is an integer representing the scaling factor, where the one used by \texttt{\textbackslash gresetgregoriofont} is 100000.

\macroname{\textbackslash greloadholehollowfonts}{\{\#1\}}{gregoriotex-main.tex}

If set to false, will not load the \texttt{hollow} and \texttt{hole} variants of the next font to load. Use it before loading third party fonts not having these variants (rare case).

\begin{argtable}
	\#1 & string & \texttt{true} or \texttt{false}.\\
\end{argtable}

\subsubsection{Glyph Alteration}
In addition to the normal glyphs loaded by the choice of font, Gregorio\TeX\ also supports several methods for fine tuning the choice of glyphs.  Using the below functions, you can choose from alternative glyphs which are already built into Gregorio\TeX\ or import custom glyphs you have designed yourself.

\macroname{\textbackslash grechangeglyph}{\{\#1\}\{\#2\}\{\#3\}}{gregoriotex-main.tex}
Substitutes the given Gregorio\TeX\ score glyph with the specified glyph
from the specified font.

\begin{argtable}
	\#1 & string & The name of the Gregorio\TeX\ glyph to replace.\\
	\#2 & string & The name of the font to use.\\
	\#3 & number & The code point of the glyph to use.\\
	& \texttt{.}string & The name of the variant (appended to \#1) to use.\\
	& string & (any other string) The name of the glyph to use.
\end{argtable}

\medskip If \texttt{\#1} has a wildcard (a \texttt{*}) in it, then
\texttt{\#3} must start with a dot and all glyphs matching \texttt{\#1}
will be replaced with corresponding glyphs whose names have \texttt{\#3}
appended.

\medskip If \texttt{\#2} is \texttt{*}, then the substitution is assumed
to be available in all score fonts.

\medskip For example, to use the old glyphs (from Caeciliae) for the
strophicus, use the following:\par\medskip
\begin{latexcode}
	\grechangeglyph{Stropha}{greciliae}{.caeciliae}
	\grechangeglyph{StrophaAucta}{greciliae}{.caeciliae}
\end{latexcode}

\medskip To replace all torculus resupinus glyphs with their alternate
versions, use the following:\par\medskip
\begin{latexcode}
	\grechangeglyph{TorculusResupinus*}{*}{.alt}
\end{latexcode}

\macroname{\textbackslash greresetglyph}{\{\#1\}}{gregoriotex-main.tex}
Removes a Gregorio\TeX\ score glyph substitution, restoring it back to
its original form.

\begin{argtable}
	\#1 & string & The name of the Gregorio\TeX\ glyph to restore.\\
\end{argtable}

\medskip If \texttt{\#1} has a wildcard (a \texttt{*}) in it, then
all glyphs matching \texttt{\#1} will be restored.

\medskip For example, to restore the strophicus back to the new glyphs,
use the following:\par\medskip
\begin{latexcode}
	\greresetglyph{Stropha}
	\greresetglyph{StrophaAucta}
\end{latexcode}

\medskip To restore all torculus resupinus glyphs to their original
form, use the following:\par\medskip
\begin{latexcode}
	\greresetglyph{TorculusResupinus*}
\end{latexcode}

\macroname{\textbackslash grechangecavumglyph}{\{\#1\}\{\#2\}\{\#3\}[\optional{\#4}][\optional{\#5}]}{gregoriotex-main.tex}
Substitutes the given Gregorio\TeX\ score cavum glyphs with the specified glyphs
from the specified font.

\begin{argtable}
	\#1 & string & The name of the Gregorio\TeX\ glyph to replace.\\
	\#2 & string & The name of the font to use for the cavum glyph.\\
	\#3 & number & The code point of the cavum glyph to use.\\
	& \texttt{.}string & The name of the variant (appended to \#1) to use for the cavum glyph.\\
	& string & (any other string) The name of the cavum glyph to use.\\
	\#4 & string & The name of the font to use for the glyph to fill in the cavum hole.\\
	\#5 & number & The code point of the glyph to use to fill in the cavum hole.\\
	& \texttt{.}string & The name of the variant (appended to \#1) to use to fill in the the cavum hole.\\
	& string & (any other string) The name of the glyph to use to fill in the cavum hole.\\
\end{argtable}

\textbf{Nota Bene:} The usage of wildcards (\texttt{*}s) for \texttt{\#1},
\texttt{\#2}, and \texttt{\#4} is similar to \verb=\grechangeglyph=.

\macroname{\textbackslash greresetcavumglyph}{\{\#1\}}{gregoriotex-main.tex}
Removes a pair of Gregorio\TeX\ score cavum glyph substitution, restoring them
back to their original form.

\begin{argtable}
	\#1 & string & The name of the Gregorio\TeX\ cavum glyph to restore.\\
\end{argtable}

\textbf{Nota Bene:} The usage of wildcards (\texttt{*}s) for \texttt{\#1} is
similar to \verb=\greresetcavumglyph=.

\macroname{\textbackslash gredefsymbol}{\{\#1\}\{\#2\}\{\#3\}}{gregoriotex-symbols.tex}
Defines (or redefines) a \TeX\ control sequence to be a non-score symbol.
If defined this way, the symbol will scale with the text font.

\begin{argtable}
	\#1 & string & The name of the \TeX\ control sequence (without leading backslash).\\
	\#2 & string & The name of the font to use.\\
	\#3 & number & The code point of the glyph to use.\\
	& string & The name of the glyph to use.
\end{argtable}

\macroname{\textbackslash gredefsizedsymbol}{\{\#1\}\{\#2\}\{\#3\}}{gregoriotex-symbols.tex}
Defines (or redefines) a \TeX\ control sequence to be a non-score symbol
which requires a single numeric argument (in points) to which the symbol
will be scaled.

\begin{argtable}
	\#1 & string & The name of the \TeX\ control sequence (without leading backslash).\\
	\#2 & string & The name of the font to use.\\
	\#3 & number & The code point of the glyph to use.\\
	& string & The name of the glyph to use.
\end{argtable}

\macroname{\textbackslash gresethepisema}{\{\#1\}}{gregoriotex-signs.tex}
Determines whether Gregorio\TeX\ should join (bridge) horizontal episemata on adjacent notes.

\begin{argtable}
	\#1 & \texttt{bridge} & Adjacent horizontal episemata are joined together (default).\\
	& \texttt{break} & Adjacent horizontal episemata are not joined.
\end{argtable}

\macroname{\textbackslash gresetpunctumcavum}{\{\#1\}}{gregoriotex-signs.tex}
A shortcut for switching to the alternative punctum cavum and back.

\begin{argtable}
	\#1 & \texttt{alternate} & use the alternate punctum cavum\\
	& \texttt{normal} & use the normal punctum cavum
\end{argtable}

Using the alternate punctum cavum is the equivalent of issuing the following commands:

\begin{latexcode}
	\grechangeglyph{PunctumCavum}{greciliae}{.caeciliae}%
	\grechangeglyph{LineaPunctumCavum}{greciliae}{.caeciliae}%
	\grechangeglyph{PunctumCavumHole}{greciliae}{.caeciliae}%
	\grechangeglyph{LineaPunctumCavumHole}{greciliae}{.caeciliae}%
\end{latexcode}

\macroname{\textbackslash gresetglyphstyle}{\{\#1\}}{gregoriotex-chars.tex}
Gregorio\TeX\ supports several glyph styles which can be changed with this macro.  These style replace some non-note glyphs with alternatives.

\begin{argtable}
	\#1 & \texttt{default} & Use the default style\\
	& \texttt{medicaea} & Use a Medicaea style\\
	& \texttt{hufnagel} & Use the hufnagel style\\
	& \texttt{mensural} & Use the mensural style
\end{argtable}

%%%  Should there be a table here or in the appendix which shows the affected glyphs in each of the styles?


\subsubsection[Barred letters (A/, etc.)]{Barred letters (\Abar, etc.)}

\macroname{\textbackslash gresimpledefbarredsymbol}{\{\#1\}\{\#2\}}{gregoriotex-symbols.tex}
Redefines a \TeX\ control sequence to be a a barred symbol.

\begin{argtable}
	\#1 & character & must be \texttt{A}, \texttt{R}, or \texttt{V}.\\
	\#2 & dimension & how much the bar will be shifted left.\\
\end{argtable}

Gregorio\TeX\ does not have precomposed barred letters, instead, it has bars that you
can use to composed barred letters in your text font. This command is the most
simple version.

For example:

\medskip \begin{latexcode}
	\gresimpledefbarredsymbol{A}{0.3em}
\end{latexcode}

Will define \texttt{\textbackslash Abar} to be a A with a bar shifted right
of \texttt{0.3em} from the beginning of the glyph. This is the default definition
and fits well with the Linux Libertine font. If you use another font, you'll
certainly have to change this value by calling the \texttt{\textbackslash gresimpledefbarglyph} command.

\macroname{\textbackslash gredefbarredsymbol}{\{\#1\}\{\#2\}\{\#3\}\{\#4\}\{\#5\}\{\#6\}}{gregoriotex-symbols.tex}
Redefines a \TeX\ control sequence to be a barred symbol.

\begin{argtable}
	\#1 & string & the name of the command you want to define.\\
	\#2 & string & command to typeset the text.\\
	\#3 & string & symbol of the bar (must be defined through \texttt{gredefsizedsymbol}).\\
	\#4 & number & the size of greextra to use (in pt).\\
	\#5 & dimension & horizontal right shift of the bar.\\
	\#6 & dimension & vertical shift of the bar glyph.\\
\end{argtable}

This is a more complete version of the previous command, it allows you to define
barred letters with a different style. For example you can choose another bar
drawing, or take a bar more adapted to small font size.

For example:

\begin{footnotesize}
\begin{latexcode}
	\gredefbarredsymbol{RBarBold}{\textbf{R}}{greRBarSmall}{13}{1.7mm}{0.1mm}
\end{latexcode}
\end{footnotesize}

\gredefbarredsymbol{RBarBold}{\textbf{R}}{greRBarSmall}{13}{1.7mm}{0.1mm}
Will define \texttt{\textbackslash RBarBold} to be a bold \textbf{R} with
the bar made for small text (a bit bolder, named \texttt{RBarSmall} in greextra)
, at 12pt, shifted right of \texttt{1.7mm} from the beginning of the glyph, and lowered down
by \texttt{0.1mm}. The result is that \texttt{\textbackslash RBarBold} will typeset \RBarBold .

See Appendix \ref{subsec:greextra} for a list of bars and other
symbols present in the greextra font.

\macroname{\textbackslash grelatexsimpledefbarredsymbol}{\{\#1\}\{\#2\}\{\#3\}\{\#4\}\{\#5\}}{gregoriotex-symbols.tex}
Redefines a \TeX\ control sequence to be a barred symbol.

\bigskip\textbf{Only available in \LaTeX.}

\begin{argtable}
	\#1 & character & must be \texttt{A}, \texttt{R}, or \texttt{V}.\\
	\#2 & dimension & how much the bar will be shifted left when upright and medium weight.\\
	\#3 & dimension & how much the bar will be shifted left when italic/slanted and medium weight.\\
	\#4 & dimension & how much the bar will be shifted left when upright and bold.\\
	\#5 & dimension & how much the bar will be shifted left when italic/slanted and bold.\\
\end{argtable}

This is like \verb=\gresimpledefbarglyph=, but allows setting different shifts
for different font shapes and weights.  If you need something more elaborate,
you will need to redefine the bar macro(s) manually.  This macro is only
available in \LaTeX{} because it depends upon the \LaTeX{} font system.

\macroname{\textbackslash grebarredsymbol}{\{\#1\}\{\#2\}\{\#3\}\{\#4\}\{\#5\}}{gregoriotex-symbols.tex}
Generates a barred symbol.  This macro does not change any barred symbol
definitions.  Instead, it actually generates the code that would show the
barred symbol.

\begin{argtable}
	\#1 & string    & command to typeset the text.\\
	\#2 & string    & symbol of the bar (must be defined through \texttt{gredefsizedsymbol}).\\
	\#3 & number    & the size of greextra to use (in pt).\\
	\#4 & dimension & horizontal right shift of the bar.\\
	\#5 & dimension & vertical shift of the bar glyph.\\
\end{argtable}

\macroname{\textbackslash gothRbar}{}{gregoriotex-symbols.tex}
Prints \gothRbar.  Defined with \verb=\gredefsymbol=.

\macroname{\textbackslash gothVbar}{}{gregoriotex-symbols.tex}
Prints \gothVbar.  Defined with \verb=\gredefsymbol=.

\macroname{\textbackslash grealtcross}{}{gregoriotex-symbols.tex}
Prints \grealtcross.  Defined with \verb=\gredefsymbol=.

\macroname{\textbackslash grecross}{}{gregoriotex-symbols.tex}
Prints \grecross.  Defined with \verb=\gredefsymbol=.

\macroname{\textbackslash greheightstar}{}{gregoriotex-symbols.tex}
Prints \greheightstar.  Defined with \verb=\gredefsymbol=.

\macroname{\textbackslash gresixstar}{}{gregoriotex-symbols.tex}
Prints \gresixstar.  Defined with \verb=\gredefsymbol=.

\macroname{\textbackslash greseparator}{\{\#1\}\{\#2\}}{gregoriotex-symbols.tex}
A macro for invoking one of the five separators (fancy lines) which are contained in the greextra font.

\begin{argtable}
	\#1 & \texttt{1}--\texttt{5} & Choose the number of the line desired\\
	\#2 & integer & the point size at which to print the line\\
\end{argtable}

\macroname{\textbackslash greornamentation}{\{\#1\}\{\#2\}}{gregoriotex-symbols.tex}
A macro for invoking one of the ornamentation elements which are contained in the greextra font.

\begin{argtable}
	\#1 & \texttt{1}--\texttt{2} & Choose the number of the ornamentation desired\\
	\#2 & integer & the point size at which to print the line\\
\end{argtable}


\subsubsection{Special Characters}

% this is defined in gregoriotex-symbols.texx as having one argument, but the
% lua code generates a \gdef\<something> which then requires a second argument
\macroname{\textbackslash gresetspecial}{\{\#1\}\{\#2\}}{gregoriotex-symbols.tex}
Sets a special character.  Special characters are used from gabc within
\texttt{<sp>} and \texttt{</sp>}.

\begin{argtable}
	\#1 & string & The text between \texttt{<sp>} and \texttt{</sp>}.\\
	\#2 & \TeX\ code & The \TeX\ code to substitute when \texttt{<sp>\#1</sp>}
										 is used in gabc.\\
\end{argtable}

\textbf{Nota Bene:} If you need to use a character in \#1 that is made special
by \TeX{} (\ie, \textbackslash, \%, \etc), you should instead use
\verb=\string\nnn=, where \texttt{nnn} is a three-digit, zero-padded number
representing the ASCII code of the character (\ie, \textbackslash{} would be
\verb=\string\092=).

\macroname{\textbackslash greunsetspecial}{\{\#1\}}{gregoriotex-symbols.tex}
Un-sets a special character.  Using an unset special character will use its
text directly.

\begin{argtable}
	\#1 & string & The text between \texttt{<sp>} and \texttt{</sp>}.\\
\end{argtable}

\textbf{Nota Bene:} The same rules apply for \#1 as in \verb=\gresetspecial=.

\macroname{\textbackslash gretilde}{}{gregoriotex-main.tex}
Macro to print $\sim$.  This macro is set using the above for \texttt{<sp>$\sim$</sp>}.



\subsubsection{Styling}
Different elements of an include score have different styles applied.  These elements and their defaults are listed below:

\begin{adjustbox}{center}
\let\stylename\texttt
\renewcommand{\thefootnote}{\fnsymbol{footnote}}
\bigskip\rowcolors{1}{lightgray}{lightgray}
\begin{tabular}{lp{7cm plus .5cm}r}
	Element Name & Description & Default\\
	\hline
	\stylename{abovelinestext} & above line text (\texttt{<alt></alt>} in gabc, appears above the staff) & normal\\
	\stylename{additionalstafflines} & short lines behind notes above or below the staff & special\footnotemark[1]\\
	\stylename{annotation} & the annotation & none\\
	\stylename{commentary} & the commentary & {\footnotesize\it footnote-size italics} (\LaTeX)\\
	&& {\textit{italics}} (Plain \TeX)\\
	\stylename{elision} & elisions (\texttt{<e></e>} in gabc) & {\textit{\small small-size italics}} (\LaTeX)\\
											&& {\textit{italics}} (Plain \TeX)\\
	\stylename{firstsyllable} & the first syllable of the score excluding the score initial & none\\
	\stylename{firstsyllableinitial} & the first letter of the first syllable of a score which is not the score initial & none\\
	\stylename{firstword} & the first word of the first score excluding the score initial & none\\
	\stylename{highchoralsign} & high choral signs & none\\
	\stylename{initial} & Score initial (the first letter of the score, when offset from the rest of the text) & 40 pt font\\
	\stylename{lowchoralsign} & low choral signs & none\\
	\stylename{modedifferentia} & the rendered annotation from the \texttt{mode-differentia: ;} header in the gabc file & \parbox[t]{2.2cm}{\raggedleft\textbf{bold}}\\
	\stylename{modeline} & the rendered annotation from the \texttt{mode: ;} header in the gabc file & \parbox[t]{2.2cm}{\raggedleft\textsc{\textbf{bold small capitals}}} (\LaTeX)\\
	& & \textbf{bold} (Plain\TeX)\\
	\stylename{modemodifier} & the rendered annotation from the \texttt{mode-modifier: ;} header in the gabc file & \parbox[t]{2.2cm}{\raggedleft\textit{\textbf{bold italics}}}\\
	\stylename{nabc} & ancient notation & {\color{gregoriocolor}gregoriocolor}\\
	\stylename{normalstafflines} & Full length staff lines & none\\
	\stylename{translation} & Translation text (appears below lyrics) & {\it italics}\\
\end{tabular}
\end{adjustbox}

\footnotemark[1]\textit{Special:} By default, \texttt{additionalstafflines} inherits its properties from \texttt{normalstafflines}.  To decouple these environments, you must manually change \texttt{additionalstafflines} using \texttt{\textbackslash grechangestyle}.
\renewcommand{\thefootnote}{\arabic{footnote}}\breakabletrue

\macroname{\textbackslash grechangestyle}{\{\#1\}\{\#2\}[\optional{\#3}]}{gregoriotex.sty \textup{and} gregoriotex.tex}
Command to change styling of a score element.

\begin{argtable}
	\#1 & string & element whose styling is to be changed (see list above for options)\\
	\#2 & \TeX\ code & the code necessary to turn on the styling\\
	\#3 & \TeX\ code & Optional.  The code necessary to turn off the styling (\eg, if the code to turn on the styling contains a \verb=\begin{environment}= then the code to turn it off must have the matching \verb=\end{environment}=.
\end{argtable}

Examples:\par\medskip
\begin{latexcode}
	% This one works for both PlainTeX and LaTeX this would make
	% the translations bold and italic
	\grechangestyle{translation}{\it\bf}

	% This one is LaTeX only, and would make the above lines
	% text small and italic
	\grechangestyle{abovelinestext}{\begin{small}\begin{itshape}}%
		[\end{itshape}\end{small}]

	% This would make the initial print in 36pt font.
	\grechangestyle{initial}{\fontsize{36}{36}\selectfont}
\end{latexcode}

\bigskip
Each element will be typeset within an isolated group to prevent styling commands from leaking from one element to the next.  As a result, if a styling command has an ``on-switch'' but no ``off-switch'' (like \verb=\it= or \verb=\bf= in the first example above) it is not necessary to encapsulate them within \verb=\begingroup= and \verb=\endgroup=.  As a result, the third argument is only necessary for styling commands which come in pairs (like the environments in the second example).

\subsubsection{Text Elements}
While the gabc headers provide support for some of the text elements commonly found on chant scores, Gregorio\TeX\ provides the following functions to allow you to enter and control those elements with a greater degree of precision than the gabc headers.

\macroname{\textbackslash greannotation}{[\optional{\#1}]\{\#2\}}{gregoriotex-main.tex}
Macro to add annotations (the text which appears above the initial) to a score.  While a single call of the function does not support multiple lines, successive calls to the function will be added to the annotation as a new line below what is already there.

\begin{argtable}
	\#1 & \texttt{c} & When adding a new line, align the center of the new line with the center of the existing lines\\
	& \texttt{l} & When adding a new line, align the left side of the new line with the left side of the existing lines\\
	& \texttt{r} & When adding a new line, align the right side of the new line with the right side of the existing lines\\
	\#2 & string & the text of the annotation
\end{argtable}

\textbf{Nota Bene:} The first argument does not affect the alignment of lines already in the annotation, only the way the new line aligns with the existing lines as a whole.

\macroname{\textbackslash grecommentary}{[\optional{\#1}]\{\#2\}}{gregoriotex-main.tex}
Macro to add commentary (the text flush right at the top, usually a scripture reference) to a score.  While a single call of this function does not support multiple lines, successive calls to the function will add a new line to the commentary directly below the previous.

\begin{argtable}
	\#1 & distance & Optional.  Additional distance to be placed between the commentary and the top staff line for the next score only.\\
	\#2 & string & The text of the commentary.\\
\end{argtable}

\textbf{Nota Bene:} If your commentary is multi-lined, then the optional argument of the last line, and only the last line, will be taken into account.  Further, pay attention to the fact that the optional argument is \emph{additional} distance, \ie, it will be added to \texttt{commentaryraise} to determine the distance from the baseline of the commentary to the top line of the staff.

\macroname{\textbackslash greillumination}{\{\#1\}}{gregoriotex-main.tex}
Macro to add an illuminated initial.

\begin{argtable}
	\#1 & \TeX\ code & the code necessary to make the illuminated initial appear\\
\end{argtable}

\textbf{Nota Bene:} Usually the argument of this command should be an \verb=\includegraphics= command, but you may use what ever you want as the illuminated initial.

\macroname{\textbackslash gresetinitiallines}{\{\#1\}}{gregoriotex-syllable.tex}
Sets the number of lines the score initial requires.

\begin{argtable}
	\#1 & number & The number of lines required by the initial.  If \texttt{0}, the score will have no separated initial.\\
\end{argtable}

\textbf{Nota Bene:} As currently implemented, you cannot set an initial which is larger than 2 lines and in order to do so you must set manual line breaks in the gabc for the first two lines.

\macroname{\textbackslash gresetmodenumbersystem}{\{\#1\}}{gregoriotex-main.tex}
Sets the number system used for the mode number.

\begin{argtable}
	\#1 & \texttt{roman-minuscule} & Use lower-case Roman numerals (the default in \LaTeX, good for small capitals).\\
			& \texttt{roman-majuscule} & Use upper-case Roman numerals (the default in Plain\TeX).\\
			& \texttt{arabic} & Use Arabic numerals.\\
\end{argtable}

\macroname{\textbackslash gresetlyrics}{\{\#1\}}{gregoriotex-syllable.tex}
Sets the visibility of the lyrics.

\begin{argtable}
	\#1 & \texttt{visible} & Lyrics are visible (default).\\
	& \texttt{invisible} & Lyrics are not visible.\\
\end{argtable}


\subsubsection{Text Alignment}
Gregorio\TeX\ allows you to manipulate the global alignment behavior of some text elements using the following commands.

\macroname{\textbackslash gresetlyriccentering}{\{\#1\}}{gregoriotex-syllable.tex}
Macro to set how the text of the lyrics aligns with the alignment point of its respective neumes.  The alignment point of the neumes is determined as follows:

\begin{itemize}
\item If the first glyph is only one note, or is a normal pes, or is composed of three or more notes, the alignment point is the middle of the first note.
\item If the first glyph is composed of two notes (other than a normal pes), the alignment point is the middle of the glyph.
\item In the case of a porrectus, the alignment point is the middle of an imaginary square punctum beginning at the same point as the porrectus.
\end{itemize}

\begin{argtable}
	\#1 & \texttt{vowel} & The center of the vowel in the syllable will align with the alignment point of the neumes\\
	& \texttt{syllable} & The center of the syllable will align with the alignment point of the neumes\\
	& \texttt{firstletter} & The center of the first letter/character of the syllable will align with the alignment point of the neumes\\
\end{argtable}

\textbf{Nota Bene:} What constitutes the ``vowel'' of the syllable is determined by the language the lyric text is written in, as specified by the use of the \texttt{language} header in the gabc file.  Out of the box, Gregorio\TeX\ explicitly supports only Latin and English, but the rules for Latin have a high degree of overlap with many Romance languages, allowing them to fall back on the Latin rules with acceptable results.

You can also define your own languages in \texttt{gregorio-vowels.dat}.  If you do define a language, please consider sharing your work by submitting it to the project (see CONTRIBUTING.md for instructions).

Finally, in cases where you want some sort of exceptional alignment, you can force Gregorio to consider a particular part of the syllable to be the ``vowel'' by enclosing it in curly braces (``\{'' and ``\}'') in your gabc file.  Curly braces only affect alignment when using vowel centering.  Syllable centering will always use the entire syllable, and firstletter centering will always use the first character of the syllable --- regardless of curly braces in the gabc file.

\macroname{\textbackslash gresetgabcforcecenters}{\{\#1\}}{gregoriotex-syllable.tex}
Macro to determine whether a forced center (\ie, one specified by curly braces (``\{'' and ``\}'') in your gabc file) should influence the alignment of that syllable when \texttt{syllable} and \texttt{firstletter} alignments are in effect.

\begin{argtable}
	\#1 & \texttt{allow} & Forced centers in gabc are allowed to influence the syllable alignment (default).\\
	& \texttt{prohibit} & Forced centers in gabc do not influence the syllable alignment.\\
\end{argtable}

\macroname{\textbackslash gresettranslationcentering}{\{\#1\}}{gregoriotex-main.tex}
Macro to specify how the translation text should be aligned with it respective syllable text.

\begin{argtable}
	\#1 & \texttt{left} & The translation text is left aligned with its respective syllable text.\\
	& \texttt{center} & The translation text is centered under its respective syllable.
\end{argtable}

\macroname{\textbackslash gresetannotationby}{\{\#1\}}{gregoriotex-main.tex}
Macro to specify which line of the annotation should be used to determine its starting placement (i.e. before \texttt{annotationraise} is applied).

\begin{argtable}
	\#1 & \texttt{firstline} & Annotation placement is determined by the first line (default)\\
	& \texttt{lastline} & annotation placement is determined by the last line\\
\end{argtable}

\macroname{\textbackslash gresetannotationvalign}{\{\#1\}}{gregoriotex-main.tex}
Macro to specify which part of the control line in the annotation should be aligned with the top line of the staff before \texttt{annotationraise} is applied.

\begin{argtable}
	\#1 & \texttt{top} & The top of the annotation control line will align with the top line of the staff\\
	& \texttt{baseline} & The baseline of the control line is used (default)\\
	& \texttt{bottom} & The bottom of the control line is used\\
\end{argtable}

\textbf{Nota Bene:} These variable refer to the actual contents of the line and not to the ``hypothetical'' limits for the font.  As a result if the top of an annotation containing only short letters will be different from one which contains tall ones even if both use the same font.  Likewise, if the annotation contains no descenders, then baseline and bottom will be the same.  If this is a problem, then the use of struts within the annotation can be used to control the line height (distance from baseline to top) and depth (distance from baseline to bottom).

\macroname{\textbackslash gresetsyllablerewriting}{\{\#1\}}{gregoriotex-syllable.tex}
Sets whether the last part of a non-final syllable of a word is moved to the
next syllable when there is no hyphen.  The ``last part'' of a syllable is
the part that comes after the part that is centered under the first note of
the syllable.  This feature may allow Lua\TeX{} to find better opportunities
for ligaturing based on \TeX{} and font settings.

\begin{argtable}
	\#1 & \texttt{auto} & Gregorio\TeX{} will move the last part of a syllable
												to the next syllable in a word when there is no
												hyphen (default).\\
			& \texttt{off}  & Gregorio\TeX{} will not attempt to rewrite any
												syllables.\\
\end{argtable}

\macroname{\textbackslash gresetprotrusionfactor}{\{\#1\}\{\#2\}}{gregoriotex-spaces.tex}
Sets a global protrusion factor.  Depending on the first argument, these
protrusion factors will be used for various characters as well as for
\verb=<pr>= tags with no specified protrusion factor.  A protrusion factor of 0
means no protrusion and 1 means full protrusion.  Any floating-point value from
0 to 1 is allowed.  All of these global protrusion factors may be set in
gsp-default.tex or in your own \TeX\ files.

\begin{argtable}
	\#1 & \texttt{,}         & Sets the automatic protrusion factor for a comma at
														 the end of a syllable.  Default is
														 \GreProtrusionFactor{comma}.\\
			& \texttt{;}         & Sets the automatic protrusion factor for a
														 semicolon at the end of a syllable.  Default is
														 \GreProtrusionFactor{semicolon}.\\
			& \texttt{:}         & Sets the automatic protrusion factor for a colon at
														 the end of a syllable.  Default is
														 \GreProtrusionFactor{colon}.\\
			& \texttt{.}         & Sets the automatic protrusion factor for a period
														 at the end of a syllable.  Default is
														 \GreProtrusionFactor{period}.\\
			& \texttt{eolhyphen} & Sets the protrusion factor for a hyphen at
														 the end of a line.  Default is
														 \GreProtrusionFactor{eolhyphen}.  This protrusion
														 factor only applies to hyphens inserted by the Lua
														 pass), so use it with caution.\\
			& \texttt{default}   & Sets the default protrusion factor for a
														 \verb=<pr>= tag in gabc.  Default is
														 \GreProtrusionFactor{default}.\\
	\#2 & factor             & The desired protrusion factor, a floating point
														 value from 0 (no protrusion) to 1 (full
														 protrusion).  See defaults above.\\
\end{argtable}

\subsubsection{End of Line Behavior}
While Gregorio\TeX\ will automatically wrap scores to fit your page, there are several ways to fine tune that line breaking behavior with the following commands.

\macroname{\textbackslash gresetbreakbeforeeuouae}{\{\#1\}}{gregoriotex-main.tex}
Macro to determine whether an automatic linebreak before a EUOUAE area is justified or not.

\begin{argtable}
	\#1 & \texttt{justified} & Automatic line breaks before EUOUAE areas should be justified (default)\\
	& \texttt{ragged} & Automatic line breaks before EUOUAE areas should be ragged\\
\end{argtable}

\textbf{Important:} When set to \texttt{ragged}, Gregorio\TeX{} will require a
second pass (run of \texttt{lualatex} or \texttt{luatex}) to typeset the line
endings correctly.  When an additional pass is required, Gregorio\TeX{} will
emit the following warning:\par\medskip

\begin{scriptsize}
\begin{latexcode}
Module gregoriotex warning: Line heights or variable brace lengths may have changed. Rerun to fix.
\end{latexcode}
\end{scriptsize}

\macroname{\textbackslash gresetbreakineuouae}{\{\#1\}}{gregoriotex-main.tex}
Macro to determine whether line breaks are allowed inside a EUOUAE area (delimited by \texttt{<eu></eu>} tags in gabc).

\begin{argtable}
	\#1 & \texttt{allow} & Line breaks are allowed\\
	& \texttt{prohibit} & Line breaks are prohibited, the entire EUOUAE area should appear on one line
\end{argtable}

\macroname{\textbackslash gresetbreakintranslation}{\{\#1\}}{gregoriotex-main.tex}
Macro to determine whether line breaks are allowed inside a translation.

\begin{argtable}
	\#1 & \texttt{allow} & Line breaks are allowed\\
	& \texttt{prohibit} & Line breaks are prohibited, the entire translation should appear on one line
\end{argtable}

\macroname{\textbackslash gresetcustosalteration}{\{\#1\}}{gregoriotex-signs.tex}
Macro for setting whether an alteration (flat, sharp, or natural) should be
rendered before a custos or not.

\begin{argtable}
	\#1 & \texttt{visible} & The custos alteration should be printed (default)\\
			& \texttt{invisible} & The custos alteration should not be printed
\end{argtable}

\macroname{\textbackslash greseteolcustos}{\{\#1\}}{gregoriotex-main.tex}
Macro to determine whether Gregorio\TeX\ should automatically place the custos at a line break.

\begin{argtable}
	\#1 & \texttt{auto} & Custos will be automatically placed at each line break\\
	& \texttt{manual} & Custos will only be placed at line breaks if they are specified in the gabc (\eg \texttt{(g+z)})
\end{argtable}

\textbf{Nota Bene:} This command only effects the custos that appears at the end of a line.  Custos which are placed at a key change are unaffected.  Further, if custos are specified in the gabc file manually and Gregorio\TeX\ is set to place custos automatically, you will get two custos at the line breaks.

\macroname{\textbackslash greseteolcustosbeforeeuouae}{\{\#1\}}{gregoriotex-main.tex}
Macro to determine whether Gregorio\TeX\ should automatically place the custos at a line break before a EUOUAE.  Since the EUOUAE block is typically not a continuation of the melody but rather a reminder of the end of the tone that follows, this is set to \texttt{suppressed} (no custos) by default.

\begin{argtable}
	\#1 & \texttt{suppressed} & Custos will not automatically be placed at a line break before a EUOUAE block (the default)\\
	& \texttt{auto} & Custos will behave according to \verb=greseteolcustos= at a line break before a EUOUAE block\\
\end{argtable}

\textbf{Nota Bene:} If \verb=\greseteolcustos= is set to \texttt{manual}, this setting is effectively ignored.

\macroname{\textbackslash greseteolshifts}{\{\#1\}}{gregoriotex-main.tex}
Macro to determine whether Gregorio\TeX\ should apply a small shift at the end of each line which allows lyrics to stretch under the final custos.

\begin{argtable}
	\#1 & \texttt{enable} & The shifts are applied (default)\\
	& \texttt{disable} & The shifts are not applied.
\end{argtable}

\macroname{\textbackslash gresetbolshifts}{\{\#1\}}{gregoriotex-main.tex}
Macro to determine whether Gregorio\TeX\ should apply a small shift at the beginning of each line so that lines are aligned on the notes rather than the syllable text.

\begin{argtable}
	\#1 & \texttt{enable} & The shifts are applied (default)\\
	& \texttt{disable} & The shifts are not applied.
\end{argtable}

\macroname{\textbackslash grebolshiftcleftype}{\{\#1\}}{gregoriotex-spaces.tex}
Macro to determine how notes should be left aligned in the case where clefs of different widths appear in the same score.

\begin{argtable}
	\#1 & \texttt{largest} & The notes are aligned as if all clefs had the width of the largest clef (default)\\
	& \texttt{current} & The notes are aligned on the current clef, which leads to unaligned notes. This was the default of Gregorio < \texttt{5.0}.
\end{argtable}

\macroname{\textbackslash grelocalbolshiftcleftype}{\{\#1\}}{gregoriotex-spaces.tex}
Equivalent of \verb=\grebolshiftcleftype= but valid only until the next end of a score, and with more options. This can be used before a score or even inside a \verb=<v>verbatim</v>= in gabc for corner cases like different alignment on a score taking two pages.

\begin{argtable}
	\#1 & \texttt{largest} & The notes are aligned as if all clefs had the width of the largest clef (default)\\
	& \texttt{current} & The notes are aligned on the current clef, which leads to unaligned notes\\
	& \texttt{f} & Force left alignment of notes as if all clef were f clef\\
	& \texttt{c} & Idem with c clef\\
	& \texttt{fb} & Idem with flatted f clef\\
	& \texttt{cb} & Idem with flatted c clef\\
\end{argtable}

\macroname{\textbackslash gresetlastline}{\{\#1\}}{gregoriotex-main.tex}
Macro to determine whether the last line of the score should be justified or not.

\begin{argtable}
	\#1 & \texttt{justified} & Set the last line justified with the rest of the score\\
	& \texttt{ragged} & Set the last line ragged (default)
\end{argtable}

\macroname{\textbackslash gresetunbreakablesyllablenotes}{\{\#1\}\{\#2\}\{\#3\}}{gregoriotex-syllable.tex}
Configures how notes should be kept together on line breaks.

\begin{argtable}
	\#1 & integer & The minimum number of notes in the syllable before the
									syllable may be broken across lines.  Defaults to
									\getgrecount{unbreakabletotalnotes}.\\
	\#2 & integer & The minimum number of notes at the start of a syllable that
									must be kept together when the syllable is broken across
									lines.  Defaults to \getgrecount{unbreakableinitialnotes}.\\
	\#3 & integer & The minimum number of notes at the end of a syllable that
									must be kept together when the syllable is broken across
									lines.  Defaults to \getgrecount{unbreakablefinalnotes}.\\
\end{argtable}


\subsubsection{Bar spacing}

\macroname{\textbackslash gresetshiftaftermora}{\{\#1\}}{gregoriotex-signs.tex}
Macro to change the behaviour for separation between notes of two syllables when the first ends with a punctum mora. The argument changes the cases in which punctum mora are ignored in space computation:

\begin{argtable}
	\#1 & \texttt{always} & punctum mora are always ignored (default)\\
	& \texttt{notesonly} & punctum mora are ignored before notes, not bars\\
	& \texttt{barsonly} & punctum mora are ignored before bars, not notes\\
	& \texttt{notextonly} & punctum mora are ignored only before bars inside syllables, or bars having their own syllable without text\\
	& \texttt{insideonly} & punctum mora are ignored only before bars inside syllables\\
	& \texttt{never} & punctum mora are never ignored\\
\end{argtable}

When a punctum mora is ignored, the bar will also be shifted by \texttt{moraadjustmentbar} (zero by default), see its description in the \nameref{distances} section.

\macroname{\textbackslash gresetbarspacing}{\{\#1\}}{gregoriotex-syllable.tex}
Macro to activate the new bar spacing algorithm.  The new algorithm attempts to place the bar line exactly midway between its surrounding notes.  Any text associated with the bar is placed midway between its surrounding text.  Since this might result in the bar line and the text being widely separated, there are also a limits to the distance between their respective centers: \texttt{maxbaroffsettextleft} and \texttt{maxbaroffsettextright} (when text center is respectively on the left or on the right of bar center).

\begin{argtable}
	\#1 & \texttt{new} & Activates the new spacing algorithm (Default)\\
	& \texttt{old} & Activates the old behavior\\
\end{argtable}

\subsubsection{Sign printing}

\macroname{\textbackslash gresetnotes}{\{\#1\}}{gregoriotex-syllable.tex}
Sets the visibility of the notes.

\begin{argtable}
	\#1 & \texttt{visible} & Notes are visible (default).\\
	& \texttt{invisible} & Notes are not visible.\\
\end{argtable}

\textbf{Nota Bene:} If the notes are set to be invisible, then bar lines, rythmic signs, and the like will also be invisible.  However, the staff lines and clefs will still show up (since their visibility is controlled by other settings).

\macroname{\textbackslash gresetnabc}{\{\#1\}\{\#2\}}{gregoriotex-nabc.tex}
Sets the visibility of a nabc voice.

\begin{argtable}
	\#1 & integer & The nabc voice number.\\
	\#2 & \texttt{visible} & Notes are visible (default).\\
	& \texttt{invisible} & Notes are not visible.\\
\end{argtable}

\macroname{\textbackslash greprintsigns}{\{\#1\}\{\#2\}}{gregoriotex-signs.tex}
Macro to prevent rythmic signs from printing (all signs are printed by default):

\begin{argtable}
	\#1 & \texttt{vepisema} & sets the printing of vertical episema\\
	& \texttt{hepisema} & sets the printing of horizontal episema\\
	& \texttt{mora} & sets the printing of punctum mora and auctum duplex\\
	& \texttt{all} & set the printing of all of these\\
	\#2 & \texttt{enable} & enable the printing\\
	& \texttt{disable} & disable the printing\\
\end{argtable}

Note that punctum mora and auctum duplex have an influence on spacings, so removing them will have an impact on that matter.

\subsubsection{Hyphenation}

\macroname{\textbackslash gresethyphen}{\{\#1\}}{gregoriotex-main.tex}
Tells Gregorio\TeX\ how to place a hyphen between syllables in polysyllabic words in a score.

\begin{argtable}
	\#1 & \texttt{force} & Hyphens will appear between all syllables in polysyllabic words.\\
	& \texttt{auto} & Hyphens will appear based on the setting of \texttt{maximumspacewithoutdash} (default)
\end{argtable}

\macroname{\textbackslash gresetemptyfirstsyllablehyphen}{\{\#1\}}{gregoriotex-syllable.tex}
Tells Gregorio\TeX\ how to place a hyphen after an empty first syllable (\ie, when the first syllable consists only of the big initial).

\begin{argtable}
	\#1 & \texttt{force} & A hyphen will appear after an empty first syllable. (default)\\
	& \texttt{auto} & A hyphen will appear after an empty first syllable based on the setting of \texttt{maximumspacewithoutdash}
\end{argtable}

\macroname{\textbackslash greseteolhyphen}{\{\#1\}}{gregoriotex-main.tex}
Marco to determine how much space the hyphen at the end of a line occupies for the purposes of spacing calculations (the visible appearance of the hyphen is unchanged).

\begin{argtable}
	\#1 & \texttt{normal} & The hyphen occupies its normal space\\
	& \texttt{zero} & The hyphen is considered to take up no space
\end{argtable}

\subsubsection{Clef Visibility}

\macroname{\textbackslash gresetclef}{\{\#1\}}{gregoriotex-signs.tex}
Macro to tell Gregorio\TeX\ whether the clefs should be printed or not.

\begin{argtable}
	\#1 & \texttt{visible} & Clefs will be printed (default)\\
	& \texttt{invisible} & Clefs will not be printed
\end{argtable}


\subsubsection{Clivis Alignment}
Since the center of the clivis is different from most neumes, Gregorio\TeX\ supports several behaviors for determining how to align it with its lyrics.

\macroname{\textbackslash gresetclivisalignment}{\{\#1\}}{gregoriotex-syllable.tex}
Macro to determine the method used for aligning the clivis with its lyrics.

\begin{argtable}
	\#1 & \texttt{always} & Align on the real center of the clivis\\
	& \texttt{never} & align on the center of the first punctum in the clivis\\
	& \texttt{special} & align on the real center of the clivis except when (1) notes would go left of text or (2) consonants after vowels are larger than \verb=\gre@dimen@clivisalignmentmin= (default)
\end{argtable}



\subsubsection{Braces}

\macroname{\textbackslash gresetbracerendering}{[\optional{\#1}]\{\#2\}}{gregoriotex-signs.tex}
Macro to tell Gregorio\TeX{} whether to use \MP{} or fonts to render
braces.  \MP{} braces, the default, are tailored to better maintain
optical line weight when stretched.  \MP{} braces are designed to
harmonize (and thus match best) with greciliae, but they still look good
with the other score fonts.

\begin{argtable}
	\#1 & \textit{(omitted)} & change all braces\\
			& \texttt{brace} & change round braces that appear over the staff\\
			& \texttt{underbrace} & change round braces that appear under the staff\\
			& \texttt{curlybrace} & change curly braces\\
			& \texttt{barbrace} & change round braces that appear over divisio bars\\
	\#2 & \texttt{metapost} & \MP{} will be used to render braces\\
			& \texttt{font} & The score font will be used to render braces\\
\end{argtable}

\macroname{\textbackslash grebarbracewidth}{}{gregoriotex-signs.tex}
Returns the em-relative width of a bar brace when braces are rendered by
\MP{} (as opposed to fonts).  The value is scaled by the Gregorio\TeX\ score
size factor and thus is a score-relative value with a precise (but
obscure) mathematical meaning.  Suffice it to say that larger numbers
make the bar brace wider and smaller numbers make the brace narrower.
This must be a positive number, defaults to \texttt{.58879}, and
harmonizes with the greciliae font.  This macro must be redefined should
a different value be desired.


\subsubsection{Headers}

\macroname{\textbackslash gresetheadercapture}{\{\#1\}\{\#2\}\{\#3\}}{gregoriotex-main.tex}
Macro to tell Gregorio\TeX{} to capture a given header of the gabc file, passing it to a
specified \TeX{} macro.  Passing an empty \#2 will cancel capture of the given header.

\begin{argtable}
	\#1 & string & The name of the gabc header\\
	\#2 & string & The name of the macro to use (without the leading backslash)
								 or empty to stop capturing the given header\\
	\#3 & string & a comma-separated list of options\\
\end{argtable}

The options are:

\begin{tabular}{ll}
	\texttt{name}   & The header name should also be passed to the macro\\
	\texttt{string} & The header value should be passed to the macro as a string\\
\end{tabular}

If the \texttt{name} option is not supplied, the macro is called with one
argument: the value of the header.

If the \texttt{name} option is supplied, the macro is called with two
arguments: the name and the value of the header (in that order).

If the \texttt{string} option is supplied, the value will be passed with
catcode 12 associated with all non-space characters (and catcode 10 for all spaces).
If not, the value will be evaluated as regular \TeX{} input.

Other than the headers that define macros, which are not passed to \TeX{},
the headers will be processed in the order they were presented in the gabc
file.  Headers will be processed in the \TeX{} state at the point of the
\verb=\gregorioscore= call.  This means, for example, that should the
capturing macro produce something, it will be typeset within the same
paragraph as the \verb=\gregorioscore= call.

As an example, you can use

\verb=\gresetheadercapture{commentary}{grecommentary}{string}=

\noindent to capture the
\texttt{commentary} header of gabc files and feed it to \verb=\grecommentary=,
thus automatically printing the content of the header above the score.

\macroname{\textbackslash grebeforeheaders}{\{\#1\}}{gregoriotex-main.tex}
Specifies \TeX{} code processed before the processing of the headers of a score.
Defaults to nothing.  If this is called multiple times, the most recent call
will define the behavior at the next set of headers.

\begin{argtable}
		\#1 & \TeX\ code & The code to process before a set of headers.\\
\end{argtable}

\macroname{\textbackslash greafterheaders}{\{\#1\}}{gregoriotex-main.tex}
Specifies \TeX{} code processed after the processing of the headers of a score.
Defaults to nothing.  If this is called multiple times, the most recent call
will define the behavior at the next set of headers.

\begin{argtable}
		\#1 & \TeX\ code & The code to process after a set of headers.\\
\end{argtable}


\subsubsection{Ancient Notation}
For a full description of how to make use of the ancient notation capabilities of Gregorio and Gregorio\TeX, look at the GregorioNabcRef documentation.  The commands listed here allow the manipulation of settings related to that notation.

\macroname{\textbackslash gresetnabcfont}{\{\#1\}\{\#2\}}{gregoriotex-nabc.tex}
Macro to set the font to be used for the ancient notation.

\begin{argtable}
	\#1 & string & the name of the font\\
	\#2 & integer & point size at which the font should be loaded\\
\end{argtable}

\subsection{Counts}\label{counts}

Each of the following counts controls some aspect of the configuration of the Gregorio\TeX\ score.  They are changed using \verb=\grechangecount=, documented above.

\begin{gcount}{additionaltopspacethreshold}
The threshold above which we start accounting notes above lines for additional
vertical space. For instance with a threshold of \texttt{2} and four line
staves, notes with a pitch of \texttt{k} and \texttt{l} will not interfere with
the space above lines.  Set it to a high value if you don't want high notes to
interfere with space above lines.
\end{gcount}

\begin{gcount}{additionaltopspacealtthreshold}
Same as \texttt{additionaltopspacethreshold} but setting the threshold for
notes taken into account with above lines text vertical placement.
\end{gcount}

\begin{gcount}{additionaltopspacenabcthreshold}
Same as \texttt{additionaltopspacethreshold} but setting the threshold for
notes taken into account with above lines nabc neume vertical placement
baseline.
\end{gcount}

\begin{gcount}{noteadditionalspacelinestextthreshold}
The number of low notes which will add on the
\texttt{noteadditionalspacelinestext} space.  For instance, with a threshold of
\texttt{2}, every note below \texttt{c} will add {noteadditionalspacelinestext}
space for each pitch needed below \texttt{c}, accounting for the various signs.
\end{gcount}

\subsection{Distances}\label{distances}

Each of the following distances controls some aspect of the spacing of the Gregorio\TeX\ score.  They are changed using \verb=\grechangedim=, documented above.  If the distance permits a rubber value, then the default value will indicate the stretch and shrink (even if they are zero by default).  Distances whose default value does not include a stretch or shrink may not take a rubber value.

While it may seem strange that many of these distances are defined to 5 decimal places in centimeters (much smaller than most people can see) this is a legacy of how these distances were originally defined in small points.  Since most people don’t know what small points are, the distances were converted to a unit more familiar to most people, but no rounding was applied to the conversions so that scores wouldn’t change their appearance as a result of the conversion.  Users should feel under no obligation to maintain this level of precision when adjusting them to suit their own tastes.

\textbf{Nota Bene:} Because of the way Gregorio\TeX\ handles distances, these cannot be manipulated as if they were normal \TeX\ dimensions or skips.  As a result they should only be changed using the command defined by Gregorio\TeX\ for this purpose.

\begin{gdimension}{additionallineswidth}
The additional width of the additional lines (\ie, the value added to the width of the glyph with which they're associated to get the width of the line).
\end{gdimension}

\begin{gdimension}{alterationspace}
Space between an alteration (flat or natural) and the next glyph.
\end{gdimension}

\begin{gdimension}{beforealterationspace}
When beginning of line shifts (bolshifts) are enabled, minimum space between a clef at the beginning of the line and a leading alteration glyph.  This distance should be larger than \texttt{clefflatspace} so that a flatted clef can be distinguished from a flat which is part of the first glyph on a line, but also smaller than \texttt{spaceafterlineclef}, the distance from the clef to the first notes.
\end{gdimension}

\begin{gdimension}{beforelowchoralsignspace}
Space before a low choral sign.
\end{gdimension}

\begin{gdimension}{clefflatspace}
Space between a clef and a flat (for clefs with flat).
\end{gdimension}

\begin{gdimension}{interglyphspace}
Space between glyphs in the same element.
\end{gdimension}

\begin{gdimension}{zerowidthspace}
Null space.
\end{gdimension}

\begin{gdimension}{halfspace}
Half-space between elements.
\end{gdimension}

\begin{gdimension}{interelementspace}
Space between elements.
\end{gdimension}

\begin{gdimension}{largerspace}
Larger space between elements.
\end{gdimension}

\begin{gdimension}{glyphspace}
Space between elements which has the size of a note.
\end{gdimension}

\begin{gdimension}{spacebeforeeolcustos}
Space before custos at the end of a line.
\end{gdimension}

\begin{gdimension}{spacebeforeinlinecustos}
Space before custos within a line.
\end{gdimension}

\begin{gdimension}{spacebeforesigns}
Space before punctum mora and augmentum duplex.
\end{gdimension}

\begin{gdimension}{moraadjustment}
When a syllable (bar or not) is shifted left because of a preceding punctum
mora, this space is also added. Use it to make the syllable a bit further from
the punctum mora if you want.
\end{gdimension}

\begin{gdimension}{moraadjustmentbar}
Same as previous one but specific to cases where puntum mora precedes a bar.
\end{gdimension}

\begin{gdimension}{spaceaftersigns}
Space after punctum mora and augmentum duplex.
\end{gdimension}

\begin{gdimension}{spaceafterlineclef}
Space after a clef at the beginning of a line.
\end{gdimension}

\begin{gdimension}{intersyllablespacenotes}
Minimum space between notes of different syllables.
\end{gdimension}

\begin{gdimension}{intersyllablespacestretchhyphen}
Stretching added in the case where the text of two syllables of the same word are
separated with an automatic hyphen.
\end{gdimension}

\begin{gdimension}{interwordspacenotes}
Minimum space between notes of syllables from different words.
\end{gdimension}

\begin{gdimension}{interwordspacetext}
Minimum space between texts of different words. Please keep the same \texttt{plus} and \texttt{minus} as \texttt{interwordspacenotes}.
\end{gdimension}

\begin{gdimension}{interwordspacenotes@alteration}
Same as \texttt{interwordspacenotes} for the case where the second syllable starts with an alteration.
\end{gdimension}

\begin{gdimension}{intersyllablespacenotes@alteration}
Same as \texttt{intersyllablespacenotes} for the case where the second syllable starts with an alteration.
\end{gdimension}

\begin{gdimension}{interwordspacenotes@euouae}
Same as \texttt{interwordspacenotes} for \texttt{euouae} blocks.
\end{gdimension}

\begin{gdimension}{interwordspacetext@euouae}
Same as \texttt{interwordspacetext} for \texttt{euouae} blocks.
\end{gdimension}

\begin{gdimension}{bitrivirspace}
Space between notes of a bivirga or trivirga.
\end{gdimension}

\begin{gdimension}{bitristrospace}
Space between notes of a bistropha or tristrophae.
\end{gdimension}

\begin{gdimension}{punctuminclinatumshift}
Space between two descending puncta inclinata.
\end{gdimension}

\begin{gdimension}{punctuminclinatumunisonshift}
Space between two unison puncta inclinata.
\end{gdimension}

\begin{gdimension}{beforepunctainclinatashift}
Space before puncta inclinata.
\end{gdimension}

\begin{gdimension}{punctuminclinatumanddebilisshift}
Space between a punctum inclinatum and a punctum inclinatum deminutus,
descending.
\end{gdimension}

\begin{gdimension}{punctuminclinatumdebilisshift}
Space between two punctum inclinatum deminutus.
\end{gdimension}

\begin{gdimension}{punctuminclinatumbigshift}
Space between descending puncta inclinata, larger ambitus (range=3rd).
\end{gdimension}

\begin{gdimension}{punctuminclinatummaxshift}
Space between descending puncta inclinata, larger ambitus (range=4th or 5th).
\end{gdimension}

\begin{gdimension}{descendingpunctuminclinatumascendingshift}
Space between descending puncta inclinata shapes in an ascent of pitch.
\end{gdimension}

\begin{gdimension}{ascendingpunctuminclinatumshift}
Space between two ascending puncta inclinata.
\end{gdimension}

\begin{gdimension}{ascendingpunctuminclinatumanddebilisshift}
Space between a punctum inclinatum and a punctum inclinatum deminutus,
ascending.
\end{gdimension}

\begin{gdimension}{ascendingpunctuminclinatumbigshift}
Space between ascending puncta inclinata, larger ambitus (range=3rd).
\end{gdimension}

\begin{gdimension}{ascendingpunctuminclinatummaxshift}
Space between ascending puncta inclinata, larger ambitus (range=4th or 5th).
\end{gdimension}

\begin{gdimension}{ascendingpunctuminclinatumdescendingshift}
Space between ascending puncta inclinata shapes in a descent of pitch.
\end{gdimension}

\begin{gdimension}{descendinginclinatumtonobarshift}
Space between a punctum inclinatum and a no-bar (stemless) glyph one pitch
below.
\end{gdimension}

\begin{gdimension}{descendinginclinatumtonobarbigshift}
Space between a punctum inclinatum and a no-bar (stemless) glyph two pitches
below.
\end{gdimension}

\begin{gdimension}{descendinginclinatumtonobarmaxshift}
Space between a punctum inclinatum and a no-bar (stemless) glyph three or four
pitches below.
\end{gdimension}

\begin{gdimension}{ascendinginclinatumtonobarshift}
Space between a punctum inclinatum and a no-bar (stemless) glyph one pitch
above.
\end{gdimension}

\begin{gdimension}{ascendinginclinatumtonobarbigshift}
Space between a punctum inclinatum and a no-bar (stemless) glyph two pitches
above.
\end{gdimension}

\begin{gdimension}{ascendinginclinatumtonobarmaxshift}
Space between a punctum inclinatum and a no-bar (stemless) glyph three or four
pitches above.
\end{gdimension}

\begin{gdimension}{ascendinginclinatumtonobarmaxshift}
Space between a punctum inclinatum and a no-bar (stemless) glyph three or four
pitches above.
\end{gdimension}

\begin{gdimension}{maximumspacewithoutdash}
Maximal space between two syllables for which we consider a dash is not needed.
\end{gdimension}

\begin{gdimension}{afterclefnospace}
An extensible space for the beginning of lines.
\end{gdimension}

\begin{gdimension}{additionalcustoslineswidth}
Width of the additional lines, used only for the custos.  The width is the one for the custos at end of lines, the line for custos in the middle of a score is the same multiplied by 2.
\end{gdimension}

\begin{gdimension}{afterinitialshift}
Space between the initial and the beginning of the score.
\end{gdimension}

\begin{gdimension}{beforeinitialshift}
Space between the initial and the beginning of the score.
\end{gdimension}

\begin{gdimension}{minimalspaceatlinebeginning}
Minimal space in front of the lyrics at the beginning of a line when \texttt{bolshift}s are enabled.
\end{gdimension}

\begin{gdimension}{manualinitialwidth}
Space to force the initial width to.  Ignored when 0.
\end{gdimension}

\begin{gdimension}{minimalinitialwidth}
Minimum width of the initial.  Ignored when \texttt{manualinitialwidth} is non-zero.
\end{gdimension}

\begin{gdimension}{annotationseparation}
This space is the one between lines in the annotation (text above the initial).

\textbf{Nota Bene:} This is the absolute space.  If the lower line contains only short letters then it will get moved up so only this space shows (not the space above the letters on a normal line plus this space).  You should use struts to control the line height of the lower line if this is a problem.
\end{gdimension}

\begin{gdimension}{annotationraise}
Amount to raise (positive) or lower (negative) the annotation from its normal position (set with \verb=\gresetannotationby= and \verb=\gresetannotationvalign=).
\end{gdimension}

\begin{gdimension}{commentaryseparation}
This space is the one between lines in the commentary (text above the first staff line on the right).

\textbf{Nota Bene:} This is the absolute space.  If the lower line contains only short letters then it will get moved up so only this space shows (not the space above the letters on a normal line plus this space).  You should use struts to control the line height of the lower line if this is a problem.
\end{gdimension}

\begin{gdimension}{commentaryraise}
Distance from the commentary to the top line of the staff.
\end{gdimension}

\begin{gdimension}{noclefspace}
Space at the beginning of the lines if there is no clef.
\end{gdimension}

\begin{gdimension}{choralsigndownshift}
The distance to shift choral signs down.  The following choral signs are shifted down:

\begin{itemize}
	\item Low choral signs that are not lower than the note
	\item High choral signs which are in a space
	\item Low choral signs that are lower than the note which are in a space
\end{itemize}
\end{gdimension}

\begin{gdimension}{choralsignupshift}
The distance to shift choral signs up.  The following choral signs are shifted up:

\begin{itemize}
	\item High choral signs which are on a line
	\item Low choral signs that are lower than the note which are on a line
\end{itemize}
\end{gdimension}

\begin{gdimension}{translationheight}
The space for the translation.
\end{gdimension}

\begin{gdimension}{spaceabovelines}
The space above the lines.
\end{gdimension}

\begin{gdimension}{spacelinestext}
The space between the lines and the bottom of the text.
\end{gdimension}

\begin{gdimension}{noteadditionalspacelinestext}
The space added between the lines and the bottom of the text for every pitch
below the \texttt{noteadditionalspacelinestextthreshold}.
\end{gdimension}

\begin{gdimension}{spacebeneathtext}
The space beneath the text.
\end{gdimension}

\begin{gdimension}{abovelinestextraise}
Height of the text above the note line.
\end{gdimension}

\begin{gdimension}{abovelinestextheight}
Height that is added at the top of the lines if there is text above the lines (it must be bigger than the text for it to be taken into consideration).
\end{gdimension}

\begin{gdimension}{braceshift}
An additional shift you can give to the brace above the staff.
\end{gdimension}

\begin{gdimension}{curlybraceaccentusshift}
A shift you can give to the accentus above the curly brace.
\end{gdimension}

\begin{gdimension}{nabcinterelementspace}
Space between elements in ancient notation.
\end{gdimension}

\begin{gdimension}{nabclargerspace}
Larger space between elements in ancient notation.
\end{gdimension}

\begin{gdimension}{clivisalignmentmin}
When \verb=\gre@clivisalignment= is 2, this distance is the maximum length of the consonants after vowels for which the clivis will be aligned on its center.
\end{gdimension}

\begin{gdimension}{clefchangespace}
Space around a clef change.
\end{gdimension}

\begin{gdimension}{initialraise}
Distance the initial will be raised above its default baseline.  The default baseline for the initial coincides with the baseline for the text below the staff.
\end{gdimension}

\begin{gdimension}{overslurshift}
Distance an over-the-notes slur will be raised above the baseline of a note at the same height.
\end{gdimension}

\begin{gdimension}{underslurshift}
Distance an under-the-notes slur will be raised above the baseline of a note at the same height.
\end{gdimension}

\begin{gdimension}{divisiofinalissep}
Space separating the two bars of a divisio finalis.
\end{gdimension}

\begin{gdimension}{overhepisemalowshift}
Distance to place a a horizontal episema over a note in a low position in the space.
\end{gdimension}

\begin{gdimension}{overhepisemahighshift}
Distance to place a horizontal episema over a note in a high position in the space.
\end{gdimension}

\begin{gdimension}{underhepisemalowshift}
Distance to place a horizontal episema under a note in a low position in the space.
\end{gdimension}

\begin{gdimension}{underhepisemahighshift}
Distance to place a horizontal episema under a note in a high position in the space.
\end{gdimension}

\begin{gdimension}{hepisemamiddleshift}
Distance to place a horizontal episema in the middle of a space.
\end{gdimension}

\begin{gdimension}{vepisemalowshift}
Distance to place a vertical episema in a low position in the space.
\end{gdimension}

\begin{gdimension}{vepisemahighshift}
Distance to place a vertical episema in a high position in the space.
\end{gdimension}

\begin{gdimension}{linepunctummorashift}
Vertical distance to place a punctum mora for a note on a line.
\end{gdimension}

\begin{gdimension}{spacepunctummorashift}
Vertical distance to place a punctum mora for a note in a space.
\end{gdimension}

\begin{gdimension}{spaceamonepespunctummorashift}
Vertical distance to place a punctum mora for the second note (in a space) of a pes with ambitus one.
\end{gdimension}

\begin{gdimension}{lineporrectuspunctummorashift}
Vertical distance to place a punctum mora for the second note in a porrectus (or similar figure), on a line
\end{gdimension}

\begin{gdimension}{spaceporrectuspunctummorashift}
Vertical distance to place a punctum mora for the second note in a porrectus (or similar figure), in a space
\end{gdimension}

\begin{gdimension}{raresignshift}
Distance to place a ``rare'' sign above the top space in a score.
\end{gdimension}

\begin{gdimension}{bracketupshift}
Distance to shift a bracket up when the lowest note in the brackets is on a
line or below the staff.
\end{gdimension}

\begin{gdimension}{bracketdownshift}
Distance to shift a bracket down when the lowest note in the brackets is
neither on a line nor below the staff.
\end{gdimension}

\begin{gdimension}{parskip}
The effective \verb=\parskip= inside of a score.
\end{gdimension}

\begin{gdimension}{lineskip}
The effective \verb=\lineskip= inside of a score.
\end{gdimension}

\begin{gdimension}{baselineskip}
The effective \verb=\baselineskip= inside of a score.
\end{gdimension}

\begin{gdimension}{lineskiplimit}
The effective \verb=\lineskiplimit= inside of a score.
\end{gdimension}

\begin{gdimension}{shortspaceafterlineclef}
Space after a clef at the beginning of a line, when the clef and first note are vertically distant.
\end{gdimension}

\subsubsection{Bar distances}

\begin{gdimension}{bar@finalfinalis}
This space is added before the final divisio final of a score (old bar spacing algorithm only).
\end{gdimension}

Spaces around bars when they are typeset inside a syllable. The \verb=@short= suffix for virgula
and divisio minima indicates the space used when the notes surrounding the bar are strictly lower
than \texttt{g} (in a four-line score).

\begin{gdimension}{bar@virgula}
\end{gdimension}

\begin{gdimension}{bar@virgula@short}
\end{gdimension}

\begin{gdimension}{bar@minima}
\end{gdimension}

\begin{gdimension}{bar@minima@short}
\end{gdimension}

\begin{gdimension}{bar@minor}
\end{gdimension}

\begin{gdimension}{bar@dominican}
\end{gdimension}

\begin{gdimension}{bar@maior}
\end{gdimension}

\begin{gdimension}{bar@finalis}
\end{gdimension}

Spaces around bars in standalone syllables, when these have text (new bar spacing algorithm only):

\begin{gdimension}{bar@virgula@standalone@text}
\end{gdimension}

\begin{gdimension}{bar@virgula@standalone@text@short}
\end{gdimension}

\begin{gdimension}{bar@minima@standalone@text}
\end{gdimension}

\begin{gdimension}{bar@minima@standalone@text@short}
\end{gdimension}

\begin{gdimension}{bar@minor@standalone@text}
\end{gdimension}

\begin{gdimension}{bar@dominican@standalone@text}
\end{gdimension}

\begin{gdimension}{bar@maior@standalone@text}
\end{gdimension}

\begin{gdimension}{bar@finalis@standalone@text}
\end{gdimension}

\begin{gdimension}{bar@finalfinalis@standalone@text}
\end{gdimension}

Spaces around bars in standalone syllables, when these have no text (new bar spacing algorithm only):

\begin{gdimension}{bar@virgula@standalone@notext}
\end{gdimension}

\begin{gdimension}{bar@virgula@standalone@notext@short}
\end{gdimension}

\begin{gdimension}{bar@minima@standalone@notext}
\end{gdimension}

\begin{gdimension}{bar@minima@standalone@notext@short}
\end{gdimension}

\begin{gdimension}{bar@minor@standalone@notext}
\end{gdimension}

\begin{gdimension}{bar@dominican@standalone@notext}
\end{gdimension}

\begin{gdimension}{bar@maior@standalone@notext}
\end{gdimension}

\begin{gdimension}{bar@finalis@standalone@notext}
\end{gdimension}

\begin{gdimension}{bar@finalfinalis@standalone@notext}
\end{gdimension}

\begin{gdimension}{spacearoundclefbars}
Additional space that will appear around bars that are preceded by a custos and followed by a key.
\end{gdimension}

\begin{gdimension}{bar@rubber}
A rubber value applied on both sides of all bars in standalone syllables, in new bar spacing algorithm only.

\textbf{Nota Bene:} This distance should always have a base value of 0pt.
\end{gdimension}

\begin{gdimension}{interwordspacetext@bars}
Minimum space between texts of different words when one of the syllable contains only a bar (new bar spacing algorithm only).
\end{gdimension}

\begin{gdimension}{interwordspacetext@bars@euouae}
Same as \texttt{interwordspacetext@bars} for \texttt{euouae} blocks (so quite rare).
\end{gdimension}

\begin{gdimension}{interwordspacetext@bars@notext}
Minimum space between texts of adjacent words when they are separated by a bar syllable which has no text associated with it (new bar spacing algorithm only).
\end{gdimension}

\begin{gdimension}{interwordspacetext@bars@notext@euouae}
Same as \texttt{interwordspacetext@bars@notext} for \texttt{euouae} blocks (so quite rare).
\end{gdimension}

\begin{gdimension}{textbartextspace}
Space between the text of previous syllable and the text associated with the bar (old bar spacing algorithm only).
\end{gdimension}

\begin{gdimension}{notebarspace}
Minimal space between a note and a bar.
\end{gdimension}

\begin{gdimension}{maxbaroffsettextleft}
Maximum distance by which the center of a bar and the center of its associated text can be separated, when the center of the text goes left of the center of the bar (new bar spacing algorithm only).
\end{gdimension}

\begin{gdimension}{maxbaroffsettextright}
Same as \texttt{maxbaroffsettextleft} but when the center of the text goes right of the center of the bar.
\end{gdimension}

\begin{gdimension}{maxbaroffsettextleft@nobar}
Maximum distance by which the center of a “no-bar” (\ie something like \texttt{*()} in gabc) and the center of its associated text can be separated, when the center of the text goes left of the center of the no-bar (new bar spacing algorithm only).
\end{gdimension}

\begin{gdimension}{maxbaroffsettextright@nobar}
Same as \texttt{maxbaroffsettextleft@nobar} but when the center of the text goes right of the center of the no-bar.
\end{gdimension}

\begin{gdimension}{maxbaroffsettextleft@eol}
Maximum distance by which the center of a bar and the center of its associated text can be separated, when the center of the text goes left of the center of the bar and the bar syllable contains a manual line break (new bar spacing algorithm only).
\end{gdimension}

\begin{gdimension}{maxbaroffsettextright@eol}
Same as \texttt{maxbaroffsettextleft@eol} but when the center of the text goes right of the center of the bar.
\end{gdimension}

\begin{gdimension}{alterationadjustmentbar}
In the case of an alteration after a bar, the alteration will go a bit left of this value. This can be compared to \texttt{moraadjustmentbar}.
\end{gdimension}

\subsection{Penalties}\label{penalties}
Penalties are used by \TeX\ to determine where line and page breaks should occur.  Gregorio\TeX\ modifies or defines a few of its own to help with that process in scores.  With the exception of \texttt{emergencystretch} (which should be changed using \verb=\grechangedim=) these should be changed using \verb=\grechangecount=, described above.

\begin{gcount}{brokenpenalty}
The vertical penalty inserted after a break on a clef change.
\end{gcount}

\begin{gcount}{clubpenalty}
The club penalty (determines how important it is to prevent orphans from occurring).
\end{gcount}

\begin{gcount}{widowpenalty}
The widow penalty (determines how important it is to prevent widows from occurring).
\end{gcount}

\macroname{emergencystretch}{}{gsp-default.tex}
The value of the last ditch stretch for overfull boxes.  This should be set using \verb=\grechangedim=.

Default: \verb=\emergencystretch=

\begin{gcount}{endafterbarpenalty}
The end after bar penalty.
\end{gcount}

\begin{gcount}{endafterbaraltpenalty}{}{gsp-default.tex}
The alternate end after bar penalty (used when there is no text under the bar).
\end{gcount}

\begin{gcount}{endofelementpenalty}{}{gsp-default.tex}
The end of element penalty.
\end{gcount}

\begin{gcount}{endofsyllablepenalty}{}{gsp-default.tex}
The end of element penalty.
\end{gcount}

\begin{gcount}{endofwordpenalty}{}{gsp-default.tex}
The end of element penalty.
\end{gcount}

\begin{gcount}{hyphenpenalty}{}{gsp-default.tex}
The hyphen penalty.
\end{gcount}

\begin{gcount}{nobreakpenalty}{}{gsp-default.tex}
Penalty to prevent a line break.
\end{gcount}

\begin{gcount}{newlinepenalty}
Penalty to force a line break.
\end{gcount}

\begin{gcount}{finalpenalty}
The penalty applied after the final element of a score.
\end{gcount}

\macroname{looseness}{}{gsp-default.tex}
The \TeX\ looseness within a score.

Default: \verb=\looseness=

\begin{gcount}{tolerance}
The \TeX\ tolerance within a score.  See \url{https://en.wikibooks.org/wiki/TeX/tolerance} for an explanation of what tolerance is.
\end{gcount}

\macroname{pretolerance}{}{gsp-default.tex}
The \TeX\ pretolerance within a score.  See \url{https://en.wikibooks.org/wiki/TeX/pretolerance} for an explanation of what pretolerance is.

Default: $-1$ (Lua\TeX\ versions prior to 0.80) or \verb=\pretolerance= (versions after, and including, 0.80)]

\textit{Nota bene:} For more details on why this is necessary see the comments in gsp-default.tex.





\subsection{Colors}\label{colors}
All colors can be redefined using \verb=\definecolor=.  See the
\verb=xcolor= (\LaTeX) or \verb=color= (Plain\TeX) package for documentation.

Example:\par\medskip
\begin{latexcode}
	\definecolor{gregoriocolor}{RGB}{229,53,44}
\end{latexcode}

\macroname{grebackgroundcolor}{}{gregoriotex.sty}
The color Gregorio\TeX\ uses to block out elements which have been printed,
but shouldn't show (\eg, the staff line going through the interior of
a punctum cavum).  The default is white.

\macroname{gregoriocolor}{}{gregoriotex.sty}
A red similar to that found in liturgical documents.  This is the color that Gregorio\TeX\ uses for text formatted with \texttt{<c></c>} tags in gabc.

%%% Local Variables:
%%% mode: latex
%%% TeX-master: "GregorioRef"
%%% End:

% !TEX root = GregorioRef.tex
% !TEX program = LuaLaTeX+se
\section{Gregorio Controls}

These functions are the ones written by gregorio to the gtex file.
While one could, in theory, use/change them to alter the appearance of
elements of the score, it is far better to make your changes in the
gabc file and let gregorio make the changes to the gtex file.

\macroname{\textbackslash GreAnnotationLines}{\#1\#2}{gregoriotex-main.tex}
A wrapper macro for placing annotations above the initial. The
arguments are provided by the \texttt{gabc} file in the
\texttt{annotation} header field.  This macro tests for the presence
of the annotation box which means that the annotation is explicitly
defined in the \texttt{main.tex} file. If so, this macro does nothing,
respecting the annotation value in the \texttt{main.tex} file.

\begin{argtable}
	\#1 & string & First line text to place above the initial.\\
	\#2 & string & Second line text to place above the initial.\\
\end{argtable}

\macroname{\textbackslash GreBeginScore}{\#1\#2\#3\#4\#5\#6\#7}{gregoriotex-main.tex}
Macro to start a score.

\begin{argtable}
	\#1 & string  & a unique identifier for the score (currently an SHA-1-based digest of the gabc file)\\
	\#2 & integer & the height number of the top pitch of the entire score, including signs\\
	\#3 & integer & the height number of the bottom pitch of the entire score, including signs\\
	\#4 & 0 & there is no translation line in the score\\
			& 1 & there is a translation line somewhere in the score\\
	\#5 & 0 & there is no above lines text in the score\\
			& 1 & there is above lines text somewhere in the score\\
	\#6 & string  & the absolute filename of the gabc file if point-and-click is enabled\\
	\#7 & integer & the number of staff lines\\
\end{argtable}

\macroname{\textbackslash GreEndScore}{}{gregoriotex-main.tex}
Macro to end a score.

\macroname{\textbackslash GreBeginHeaders}{}{gregoriotex-main.tex}
Macro called at the beginning of a set of gabc headers.

\macroname{\textbackslash GreEndHeaders}{}{gregoriotex-main.tex}
Macro called at the end of a set of gabc headers.

\macroname{\textbackslash GreAccentus}{\#1\#2}{gregoriotex-signs.tex}
Macro for typesetting an accentus.

\begin{argtable}
	\#1 & integer & height number of episema\\
	\#2 & string  & Type of glyph the episema is attached to. See \nameref{EpisemaSpecial} argument for description of options.\\
\end{argtable}

\macroname{\textbackslash GreAdditionalLine}{\#1\#2\#3}{gregoriotex-signs.tex}
Macro to typeset the additional line above or below the staff.

\begin{argtable}
	\#1 & string  & See \nameref{EpisemaSpecial}.\\
	\#2 & integer & The ambitus of the porrectus or porrectus flexus if the first references these glyph types; ignored otherwise.\\
	\#3 & integer & Set horizontal episema (0), horizontal episema under a note (1), line at top of staff (2), line at bottom of staff (3), choral sign (4).\\
\end{argtable}

\macroname{\textbackslash GreAdHocSpaceEndOfElement}{\#1\#2}{gregoriotex-main.tex}
Macro to end an element with an ad-hoc space.

\begin{argtable}
	\#1 & float & The factor to scale the default space for use as an ad-hoc space.\\
	\#2 & \texttt{0} & Space is breakable.\\
	& \texttt{1} & Space is unbreakable.\\
\end{argtable}

\macroname{\textbackslash GreAugmentumDuplex}{\#1\#2\#3}{gregoriotex-signs.tex}
Macro for typesetting an augmentum duplex (a pair of punctum mora)

\begin{argtable}
	\#1 & integer & Height number for first punctum mora.\\
	\#2 & integer & Height number for second punctum mora.\\
	\#3 & integer & First punctum mora occurs before last note of a podatus, prorectus, or toculus resupinus (1), or not (0).\\
\end{argtable}

\macroname{\textbackslash GreBarBrace}{\#1}{gregoriotex-signs.tex}
Macro for typesetting a bar brace.

\begin{argtable}
	\#1 & string & Type of glyph the episema is attached to.  See \nameref{EpisemaSpecial} argument for description of options.\\
\end{argtable}

\macroname{\textbackslash GreBarSyllable}{\#1\#2\#3\#4\#5\#6\#7\#8\#9}{gregoriotex-syllable.tex}
Macro for typesetting a bar syllable.

\begin{argtable}
	\#1 & \TeX\ code & macro setting syllable letters for the current syllable\\
	\#2 & empty & reserved for future use\\
	\#3 & \TeX\ control sequence & the control sequence to use for styling the hyphen\\
	\#4 & \texttt{0} & this syllable is not the end of a word\\
	& \texttt{1} & this syllable is the end of a word\\
	\#5 & \TeX\ code & macros setting syllable letters for the next syllable\\
	\#6 & string & the line, byte offset, and column address for textedit links when point-and-click is enabled\\
	\#7 & & alignment type of the first next glyph\\
	\#8 &\TeX\ code & other macros (translation, double text, etc.) that don't fit in the limitation of the number of arguments\\
	\#9 & \TeX\ code & The bar line (usually a \textit{writebar} call).
\end{argtable}

\macroname{\textbackslash GreBarVEpisema}{\#1}{gregoriotex-signs.tex}
Macro to typeset a vertical episema on a bar.

\begin{argtable}
	\#1 & string & Type of glyph the episema is attached to.  See \nameref{EpisemaSpecial} argument for description of options.\\
\end{argtable}

\macroname{\textbackslash GreBeginEUOUAE}{\#1}{gregoriotex-main.tex}
Macro to mark the beginning of a EUOUAE block.  Alters spacings and prohibits a line break until the end of the block.

\begin{argtable}
	\#1 & integer & The identifier of the EUOUAE block.\\
\end{argtable}

\macroname{\textbackslash GreBeginNLBArea}{\#1\#2}{gregoriotex-main.tex}
Macro called at beginning of a no line break area.

\begin{argtable}
	\#1 & \texttt{0} & Not in the neumes.\\
	& \texttt{1} & In the neumes.\\
	\#2 & \texttt{0} & Call didn't come from translation centering.\\
	& \texttt{1} & Call came from translation centering.
\end{argtable}

\macroname{\textbackslash GreBold}{\#1}{gregoriotex.sty and gregoriotex.tex}
Makes argument bold.  Accesses \LaTeX\ \verb=\textbf= (\textit{gregoriotex.sty}) or Plain \TeX\ \verb=\bf= (\textit{gregoriotex.tex}) as appropriate.  Corresponds to ``<b></b>'' tags in gabc.

\begin{argtable}
	\#1 & string & Text to be typeset in bold.\\
\end{argtable}

\macroname{\textbackslash GreChangeClef}{\#1\#2\#3\#4\#5\#6\#7}{gregoriotex-signs.tex}
Macro called when key changes

\begin{argtable}
	\#1 & character & Type of new clef (c or f).\\
	\#2 & \texttt{1}--\texttt{5} & Line of new clef.\\
	\#3 & \texttt{0} & Print space before clef.\\
	& \texttt{1} & Do not print space before clef.\\
	\#4 & integer & Height number of flat in clef (\texttt{3} for no flat).\\
	\#5 & \texttt{c} or \texttt{f} & Type of secondary clef.\\
	\#6 & \texttt{0}--\texttt{5} & Line of secondary clef (\texttt{0} for no secondary clef).\\
	\#7 & integer & Height of flat in secondary clef (\texttt{3} for no flat).\\
\end{argtable}

\macroname{\textbackslash GreCirculus}{\#1\#2}{gregoriotex-signs.tex}
Macro for typesetting a circulus.

\begin{argtable}
	\#1 & integer & Height number of circulus.\\
	\#2 & string  & Type of glyph the circulus is attached to.  See \nameref{EpisemaSpecial} argument for description of options.\\
\end{argtable}

\macroname{\textbackslash GreColored}{\#1}{gregoriotex.sty \textup{and} gregoriotex.tex}
Colors argument (a string) in \verb=gregoriocolor.=  Corresponds to ``<c></c>'' tags in gabc.  Does nothing in Plain \TeX\ because color is not supported there.

\macroname{\textbackslash GreCPVirgaReversaAscendensOnDLine}{\#1}{gregoriotex-main.tex}
Allows the Dominican rule set to force long stems to be used for virga
reversa ascendens neumes on the ``d'' (lowest) line.  This macro is
defined and re-defined by the \verb=\gresetgregoriofont= macro.

\begin{argtable}
	\#1 & \TeX{} code & The \TeX{} code to use when long stems are not forced.\\
\end{argtable}

\macroname{\textbackslash GreCP...}{}{gregoriotex-main.tex}
A class of macros which point to the individual characters in a Gregoiro\TeX\ compatible font.  This class of macros is dynamically mapped from the glyph names embedded in the \texttt{ttf} file via a Lua script to ensure that the code points match up with the installed font.

\macroname{\textbackslash GreCustos}{\#1}{gregoriotex-signs.tex}
Typesets a custos.

\begin{argtable}
	\#1 & integer & Height number of custos.\\
\end{argtable}

\macroname{\textbackslash GreDagger}{}{gregoriotex-symbols.tex}
Macro to typeset a dagger (\GreDagger).

\macroname{\textbackslash GreDiscretionary}{\#1\#2\#3}{gregoriotex-signs.tex}
A Gregorio\TeX-specific discretionary. Currently only used to avoid clef change at beginning or end of line, or even with more complex data (z0::c3 for instance).  We require a special function because in the normal discretionary function you cannot use \verb=\hskip= (but you can use \verb=\kern=) and you cannot use \verb=\penalty= (which is useless indeed).  This macro corrects for these two limitations. The first argument allows to select the penalty assigned to the discretionary by recent version of Lua\TeX.

\begin{argtable}
	\#1 & integer & Type of discretionary (for penalty assignment). Currently possible value is 0 for clef change discretionaries.\\
	\#2 & \TeX\ code & First argument of resulting \verb=\discretionary=.\\
	\#3 & \TeX\ code & Third argument of resulting \verb=\discretionary=.\\
\end{argtable}

\macroname{\textbackslash GreDivisioFinalis}{\#1\#2}{gregoriotex-signs.tex}
Macro to typeset a divisio finalis.

\begin{argtable}
	\#1 & \texttt{0} & There is no text under the bar.\\
	& \texttt{1} & There is text under the bar.\\
	\#2 & \TeX\ code & Macros which may happen before the skip but after the divisio finalis (typically \verb=\grevepisema=).\\
\end{argtable}

\macroname{\textbackslash GreDivisioMaior}{\#1\#2}{gregoriotex-signs.tex}
Macro to typeset a divisio maior.

\begin{argtable}
	\#1 & \texttt{0} & There is no text under the bar.\\
	& \texttt{1} & There is text under the bar.\\
	\#2 & \TeX\ code & Macros which may happen before the skip but after the divisio maior (typically \verb=\grevepisema=).\\
\end{argtable}

\macroname{\textbackslash GreDivisioMinima}{\#1\#2}{gregoriotex-signs.tex}
Macro to typeset a divisio minima.

\begin{argtable}
	\#1 & \texttt{0} & There is no text under the bar.\\
	& \texttt{1} & There is text under the bar.\\
	\#2 & \TeX\ code & Macros which may happen before the skip but after the divisio minima (typically \verb=\grevepisema=).\\
\end{argtable}

\macroname{\textbackslash GreDivisioMinor}{\#1\#2}{gregoriotex-signs.tex}
Macro to typeset a divisio minor.

\begin{argtable}
	\#1 & \texttt{0} & There is no text under the bar.\\
	& \texttt{1} & There is text under the bar.\\
	\#2 & \TeX\ code & Macros which may happen before the skip but after the divisio minor (typically \verb=\grevepisema=).\\
\end{argtable}

\macroname{\textbackslash GreDominica}{\#1\#2\#3}{gregoriotex-signs.tex}
Macro to typeset a dominican bar.

\begin{argtable}
	\#1 & \texttt{1}--\texttt{6} & Type of dominican bar.  Corresponds to bar types 6--13 in \verb=\grewritebar=.\\
	\#2 & \texttt{0} & There is no text under the bar.\\
	& \texttt{1} & There is text under the bar.\\
	\#3 & \TeX\ code    & Macros which may happen before the skip but after the divisio dominica (typically \verb=\grevepisema=).\\
\end{argtable}

\macroname{\textbackslash GreDrawAdditionalLine}{\#1\#2\#3\#4\#5\#6}{gregoriotex-signs.tex}
Macro to draw ledger lines.

\begin{argtable}
	\#1 & \texttt{0} & Draw an over-the-staff ledger line. \\
			& \texttt{1} & Draw an under-the-staff ledger line. \\
	\#2 & distance   & The length of the line, with TeX units, excluding any left or right distances coming from the rest of the arguments. \\
	\#3 & \texttt{0} & Start the line at this point. \\
			& \texttt{1} & Start the line to the left of this point by \verb=gre@dimen@additionallineswidth=. \\
			& \texttt{2} & Start the line to the left of this point by \#4. \\
	\#4 & distance   & The distance to move left before starting the line if \#3 is \texttt{2}. \\
	\#5 & \texttt{0} & End the line exactly \#2 to the right of this point. \\
			& \texttt{1} & End the line \verb=gre@dimen@additionallineswidth= to the right of \#2 from this point. \\
			& \texttt{2} & End the line \#6 to the right of \#2 from this point. \\
	\#6 & distance   & The distance to end the line after \#2 from this point if \#3 is \texttt{2}. \\
\end{argtable}

\macroname{\textbackslash GreElision}{\#1}{gregoriotex-syllable.tex}
Typesets \#1 using the \texttt{elision} style.

\begin{argtable}
	\#1 & string & Text to be typeset in the \texttt{elision} style.\\
\end{argtable}

\macroname{\textbackslash GreEmptyFirstSyllableHyphen}{}{gregoriotex-syllable.tex}
Macro that indicates the position of an empty-first-syllable hyphen, should one be desired.

\macroname{\textbackslash GreEndEUOUAE}{\#1}{gregoriotex-main.tex}
Macro to mark the end of a EUOUAE block.

\begin{argtable}
	\#1 & \texttt{0} & ending element\\
	& \texttt{1} & ending syllable\\
	& \texttt{2} & ending score\\
	& \texttt{3} & before bar
\end{argtable}

\macroname{\textbackslash GreEndOfElement}{\#1\#2}{gregoriotex-main.tex}
Macro to end elements.

\begin{argtable}
	\#1 & \texttt{0} & Default space.\\
	& \texttt{1} & Larger space.\\
	& \texttt{2} & Glyph space.\\
	& \texttt{3} & Zero-width space.\\
	& \texttt{4} & Ad-hoc space.\\
	\#2 & \texttt{0} & Space is breakable.\\
	& \texttt{1} & Space is unbreakable.\\
\end{argtable}

\macroname{\textbackslash GreEndNLBArea}{\#1\#2}{gregoriotex-main.tex}
Macro to end a no line break area.

\begin{argtable}
	\#1 & 0 & ending element\\
	& \texttt{1} & ending syllable\\
	& \texttt{2} & ending score\\
	& \texttt{3} & before bar\\
	\#2 & \texttt{0} & ??\\ %I can’t tell what this flag is for
	& else & ??
\end{argtable}

\macroname{\textbackslash GreEndOfGlyph}{\#1}{gregoriotex-main.tex}
Macro to end a glyph without ending the element.

\begin{argtable}
	\#1 & \texttt{0} & Default space.\\
	& \texttt{1} & Zero-width space.\\
	& \texttt{2} & Space between flat or natural and a note.\\
	& \texttt{3} & Space between two puncta inclinata, descending.\\
	& \texttt{4} & Space between bivirga or trivirga.\\
	& \texttt{5} & space between bistropha or tristropha.\\
	& \texttt{6} & Space after a punctum mora XXX: not used yet, not so sure it is a good idea\ldots\\
	& \texttt{7} & Space between a punctum inclinatum and a punctum inclinatum debilis, descending.\\
	& \texttt{8} & Space between two puncta inclinata debilis.\\
	& \texttt{9} & Space before a punctum (or something else) and a punctum inclinatum.\\
	& \texttt{10} & Space between puncta inclinata (also debilis for now), larger ambitus (range=3rd), descending.\\
	& \texttt{11} & Space between puncta inclinata (also debilis for now), larger ambitus (range=4th or 5th), descending.\\
	& \texttt{12} & Space between two puncta inclinata, ascending. \\
	& \texttt{13} & Space between a punctum inclinatum and a punctum inclinatum debilis, ascending. \\
	& \texttt{14} & Space between puncta inclinata (also debilis for now), larger ambitus (range=3rd), ascending. \\
	& \texttt{15} & Space between puncta inclinata (also debilis for now), larger ambitus (range=4th or 5th), ascending. \\
	& \texttt{16} & Space between a punctum inclinatum and a ``no-bar'' glyph one pitch below. \\
	& \texttt{17} & Space between a punctum inclinatum and a ``no-bar'' glyph two pitches below. \\
	& \texttt{18} & Space between a punctum inclinatum and a ``no-bar'' glyph three or four pitches below \\
	& \texttt{19} & Space between a punctum inclinatum and a ``no-bar'' glyph one pitch above. \\
	& \texttt{20} & Space between a punctum inclinatum and a ``no-bar'' glyph two pitches above. \\
	& \texttt{21} & Space between a punctum inclinatum and a ``no-bar'' glyph three or four pitches above \\
	& \texttt{22} & Half-space. \\
\end{argtable}

\macroname{\textbackslash GreFinalCustos}{\#1}{gregoriotex-signs.tex}
Typesets a custos after the final bar in a score.

\begin{argtable}
	\#1 & integer & Height number of custos.\\
\end{argtable}

\macroname{\textbackslash GreFinalDivisioFinalis}{\#1}{gregoriotex-signs.tex}
Macro to end a score with a divisio finalis.

\begin{argtable}
	\#1 & \texttt{0} & Something does not need to be placed after the divisio finalis.\\
	& \texttt{1} & Something needs to be placed after the divisio finalis.\\
\end{argtable}

\macroname{\textbackslash GreFinalDivisioMaior}{\#1}{gregoriotex-signs.tex}
Macro to end a score with a divisio maior.

\begin{argtable}
	\#1 & \texttt{0} & Something does not need to be placed after the divisio maior.\\
	& \texttt{1} & Something needs to be placed after the divisio maior.\\
\end{argtable}

\macroname{\textbackslash GreFirstSyllable}{\#1}{gregoriotex-syllable.tex}
A macro which is called with the text of the first syllable, excluding the
initial of the score.  This macro may be redefined to style the first syllable
appropriately.  This macro may be called up to three times: for the letters
before the centered letters, for the centered letters, and for the letters
after the centered letters.

\begin{argtable}
	\#1 & string & Text from the first syllable.
\end{argtable}

\macroname{\textbackslash GreFirstSyllableInitial}{\#1}{gregoriotex-syllable.tex}
A macro which is called with the first letter of the first syllable which is
not the initial of the score.  If the \texttt{initial-style} is \texttt{0}, the
first letter of the syllable will be passed.  If the \texttt{initial-style} is
\texttt{1} or \texttt{2}, the \emph{second} letter will be passed.  This macro
may be redefined to style the first letter appropriately.

\begin{argtable}
	\#1 & string & The first letter of the first syllable which is not the
								 initial of the score.
\end{argtable}

\macroname{\textbackslash GreFirstWord}{\#1}{gregoriotex-syllable.tex}
A macro which is called with the text of the first word, excluding the
initial of the score.  This macro may be redefined to style the first word
appropriately.  This macro may be called multiple times, depending on how
many syllables are in the word.

\begin{argtable}
	\#1 & string & Text from the first word.
\end{argtable}

\macroname{\textbackslash GreFlat}{\#1\#2\#3\#4\#5}{gregoriotex-signs.tex}
Macro to typeset a flat.

\begin{argtable}
	\#1 & integer & Height number of the flat.\\
	\#2 & \texttt{0} & The flat is not part of the clef.\\
	& \texttt{1} & The flat is part of the clef.\\
	\#3 & \TeX\ code & signs to typeset before the glyph (typically additional bars, as they must be "behind" the glyph)\\
	\#4 & \TeX\ code & signs to typeset after the glyph (almost all signs)\\
	\#5 & string & the line, byte offset, and column address for textedit links when point-and-click is enabled\\
\end{argtable}

\macroname{\textbackslash GreForceHyphen}{}{gregoriotex-syllable.tex}
Macro that indicates that a hyphen should be forced (if enabled) after the given syllable.

\macroname{\textbackslash GreFuse}{}{gregoriotex-main.tex}
Macro used between two fused glyphs.

\macroname{\textbackslash GreFuseTwo}{\#1\#2}{gregoriotex-main.tex}
Macro for fusing two glyphs to create a larger neume.

\begin{argtable}
	\#1 & Gregorio\TeX\ glyph & The first glyph in the sequence.\\
	\#2 & Gregorio\TeX\ glyph & The second.
\end{argtable}

\macroname{\textbackslash GreGlyph}{\#1\#2\#3\#4\#5\#6\#7}{gregoriotex-syllable.tex}
Macro to typeset a glyph.

\begin{argtable}
	\#1 & character & the character that it must call\\
	\#2 & integer & The number for where the glyph is located.  \texttt{a} in gabc is 1, \texttt{b} is 2, \etc\\
	\#3 & integer & height number of the next note\\
	\#4 & \texttt{0} & One-note glyph or more than two notes glyph except porrectus: \ie,  we must put the aligncenter in the middle of the first note\\
	& \texttt{1} & Two notes glyph (podatus is considered as a one-note glyph): \ie, we put the aligncenter in the middle of the glyph\\
	& \texttt{2} & Porrectus: has a special align center.\\
	& \texttt{3} & initio-debilis : same as 1 but the first note is much smaller\\
	& \texttt{4} & case of a glyph starting with a quilisma\\
	& \texttt{5} & case of a glyph starting with an oriscus\\
	& \texttt{6} & case of a punctum inclinatum\\
	& \texttt{7} & case of a stropha\\
	& \texttt{8} & flexus with an ambitus of one\\
	& \texttt{9} & flexus deminutus\\
	\#5 & \TeX\ code & signs to typeset before the glyph (typically additional bars, as they must be "behind" the glyph)\\
	\#6 & \TeX\ code & signs to typeset after the glyph (almost all signs)\\
	\#7 & string & the line, byte offset, and column address for textedit links when point-and-click is enabled
\end{argtable}

\macroname{\textbackslash GreGlyphHeights}{\#1\#2}{gregoriotex-syllable.tex}
Passes the glyph height limits.

\begin{argtable}
	\#1 & integer & the high height\\
	\#2 & integer & the low height
\end{argtable}

\macroname{\textbackslash GregorioTeXAPIVersion}{\#1}{gregoriotex-main.tex}
Checks to see if Gregorio\TeX\ API is version specified by argument (and
therefore compatible with the score.

\begin{argtable}
	\#1 & string & Version number for Gregorio\TeX.\\
\end{argtable}

\macroname{\textbackslash GreHeader}{\#1\#2}{gregoriotex-main.tex}
Macro used to pass headers to TeX.

\begin{argtable}
	\#1 & string & The header name.\\
	\#2 & string & The header value.\\
\end{argtable}

\macroname{\textbackslash GreHEpisema}{\#1\#2\#3\#4\#5\#6\#7\#8\#9}{gregoriotex-signs.tex}
Macro to typeset an horizontal episema.

\begin{argtable}
	\#1 & integer & Height number of the episema.\\
	\#2 & string  & See \nameref{EpisemaSpecial}.\\
	\#3 & integer & The ambitus for a two note episema at the diagonal stroke of a
		porrectus, porrectus flexus, orculus resupinus, or torculus resupinus
		flexus.\\
	\#4 & \texttt{0} & an horizontal episema\\
	& \texttt{1} & an horizontal episema under a note\\
	& \texttt{2} & a line at the top\\ 
	& \texttt{3} & a line at the bottom\\
	\#5 & \texttt{f} & a normal episema\\
	& \texttt{l} & a small episema aligned left\\
	& \texttt{c} & a small episema aligned center\\
	& \texttt{r} & a small episema aligned right\\
	\#6 & integer & Replacement for \#1 if a bridge causes a height substitution.\\
	\#7 & \TeX\ code & code that sets heuristics\\
	\#8 & string & a positive or negative "nudge" (dimension) for the vertical position of the horizontal episema\\
	\#9 & \texttt{0} & for horizontal episema cases, use automatic positioning within the space\\
	& \texttt{1} & for horizontal episema cases, position in the middle of the space\\
	& \texttt{2} & for horizontal episema cases, position low within the space as if the episema is over the note\\
	& \texttt{3} & for horizontal episema cases, position high within the space as if the episema is under the note\\
	& \texttt{4} & for horizontal episema cases, position low within the space as if the episema is over the note\\
	& \texttt{5} & for horizontal episema cases, position high within the space as if the episema is under the note\\
\end{argtable}

\macroname{\textbackslash GreHEpisemaBridge}{\#1\#2\#3\#4\#5\#6}{gregoriotex-signs.tex}
Macro to typeset a bridge episema for the last note of a glyph
(element, syllable) if the next episema is at the same height.

\begin{argtable}
	\#1 & integer & Height number of the episema.\\
	\#2 & \texttt{0} & Episema above the note.\\
	& \texttt{1} & Episema below the note.\\
	\#3 & \texttt{0} & Default space.\\
	& \texttt{1} & Zero-width space.\\
	& \texttt{2} & Space between flat or natural and a note.\\
	& \texttt{3} & Space between two puncta inclinata.\\
	& \texttt{4} & Space between bivirga or trivirga.\\
	& \texttt{5} & space between bistropha or tristropha.\\
	& \texttt{6} & Space after a punctum mora XXX: not used yet, not so sure it is a good idea\ldots\\
	& \texttt{7} & Space between a punctum inclinatum and a punctum inclinatum debilis.\\
	& \texttt{8} & Space between two puncta inclinata debilis.\\
	& \texttt{9} & Space before a punctum (or something else) and a punctum inclinatum.\\
	& \texttt{10} & Space between puncta inclinata (also debilis for now), larger ambitus (range=3rd).\\
	& \texttt{11} & Space between puncta inclinata (also debilis for now), larger ambitus (range=4th or more).\\
	\#4 & \TeX\ code & code that sets heuristics\\
	\#5 & string & a positive or negative "nudge" (dimension) for the vertical position of the horizontal episema\\
	\#6 & \texttt{0} & for horizontal episema cases, use automatic positioning within the space\\
	& \texttt{1} & for horizontal episema cases, position in the middle of the space\\
	& \texttt{2} & for horizontal episema cases, position low within the space as if the episema is over the note\\
	& \texttt{3} & for horizontal episema cases, position high within the space as if the episema is under the note\\
	& \texttt{4} & for horizontal episema cases, position low within the space as if the episema is over the note\\
	& \texttt{5} & for horizontal episema cases, position high within the space as if the episema is under the note\\
\end{argtable}

\macroname{\textbackslash GreHighChoralSign}{\#1\#2\#3}{gregoriotex-signs.tex}
Macro for typesetting high choral signs.

\begin{argtable}
	\#1 & integer & Height number of the sign.\\
	\#2 & string  & The choral sign.\\
	\#3 & \texttt{0} & Choral sign does not occur before last note of podatus, porrectus, or torculus resupinus.\\
	& \texttt{1} & Choral sign occurs before last note of podatus, porrectus, or torculus resupinus.\\
\end{argtable}

\macroname{\textbackslash GreHyph}{}{gregoriotex-main.tex}
Macro used for end of line hyphens.  Defaults to \verb=\gre@char@normalhyph=.

\macroname{\textbackslash GreInDivisioFinalis}{\#1\#2}{gregoriotex-signs.tex}
Same as \verb=\GreDivisioFinalis= except inside a syllable.

\macroname{\textbackslash GreInDivisioMaior}{\#1\#2}{gregoriotex-signs.tex}
Same as \verb=\GreDivisioMaior= except inside a syllable.

\macroname{\textbackslash GreInDivisioMinima}{\#1\#2}{gregoriotex-signs.tex}
Same as \verb=\GreDivisioMinima= except inside a syllable.

\macroname{\textbackslash GreInDivisioMinor}{\#1\#2}{gregoriotex-signs.tex}
Same as \verb=\GreDivisioMinor= except inside a syllable.

\macroname{\textbackslash GreInDominica}{\#1\#2\#3}{gregoriotex-signs.tex}
Same as \verb=\GreDominica= except inside a syllable.

\macroname{\textbackslash GreInVirgula}{\#1\#2}{gregoriotex-signs.tex}
Same as \verb=\GreVirgula= except inside a syllable.

\macroname{\textbackslash GreItalic}{\#1}{gregoriotex.sty or gregoriotex.tex}
Makes argument (a string) italic.  Accesses \LaTeX\ \verb=\textit= or
Plain \TeX\ \verb=\it= as appropriate.  Corresponds to ``<i></i>'' tags
in gabc.

\begin{argtable}
	\#1 & string & Text to be typeset in italic font.\\
\end{argtable}

\macroname{\textbackslash GreLastOfLine}{}{gregoriotex-main.tex}
Macro to set \verb=\gre@lastoflinecount= to 1 (\ie, mark that this syllable is the last of the line).

\macroname{\textbackslash GreLastOfScore}{}{gregoriotex-main.tex}
Macro to mark the syllable as the last of the score.

\macroname{\textbackslash GreLinea}{\#1\#2\#3}{gregoriotex-signs.tex}
Macro for typesetting a linea.

\begin{argtable}
	\#1 & length  & Argument \#2 from \verb=\GreGlyph=. Height to raise the glyph.\\
	\#2 & length  & Argument \#3 from \verb=\GreGlyph=. Height of the next note.\\
	\#3 & integer & Argument \#4 from \verb=\GreGlyph=. The type of glyph.\\
\end{argtable}

\macroname{\textbackslash GreLineaPunctumCavum}{\#1\#2\#3\#4\#5\#6}{gregoriotex-signs.tex}
Macro to typeset a linea punctum cavum.

\begin{argtable}
	\#1 & length  & Argument \#2 from \verb=\GreGlyph=. Height to raise the glyph.\\
	\#2 & length  & Argument \#3 from \verb=\GreGlyph=. Height of the next note.\\
	\#3 & integer & Argument \#4 from \verb=\GreGlyph=. The type of glyph.\\
	\#4 & \TeX\ code    & Macros executed before the punctum cavum is written.\\
	\#5 & character & Argument \#5 from \verb=\GreGlyph=. The signs to typeset before the glyph.\\
	\#6 & string & the line, byte offset, and column address for textedit links when point-and-click is enabled.
\end{argtable}

\macroname{\textbackslash GreLowChoralSign}{\#1\#2\#3}{gregoriotex-signs.tex}
Macro for typesetting low choral signs.

\begin{argtable}
	\#1 & integer & Height number of the sign.\\
	\#2 & string  & The choral sign.\\
	\#3 & \texttt{0} & Choral sign does not occur before last note of podatus, porrectus, or torculus resupinus.\\
	& \texttt{1} & Choral sign occurs before last note of podatus, porrectus, or torculus resupinus.\\
\end{argtable}

\macroname{\textbackslash GreMode}{\#1\#2\#3}{gregoriotex-main.tex}
If the gabc file contains a mode in the header, then this function
places said mode as the first (top) annotation.  If the user has
manually added a first annotation in the \TeX\ file, then this
function does nothing. Also, if the \texttt{annotation} header field
is used, then this function does nothing.

\begin{argtable}
	\#1 & \TeX\ code & Mode text to place above the initial of a score in the \texttt{modeline} style.\\
	\#2 & \TeX\ code & Arbitrary code to typeset, in the \texttt{modemodifier} style, after the mode text.\\
	\#3 & \TeX\ code & Arbitrary code to typeset, in the \texttt{modedifferentia} style, after \#2.\\
\end{argtable}

\macroname{\textbackslash GreNatural}{\#1\#2\#3\#4\#5}{gregoriotex-signs.tex}
Macro to typeset a natural.

\begin{argtable}
	\#1 & integer & Height number of the natural.\\
	\#2 & \texttt{0} & The natural is not part of the clef.\\
	& \texttt{1} & The natural is part of the clef (doesn't happen).\\
	\#3 & \TeX\ code & signs to typeset before the glyph (typically additional bars, as they must be "behind" the glyph)\\
	\#4 & \TeX\ code & signs to typeset after the glyph (almost all signs)\\
	\#5 & string & the line, byte offset, and column address for textedit links when point-and-click is enabled\\
\end{argtable}

\macroname{\textbackslash GreNewLine}{}{gregoriotex-main.tex}
Macro to call if you want to go to the next line.

\macroname{\textbackslash GreNewParLine}{}{gregoriotex-main.tex}
Same as \verb=\GreNewLine= except line is not justified.

\macroname{\textbackslash GreNextCustos}{\#1}{gregoriotex-signs.tex}
Sets the pitch to use for the next custos if it were to happen at the point
where this macro is called.

\begin{argtable}
	\#1 & integer & Height number of the custos.\\
\end{argtable}

\macroname{\textbackslash GreNextSyllableBeginsEUOUAE}{\#1}{gregoriotex-syllable.tex}
Indicates that the syllable which follows begins a EUOUAE block.

\begin{argtable}
	\#1 & integer & The identifier of the EUOUAE block.\\
\end{argtable}

\macroname{\textbackslash GreOriscusCavum}{\#1\#2\#3\#4\#5\#6}{gregoriotex-signs.tex}
Macro to typeset an oriscus cavum (the oriscus points at a higher note).

\begin{argtable}
	\#1 & length  & Argument \#2 from \verb=\GreGlyph=. Height to raise the glyph.\\
	\#2 & length  & Argument \#3 from \verb=\GreGlyph=. Height of the next note.\\
	\#3 & integer & Argument \#4 from \verb=\GreGlyph=. The type of glyph.\\
	\#4 & \TeX\ code & Macros executed before the oriscus cavum is written.\\
	\#5 & character & Argument \#5 from \verb=\GreGlyph=. The signs to typeset before the glyph.\\
	\#6 & string & the line, byte offset, and column address for textedit links when point-and-click is enabled.
\end{argtable}

\macroname{\textbackslash GreOriscusCavumAuctus}{\#1\#2\#3\#4\#5\#6}{gregoriotex-signs.tex}
Macro to typeset a reverse oriscus cavum (the oriscus points at a lower note).

\begin{argtable}
	\#1 & length  & Argument \#2 from \verb=\GreGlyph=. Height to raise the glyph.\\
	\#2 & length  & Argument \#3 from \verb=\GreGlyph=. Height of the next note.\\
	\#3 & integer & Argument \#4 from \verb=\GreGlyph=. The type of glyph.\\
	\#4 & \TeX\ code & Macros executed before the oriscus cavum is written.\\
	\#5 & character & Argument \#5 from \verb=\GreGlyph=. The signs to typeset before the glyph.\\
	\#6 & string & the line, byte offset, and column address for textedit links when point-and-click is enabled.
\end{argtable}

\macroname{\textbackslash GreOriscusCavumDeminutus}{\#1\#2\#3\#4\#5\#6}{gregoriotex-signs.tex}
Macro to typeset a reverse oriscus cavum with a deminutus tail.

\begin{argtable}
	\#1 & length  & Argument \#2 from \verb=\GreGlyph=. Height to raise the glyph.\\
	\#2 & length  & Argument \#3 from \verb=\GreGlyph=. Height of the next note.\\
	\#3 & integer & Argument \#4 from \verb=\GreGlyph=. The type of glyph.\\
	\#4 & \TeX\ code & Macros executed before the oriscus cavum is written.\\
	\#5 & character & Argument \#5 from \verb=\GreGlyph=. The signs to typeset before the glyph.\\
	\#6 & string & the line, byte offset, and column address for textedit links when point-and-click is enabled.
\end{argtable}

\macroname{\textbackslash GreOverBrace}{\#1\#2\#3\#4}{gregoriotex-signs.tex}
Macro to typeset a round brace above the lines.

\begin{argtable}
	\#1 & length & The width of the brace.\\
	\#2 & length & A vertical shift.\\
	\#3 & length & A horizontal shift.\\
	\#4 & \texttt{0} & Don't shift before starting the brace.\\
	& \texttt{1} & Shift back a punctum's width before starting the brace.
\end{argtable}

\macroname{\textbackslash GreOverCurlyBrace}{\#1\#2\#3\#4\#5}{gregoriotex-signs.tex}
Macro to typeset a curly brace above the lines.

\begin{argtable}
	\#1 & length & The width of the brace.\\
	\#2 & length & A vertical shift.\\
	\#3 & length & A horizontal shift.\\
	\#4 & \texttt{0} & Don't shift before starting the brace.\\
	& \texttt{1} & Shift back a punctum's width before starting the brace.\\
	\#5 & \texttt{0} & No accentus above the brace.\\
	& \texttt{1} & Typeset an accentus above the brace.
\end{argtable}

\macroname{\textbackslash GrePunctumCavum}{\#1\#2\#3\#4\#5\#6}{gregoriotex-signs.tex}
Macro to typeset a punctum cavum.

\begin{argtable}
	\#1 & length  & Argument \#2 from \verb=\GreGlyph=. Height to raise the glyph.\\
	\#2 & length  & Argument \#3 from \verb=\GreGlyph=. Height of the next note.\\
	\#3 & integer & Argument \#4 from \verb=\GreGlyph=. The type of glyph.\\
	\#4 & \TeX\ code & Macros executed before the punctum cavum is written.\\
	\#5 & character & Argument \#5 from \verb=\GreGlyph=. The signs to typeset before the glyph.\\
	\#6 & string & the line, byte offset, and column address for textedit links when point-and-click is enabled.
\end{argtable}

\macroname{\textbackslash GrePunctumCavumInclinatum}{\#1\#2\#3\#4\#5\#6}{gregoriotex-signs.tex}
Macro to typeset a punctum cavum inclinatus.

\begin{argtable}
	\#1 & length  & Argument \#2 from \verb=\GreGlyph=. Height to raise the glyph.\\
	\#2 & length  & Argument \#3 from \verb=\GreGlyph=. Height of the next note.\\
	\#3 & integer & Argument \#4 from \verb=\GreGlyph=. The type of glyph.\\
	\#4 & \TeX\ code & Macros executed before the punctum cavum is written.\\
	\#5 & character & Argument \#5 from \verb=\GreGlyph=. The signs to typeset before the glyph.\\
	\#6 & string & the line, byte offset, and column address for textedit links when point-and-click is enabled.
\end{argtable}

\macroname{\textbackslash GrePunctumCavumInclinatumAuctus}{\#1\#2\#3\#4\#5\#6}{gregoriotex-signs.tex}
Macro to typeset a punctum cavum inclinatus auctus.

\begin{argtable}
	\#1 & length  & Argument \#2 from \verb=\GreGlyph=. Height to raise the glyph.\\
	\#2 & length  & Argument \#3 from \verb=\GreGlyph=. Height of the next note.\\
	\#3 & integer & Argument \#4 from \verb=\GreGlyph=. The type of glyph.\\
	\#4 & \TeX\ code & Macros executed before the punctum cavum is written.\\
	\#5 & character & Argument \#5 from \verb=\GreGlyph=. The signs to typeset before the glyph.\\
	\#6 & string & the line, byte offset, and column address for textedit links when point-and-click is enabled.
\end{argtable}

\macroname{\textbackslash GrePunctumMora}{\#1\#2\#3\#4}{gregoriotex-signs.tex}
Macro for typesetting punctum mora.

\begin{argtable}
	\#1 & integer & Height number of punctum mora.\\
	\#2 & \texttt{0} & General case.\\
	& \texttt{1} & Make the punctum mora zero width.\\
	& \texttt{2} & Shift left width of 1 punctum.\\
	& \texttt{3} & Shift left width of 1 punctum if last ambitus is 1.\\
	\#3 & \texttt{0} & Punctum mora does not occur before last note of podatus, porrectus, or torculus resupinus.\\
	& \texttt{1} & Punctum mora occurs before last note of podatus, porrectus, or torculus resupinus.\\
	\#4 & \texttt{0} & No punctum inclinatum.\\
	& \texttt{1} & Punctum inclinatum.\\
\end{argtable}

\macroname{\textbackslash GreReversedAccentus}{\#1\#2}{gregoriotex-signs.tex}
Macro for typesetting a reversed accentus.

\begin{argtable}
	\#1 & integer & Height number of accentus.\\
	\#2 & string  & Type of glyph the accentus is attached to. See \nameref{EpisemaSpecial} argument for description of options.\\
\end{argtable}

\macroname{\textbackslash GreReversedSemicirculus}{\#1\#2}{gregoriotex-signs.tex}
Macro for typesetting a reversed semicirculus.

\begin{argtable}
	\#1 & integer & Height number of semicirculus.\\
	\#2 & string  & Type of glyph the semicirculus is attached to. See \nameref{EpisemaSpecial} argument for description of options.\\
\end{argtable}

\macroname{\textbackslash GreScoreOpening}{\#1\#2\#3\#4\#5}{gregoriotex-syllable.tex}
Opens the score.

\begin{argtable}
	\#1 & \TeX\ code & Macros rendering the things after the initial but before the notes.\\
	\#2 & \TeX\ code & Macros rendering the things after starting notes but before the syllable.\\
	\#3 & \TeX\ code & Macros rendering the things before the initial.\\
	\#4 & \TeX\ control sequence & Control sequence for the syllable.\\
	\#5 & \TeX\ code & Macros rendering the first syllable; should emit the initial and populate \verb=\gre@opening@syllabletext=.\\
\end{argtable}

\macroname{\textbackslash GreSemicirculus}{\#1\#2}{gregoriotex-signs.tex}
Macro for typesetting a semicirculus.

\begin{argtable}
	\#1 & integer & Height number of semicirculus.\\
	\#2 & string  & Type of glyph the semicirculus is attached to. See \nameref{EpisemaSpecial} argument for description of options.\\
\end{argtable}

\macroname{\textbackslash GreSetFirstSyllableText}{\#1\#2\#3\#4\#5\#6}{gregoriotex-syllable.tex}
Sets the first syllable text.

\begin{argtable}
	\#1 & \TeX\ code & Initial.\\
	\#2 & \TeX\ code & First letter after the initial.\\
	\#3 & \TeX\ code & Everything else in the syllable.\\
	\#4 & \TeX\ code & Three syllable parts when there is a separated initial.\\
	\#5 & \TeX\ code & Three syllable parts where there is no separated initial.\\
	\#6 & \TeX\ code & Extra macros to run if there is an initial.\\
\end{argtable}

\macroname{\textbackslash GreSetFixedNextTextFormat}{\#1}{gregoriotex-syllable.tex}
Same as \verb=\GreSetFixedTextFormat= except for next syllable.

\macroname{\textbackslash GreSetFixedTextFormat}{\#1}{gregoriotex-syllable.tex}
Macro to specify a text which is different from \verb=#1#2#3= (of \verb=\GreSyllable=). It is useful for styles, for instance with:
\par\medskip
\begin{gabccode}
	<i>ffj</i>(gh)
\end{gabccode}

we will have

\begin{latexcode}
	#1 = \textit{f}
	#2 = \textit{f}
	#3 = \textit{j}
\end{latexcode}

and thus \verb=#1#2#3= will be \verb=\textit{f}\textit{f}\textit{j}=, which won't typeset
ligatures. In this example we should call \verb=\grefixedtext{\textit{ffj}}=.

\begin{argtable}
	\#1 & \texttt{0} & nothing (normal text)\\
	& \texttt{1}& italic\\
	& \texttt{2} & bold\\
	& \texttt{3} & small caps\\
	& \texttt{4} & typewriter\\
	& \texttt{5} & underline
\end{argtable}

\begin{argtable}
	\#1 & character & The initial letter of the score.\\
\end{argtable}

\macroname{\textbackslash GreSetInitialClef}{\#1\#2\#3\#4\#5\#6}{gregoriotex-signs.tex}
Macro for writing initial clef.

\begin{argtable}
	\#1 & \texttt{c} or \texttt{f} & Type of clef.\\
	\#2 & \texttt{1}--\texttt{5} & Line of clef.\\
	\#3 & integer & Height number of flat in clef (\texttt{3} for no flat).\\
	\#4 & \texttt{c} or \texttt{f} & Type of secondary clef.\\
	\#5 & \texttt{0}--\texttt{5} & Line of secondary clef (\texttt{0} for no secondary clef).\\
	\#6 & integer & Height of flat in secondary clef (\texttt{3} for no flat).\\
\end{argtable}

\macroname{\textbackslash GreSetLinesClef}{\#1\#2\#3\#4\#5\#6\#7}{gregoriotex-main.tex}
Macro to define the clef that will appear at the beginning of the lines.

\begin{argtable}
	\#1 & \texttt{c} or \texttt{f} & Type of clef.\\
	\#2 & \texttt{1}--\texttt{5} & Line of clef.\\
	\#3 & \texttt{0} & No space after clef.\\
	& \texttt{1} & Space after clef.\\
	\#4 & integer & Height of flat in clef (\texttt{3} for no flat).\\
	\#5 & \texttt{c} or \texttt{f} & Type of secondary clef.\\
	\#6 & \texttt{0}--\texttt{5} & Line of secondary clef (\texttt{0} for no secondary clef).\\
	\#7 & integer & Height of flat in secondary clef (\texttt{3} for no flat).\\
\end{argtable}

\macroname{\textbackslash GreSetNextSyllable}{\#1\#2\#3}{gregoriotex-syllable.tex}
Macro to set the text of the next syllable for spacing purposes.

\begin{argtable}
	\#1 & string & the first letters of the syllable, that don't count for the alignment\\
	\#2 & string & the middle letters of the syllable, we must align in the middle of them\\
	\#3 & string & the end letters, they don't count for alignment\\
\end{argtable}

\macroname{\textbackslash GreSetNoFirstSyllableText}{}{gregoriotex-syllable.tex}
Macro that indicates there is no next in the first syllable.

\macroname{\textbackslash GreSetTextAboveLines}{\#1}{gregoriotex-main.tex}
Macro to place argument above the lines and empty
\verb=\gre@currenttextabovelines= when done.

\begin{argtable}
	\#1 & string & Text to be placed above the lines.\\
\end{argtable}

\macroname{\textbackslash GreSetThisSyllable}{\#1\#2\#3}{gregoriotex-syllable.tex}
Macro to set the text of the current syllable.

\begin{argtable}
	\#1 & string & the first letters of the syllable, that don't count for the alignment\\
	\#2 & string & the middle letters of the syllable, we must align in the middle of them\\
	\#3 & string & the end letters, they don't count for alignment\\
\end{argtable}

\macroname{\textbackslash GreSharp}{\#1\#2\#3\#4\#5}{gregoriotex-signs.tex}
Macro to typeset a sharp.

\begin{argtable}
	\#1 & integer & Height number of the sharp.\\
	\#2 & \texttt{0} & The sharp is not part of the clef.\\
	& \texttt{1} & The sharp is part of the clef (doesn't happen).\\
	\#3 & \TeX\ code & signs to typeset before the glyph (typically additional bars, as they must be "behind" the glyph)\\
	\#4 & \TeX\ code & signs to typeset after the glyph (almost all signs)\\
	\#5 & string & the line, byte offset, and column address for textedit links when point-and-click is enabled\\
\end{argtable}

\macroname{\textbackslash GreSmallCaps}{\#1}{gregoriotex.sty and gregoriotex.tex}
Makes argument small capitals. Accesses \LaTeX\ \verb=\textsc= or
Plain \TeX\ \verb=\sc= as appropriate Corresponds to ``<sc></sc>'' tags
in gabc.

\begin{argtable}
	\#1 & string & Text to be typeset in small caps font.\\
\end{argtable}

\macroname{\textbackslash GreSlur}{\#1\#2\#3\#4\#5\#6}{gregoriotex-signs.tex}
Typesets a slur.

\begin{argtable}
	\#1 & integer & Height number of the pitch.\\
	\#2 & \texttt{-1} & The slur should appear under the note.\\
			& \texttt{1} & The slur should appear over the note.\\
	\#3 & \texttt{0} & The slur should start at the right end of the note.\\
			& \texttt{1} & The slur should start at one punctum's width to the left of the right end of the note.\\
			& \texttt{2} & The slur should start at one-half punctum's width to the left of the right end of the note.\\
	\#4 & string & The horizontal dimension of the slur.\\
	\#5 & string & The vertical dimension of the slur.\\
	\#6 & integer & Height number of the pitch.\\
\end{argtable}

\macroname{\textbackslash GreSpecial}{\#1}{gregoriotex-symbols.tex}
Typesets a special character.  If the \#1 special character wasn't defined by
\verb=\gresetspecial=, the text of \#1 will be output directly.

\begin{argtable}
	\#1 & string & The text between \texttt{<sp>} and \texttt{</sp>}.\\
\end{argtable}

\macroname{\textbackslash GreStar}{}{gregoriotex-symbol.tex}
Macro to typeset an asterisk (\GreStar).

\macroname{\textbackslash GreSupposeHighLedgerLine}{}{gregoriotex-spaces.tex}
Indicates that the system should act as if a ledger line exists above the staff.

\macroname{\textbackslash GreSupposeLowLedgerLine}{}{gregoriotex-spaces.tex}
Indicates that the system should act as if a ledger line exists below the staff.

\macroname{\textbackslash GreSyllable}{\#1\#2\#3\#4\#5\#6\#7\#8\#9}{gregoriotex-syllable.tex}
Macro to typeset the syllable.

\begin{argtable}
	\#1 & \TeX\ code & macro setting syllable letters for the current syllable\\
	\#2 & empty & reserved for future use\\
	\#3 & \TeX\ control sequence & the control sequence to use for styling the hyphen\\
	\#4 & \texttt{0} & this syllable is not the end of a word\\
	& \texttt{1} & this syllable is the end of a word\\
	\#5 & \TeX\ code & macros setting syllable letters for the next syllable\\
	\#6 & string & the line, byte offset, and column address for textedit links when point-and-click is enabled\\
	\#7 & & alignment type of the first next glyph\\
	\#8 &\TeX\ code & other macros (translation, double text, etc.) that don't fit in the limitation of the number of arguments\\
	\#9 & Gregorio\TeX\ glyphs & all the notes
\end{argtable}

\macroname{\textbackslash GreTilde}{}{gregoriotex-main.tex}
Macro to print $\sim$.

\macroname{\textbackslash GreTranslationCenterEnd}{}{gregoriotex-main.tex}
Macro to end the centering of the translation text.

\macroname{\textbackslash GreTypewriter}{\#1}{gregoriotex.sty and gregoriotex.tex}
Makes argument typewriter font.  Accesses \LaTeX\ \verb=\texttt= or
Plain \TeX\ \verb=\tt= as appropriate.

\begin{argtable}
	\#1 & string & Text to typeset in typewriter font.\\
\end{argtable}

\macroname{\textbackslash GreUnderBrace}{\#1\#2\#3\#4}{gregoriotex-signs.tex}
Macro to typeset a round brace below the lines.

\begin{argtable}
	\#1 & length & The width of the brace.\\
	\#2 & length & A vertical shift.\\
	\#3 & length & A horizontal shift.\\
	\#4 & \texttt{0} & Don't shift before starting the brace.\\
	& \texttt{1} & Shift back a punctum's width before starting the brace.
\end{argtable}

\macroname{\textbackslash GreUnderline}{\#1}{gregoriotex.sty and gregoriotex.tex}
Makes argument underlined under \LaTeX\ using \verb=\underline=.  Does
nothing in Plain \TeX.

\begin{argtable}
	\#1 & string & Text to typeset underlined.\\
\end{argtable}

\macroname{\textbackslash GreUnstyled}{\#1}{gregoriotex-syllable.tex}
Returns its argument as-is.

\begin{argtable}
	\#1 & string & Text to typeset without any extra styling.\\
\end{argtable}

\macroname{\textbackslash GreUpcomingNewLineForcesCustos}{\#1}{gregoriotex-syllable.tex}
Indicates that the new line in the next syllable forces a custos.

\begin{argtable}
	\#1 & \texttt{0} & The custos is forced off.\\
			& \texttt{1} & The custos is forced on.\\
\end{argtable}

\macroname{\textbackslash GreVarBraceLength}{\#1}{gregoriotex-signs.tex}
Returns the computed length of the given brace or ledger line.

\begin{argtable}
	\#1 & string & unique identifier for the brace within the score.
\end{argtable}

\macroname{\textbackslash GreVarBraceSavePos}{\#1\#2\#3}{gregoriotex-signs.tex}
Records positions to compute the lengths of variable-sized braces and ledger lines.

\begin{argtable}
	\#1 & string & unique identifier for the brace within the score.\\
	\#2 & \texttt{0} & Don't shift before recording the position.\\
	& \texttt{1} & Shift back a punctum's width before recording the position.\\
	& \texttt{2} & Shift back one-half a punctum's width before recording the position.\\
	\#3 & \texttt{1} & Position to save is the start of brace.\\
	& \texttt{2} & Position to save is the end of brace.
\end{argtable}

\macroname{\textbackslash GreVEpisema}{\#1\#2}{gregoriotex-signs.tex}
Macro for typesetting the vertical episema.

\begin{argtable}
	\#1 & integer & Height number of episema.\\
	\#2 & string  & Type of glyph the episema is attached to. See \nameref{EpisemaSpecial} argument for description of options.\\
\end{argtable}

\macroname{\textbackslash GreVirgula}{\#1\#2}{gregoriotex-signs.tex}
Macro to typeset a virgula.

\begin{argtable}
	\#1 & \texttt{0} & There is no text under the bar.\\
	& \texttt{1} & There is text under the bar.\\
	\#2 & code & Macros which may happen before the skip but after the virgula (typically \verb=\grevepisema=).\\
\end{argtable}

\macroname{\textbackslash GreWriteTranslation}{\#1}{gregoriotex-main.tex}
Macro to typeset argument in the translation position.

\begin{argtable}
	\#1 & string & Text to typeset in the translation.\\
\end{argtable}

\macroname{\textbackslash GreWriteTranslationWithCenterBeginning}{\#1}{gregoriotex-main.tex}
Macro to typeset argument (a string) in the translation position (at
the beginning of a line?).

\begin{argtable}
	\#1 & string & Text to typeset in the translation (at the beginning of a line).\\
\end{argtable}

\macroname{\textbackslash GreZeroHyph}{}{gregoriotex-main.tex}
Macro to typeset a zero-width hyphen (the hyphen is visible, it is only
treated as if it had 0 width when calculating spaces).  Used for fine tuning spacing
(especially at line endings).

\macroname{\textbackslash GreForceBreak}{}{gregoriotex-spaces.tex}
Macro used to force a line break to occur at a given position.

\macroname{\textbackslash GreNoBreak}{}{gregoriotex-spaces.tex}
Macro used to prevent a line break from occurring at a given position.

\macroname{\textbackslash GreScoreId}{}{gregoriotex-main.tex}
A Lua\TeX\ attribute which designates a unique identifier for each score.

\macroname{\textbackslash GreNABCNeumes}{\#1\#2}{gregoriotex-nabc.tex}
Macro to print a nabc character above the lines.

\begin{argtable}
	\#1 & integer & the line on which the character should appear (currently unused)\\
	\#2 & string & The \texttt{nabc} syntax which indicates what neumes are to be printed\\
\end{argtable}

\macroname{\textbackslash GreNABCChar}{\#1}{gregoriotex-nabc.tex}
Macro to print a nabc character.

\begin{argtable}
	\#1 & string & The \texttt{nabc} syntax which indicates what neumes are to be printed\\
\end{argtable}

\macroname{\textbackslash GreScoreNABCLines}{\#1}{gregoriotex-nabc.tex}
Macro which sets the number of \texttt{nabc} lines in the score.

\begin{argtable}
	\#1 & integer & the number of \texttt{nabc} lines (currently only 1 is supported)\\
\end{argtable}


\macroname{\textbackslash GreModeNumber}{\#1}{gregoriotex-main.tex}
Macro which formats the mode in roman or arabic numerals according to the appropriate setting.

\begin{argtable}
	\#1 & \texttt{1}--\texttt{8} & The mode to be formated\\
\end{argtable}

%%% Local Variables:
%%% mode: latex
%%% TeX-master: "GregorioRef"
%%% End:

% !TEX root = GregorioRef.tex
% !TEX program = LuaLaTeX+se
\section{Gregorio\TeX{} Controls}

These functions are the ones used by Gregorio\TeX{} internally as it
process the commands listed above.  They should not appear in any user
document and are listed here for programmer documentation purposes
only.

\macroname{\textbackslash gre@error}{\#1}{gregoriotex.sty \textup{and} gregoriotex.tex}
Prints an error to the \TeX\ output log.

\begin{argtable}
	\#1 & string & error message\\
\end{argtable}


\macroname{\textbackslash gre@warning}{\#1}{gregoriotex.sty \textup{and} gregoriotex.tex}
Prints a warning to the \TeX\ output log.

\begin{argtable}
	\#1 & string & warning message\\
\end{argtable}

\macroname{\textbackslash gre@metapost}{\#1}{gregoriotex.sty \textup{and} gregoriotex.tex}
Executes \MP{} commands using luamplib.

\begin{argtable}
	\#1 & \MP{} commands & The \MP{} commands to execute.
\end{argtable}

\macroname{\textbackslash gre@deprecated}{\#1\#2}{gregoriotex-main.tex}
Macro that handles deprecation messages. By default, deprecated macros
are allowed and a warning is printed. If the package option
\texttt{deprecated=false} is set, then deprecated macros raise a
package error, halting \TeX.

\begin{argtable}
	\#1 & string & name of the deprecated macro\\
	\#2 & string & name of the correct macro to use\\
\end{argtable}

\macroname{\textbackslash gre@obsolete}{\#1\#2}{gregoriotex-main.tex}
Macro that handles obsolescence errors.

\begin{argtable}
	\#1 & string & name of the obsolete macro\\
	\#2 & string & name of the correct macro to use\\
\end{argtable}

\macroname{\textbackslash gre@loadgregoriofont}{}{gregoriotex-main.tex}
Loads the chosen font for the neumes at the correct size.

\macroname{\textbackslash gre@calculate@constantglyphraise}{}{gregoriotex-spaces.tex}
Macro to calculate \verb=\gre@constantglyphraise=

\macroname{\textbackslash gre@addtranslationspace}{}{gregoriotex-spaces.tex}
Macro to tell Gregorio to set space for the translation.

\macroname{\textbackslash gre@removetranslationspace}{}{gregoriotexspaces.tex}
Macro to tell Gregorio to remove the space allocated to the translation.

\macroname{\textbackslash gre@calculate@additionalspaces}{\#1\#2\#3\#4}{gregoriotex-spaces.tex}
Macro which calculates \verb=\gre@additionalbottomspace= and\\
\verb=\gre@additionaltopspace=

\begin{argtable}
	\#1 & integer & the height number of the top pitch, including signs\\
	\#2 & integer & the height number of the bottom pitch, including signs\\
	\#3 & \texttt{0} & there is no translation line\\
	& \texttt{1} & there is a translation line\\
	\#4 & \texttt{0} & there is no above lines text\\
	& \texttt{1} & there is above lines text
 \end{argtable}

\macroname{\textbackslash gre@calculate@textaligncenter}{\#1\#2\#3}{gregoriotex-spaces.tex}
Macro for calculating \verb=\gre@textaligncenter=.

\begin{argtable}
	\#1 & string & The first part of the syllable (any preceding consonants in Latin).\\
	\#2 & string & The middle part of the syllable (the vowel in Latin, the whole syllable in English).\\
	\#3 & \texttt{0} & Calculation is being performed for the current syllable.\\
	& \texttt{1} & Calculation is being performed for the next syllable.\\
\end{argtable}

\macroname{\textbackslash gre@calculate@annotationtrueraise}{}{gregoriotex-spaces.tex}
Macro to calculate \verb=\gre@dimen@annotationtrueraise=.

\macroname{\textbackslash gre@calculate@commentarytrueraise}{}{gregoriotex-spaces.tex}
Macro to calculate \verb=\gre@dimen@commentarytrueraise=.

\macroname{\textbackslash gre@calculate@textlower}{}{gregoriotex-spaces.tex}
Calculates the value of \texttt{textlower}.  Default is \texttt{spacebeneathtext}.

\macroname{\textbackslash gre@calculate@linewidth}{}{gregoriotex-spaces.tex}
Calculates the line width.  Default is the width of the printable space (\verb=\hsize=).

\macroname{\textbackslash gre@calculate@stafflinewidth}{}{gregoriotex-spaces.tex}
Calculates the width of the staff lines.  Default is \texttt{linewidth}.

\macroname{\textbackslash gre@calculate@stafflineheight}{}{gregoriotex-spaces.tex}
Calculates the height (thickness) of the staff lines.  Dependent on \texttt{stafflineheightfactor} and \texttt{gre@factor}.

\macroname{\textbackslash gre@calculate@interstafflinespace}{}{gregoriotex-spaces.tex}
Calculates the distance between the staff lines.  Dependent on \texttt{stafflineheight} and \texttt{gre@factor}

\macroname{\textbackslash gre@calculate@stafflinediff}{}{gregoriotex-spaces.tex}
Calculates a correction factor for when the staff lines are not their default thickness.  Dependent on \texttt{stafflineheight} and \texttt{gre@factor}.

\macroname{\textbackslash gre@calculate@staffheight}{}{gregoriotex-spaces.tex}
Calculates the total height of the staff.  Dependent on \texttt{stafflineheight} and \texttt{interstafflinespace}.

\macroname{\textbackslash gre@calculate@constantglyphraise}{}{gregoriotex-spaces.tex}
Calculates the baseline correction for the glyphs.  Dependent on \texttt{gre@factor}, \texttt{additionalbottomspace}, \texttt{spacebeneathtext}, \texttt{spacelinestext}, \texttt{interstafflinespace}, \texttt{stafflineheight}, \texttt{currenttranslationheight}, and \texttt{stafflinediff}.

\macroname{\textbackslash gre@computespaces}{}{gregoriotex-spaces.tex}
Aggregates all of the global distance calculations and calls them in the order needed to respect dependencies.

\macroname{\textbackslash gre@calculate@glyphraisevalue}{\#1\#2}{gregoriotex-spaces.tex}
Calculates the raise value for a glyph (glyphraisevalue) based on where it is to be placed and what kind of a glyph it is.  This is a time of use calculation.

\begin{argtable}
	\#1 & integer & The number for where the glyph is located.  \texttt{a} in gabc is \texttt{1}, \texttt{b} is \texttt{2}, \etc\\
	\#2 & \texttt{0} & no modification\\
	& \texttt{1} & puts the value on the interline just above if it is on a line\\
	& \texttt{2} & puts the value on the interline just beneath if it is on a line\\
	& \texttt{3} & case of the vertical episema, which is not placed at the same place if the corresponding note is on a line or not\\
	& \texttt{4} & case of the punctum mora, for the same reason\\
	& \texttt{5} & case of the horizontal episema under a note, that must be placed a bit lower if the note is on a line\\
	& \texttt{6} & case of the signs above (accentus, \etc)\\
	& \texttt{8} & case of the punctum mora of the first note of a podatus or the 2nd note of a porrectus, \etc\\
	& \texttt{9} & case of the horizontal episema, that must be placed a bit lower if the note is on a line\\
	& \texttt{10} & case of the low choral sign\\
	& \texttt{11} & case of the high choral sign\\
	& \texttt{12} & case of the low choral sign which is lower than the note\\
	& \texttt{13} & case of the brace above the bars\\
	& \texttt{14} & case of the punctum mora in a space with a note on the line below it\\
	& \texttt{15} & case of the over-the-notes slur\\
	& \texttt{16} & case of the under-the-notes slur\\
\end{argtable}

\macroname{\textbackslash gre@stafflinefactor}{}{gregoriotex-spaces.tex}
A number indicating the thickness of the staff lines.

\macroname{\textbackslash gre@calculate@textaligncenter}{\#1\#2\#3}{gregoriotex-spaces.tex}
Macro to calculate the distance from the beginning of the text of a syllable to its alignment point (the center of the vowel for Latin centering, the center of the syllable for English centering).  This is a time of use calculation.

\begin{argtable}
	\#1 & string & the first part of the syllable\\
	\#2 & string & the middle part of the syllable\\
	\#3 & \texttt{0} & perform this calculation for the current syllable\\
	& \texttt{1} & perform this calculation for the next syllable
\end{argtable}

\macroname{\textbackslash gre@calculate@enddifference}{\#1\#2\#3\#4\#5}{gregoriotex-spaces.tex}
Calculates the difference between the end of the notes and the end of the syllable text.  Also stores the value for the previous syllable if needed.  This is a time of use calculation.

\begin{argtable}
	\#1 & length & the total width of the notes\\
	\#2 & length & the total width of the syllable text\\
	\#3 & length & the alignment distance for the text (\texttt{textaligncenter})\\
	\#4 & length & the alignment distance for the notes (\texttt{notesaligncenter})\\
	\#5 & \texttt{0} & do not save the value for the previous syllable before calculating the new value\\
	& \texttt{1} & save the value for the previous syllable before calculating the new value
\end{argtable}

\macroname{\textbackslash gre@changeonedimenfactor}{\#1\#2\#3}{gregoriotex-spaces.tex}
Change the scale of a single distance from one factor to another.

\begin{argtable}
	\#1 & string & name of the distance to be scaled.  See \nameref{distances}.\\
	\#2 & integer & the factor the distance is currently in\\
	\#3 & integer & the factor the distance is to be put into\\
\end{argtable}

\macroname{\textbackslash gre@changedimenfactor}{\#1\#2}{gregoriotex-spaces.tex}
Rescales all the distances (and \texttt{stafflinefactor}) which are supposed to scale with a change in staff size.

\begin{argtable}
	\#1 & integer & the factor the distances are currently in\\
	\#2 & integer & the factor the distances are to be put into\\
\end{argtable}

\macroname{\textbackslash gre@calculate@nextbegindifference}{\#1\#2\#3\#4}{gregoriotex-spaces.tex}
Macro to calculate \texttt{nextbegindifference}.

\begin{argtable}
	\#1 & string & the first letters of the next syllable\\
	\#2 & string & the middle letters of the next syllable (the vowel in Latin, the whole syllable in English)\\
	\#3 & string & the end letters of the next syllable\\
	\#4 & $0 \le$ integer $\le 19$ & the type of notes alignment.  See \nameref{notesalign}.\\
	& $20 \le$ integer $\le 39$ & Same as below 20 except there is a flat before the notes.  Subtract 20 to get the type of notes alignment.\\
	& $40 \le$ integer $\le 59$ & Same as below 20 except there is a natural before the notes.  Subtract 40 to get the type of notes alignment.
\end{argtable}

\macroname{\textbackslash gre@strip@pt}{\#1}{gregoriotex.sty \textup{and} gregoriotex.tex}
Strips the units from a distance.  Under \LaTeX{}, this is an alias to \verb=\strip@pt=.

\begin{argtable}
	\#1 & control sequence & should be the control sequence for the the distance register (including the leading backslash)\\
\end{argtable}

\macroname{\textbackslash gre@rem@pt}{\#1}{gregoriotex.tex}
Strips the units from a distance.  Used internally by \verb=\gre@strip@pt=.  Under \LaTeX{}, this is not defined.

\begin{argtable}
	\#1 & distance & should be in the form ``[0-9]+.[0-9]+pt’’. (\ie the result of applying \verb=\the= to a distance register)\\
\end{argtable}

\macroname{\textbackslash gre@count@temp@...}{}{gregoriotex-spaces.tex}
Temporary count used in calculations.  There are currently three of these.

\macroname{\textbackslash gre@convertto}{\#1\#2}{gregoriotex-spaces.tex}
Macro which converts a distance into a particular set of units.  Result is placed in \verb=\gre@converted= as a string.

\begin{argtable}
	\#1 & string & two letter abbreviation for the units.  Should recognize all legal \TeX\ units.\\
	\#2 & distance & Distance to be converted.
\end{argtable}

\macroname{\textbackslash gre@converted}{}{gregoriotex-spaces.tex}
Macro holding result of last call to \verb=\gre@convertto=.

\macroname{\textbackslash gre@consistentunits}{\#1\#2}{gregoriotex-spaces.tex}
This function takes a distance and formats it as a string so that its units conform to the pattern set by a string representation of a distance.  Result is placed in \verb=\gre@stringdist=.

\begin{argtable}
	\#1 & string & the standard whose format is to be matched.\\
	\#2 & distance & the distance to be adjusted.
\end{argtable}

\macroname{\textbackslash gre@stringdist}{}{gregoriotex-spaces.tex}
Macro holding result of last call to \verb=\gre@consistentunits=.

\macroname{\textbackslash gre@gregorioscore}{\#1}{gregoriotex-main.tex}
Macro that handles \verb=\gregorioscore= calls when they do not have an
optional argument.

\begin{argtable}
	\#1 & string & Relative or absolute path to the score.\\
\end{argtable}

\macroname{\textbackslash gre@gregorioscore@option}{[\#1]\#2}{gregoriotex-main.tex}
Macro that handles \verb=\gregorioscore= calls when they have an optional
argument.

\begin{argtable}
	\#1 & \texttt{n} & \#2 will be included as is. \\
			& \texttt{a} & Gregorio\TeX\ will automatically compile gabc files if necessary.\\
			& \texttt{f} & Forces Gregorio\TeX\ to compile the gabc file.\\
	\#2 & string & Relative or absolute path to the score.\\
\end{argtable}

\macroname{\textbackslash gre@gabcsnippet}{\#1}{gregoriotex-main.tex}
Macro that handles \verb=\gabcsnippet= calls when they do not have an
optional argument.

\begin{argtable}
	\#1 & string & Snippet of gabc code.\\
\end{argtable}

\macroname{\textbackslash gre@writemode}{\#1\#2\#3}{gregoriotex-main.tex}
Macro that writes its arguments with \verb=\greannotation=.  This
macro is typically called by \verb=\GreMode= in the gtex file.

\begin{argtable}
	\#1 & \TeX\ code & Mode text to place above the initial of a score in the \texttt{modeline} style.\\
	\#2 & \TeX\ code & Arbitrary code to typeset, in the \texttt{modemodifier} style, after the mode text.\\
	\#3 & \TeX\ code & Arbitrary code to typeset, in the \texttt{modedifferentia} style, after \#2.\\
\end{argtable}

\macroname{\textbackslash gre@setallbracerendering}{\#1}{gregoriotex-signs.tex}
Macro used by \verb=\gresetbracerendering= to change all braces.

\begin{argtable}
	\#1 & \texttt{metapost} & \MP{} will be used to render braces\\
			& \texttt{font} & The score font will be used to render braces\\
\end{argtable}

\macroname{\textbackslash gre@setbracerendering}{[\#1]\#2}{gregoriotex-signs.tex}
Macro used by \verb=\gresetbracerendering= to change a single type of brace.

\begin{argtable}
	\#1 & \texttt{brace} & change round braces that appear over the staff\\
			& \texttt{underbrace} & change round braces that appear under the staff\\
			& \texttt{curlybrace} & change curly braces\\
			& \texttt{barbrace} & change round braces that appear over divisio bars\\
	\#2 & \texttt{metapost} & \MP{} will be used to render braces\\
			& \texttt{font} & The score font will be used to render braces\\
\end{argtable}

\macroname{\textbackslash gre@@setbracerendering}{\#1\#2}{gregoriotex-signs.tex}
Secondary macro used by \verb=\gre@setallbracerendering= and
\verb=\gre@setbracerendering= to change a single type of brace.  As a
secondary macro, it doesn't check its first argument.

\begin{argtable}
	\#1 & \texttt{brace} & change round braces that appear over the staff\\
			& \texttt{underbrace} & change round braces that appear under the staff\\
			& \texttt{curlybrace} & change curly braces\\
			& \texttt{barbrace} & change round braces that appear over divisio bars\\
	\#2 & \texttt{metapost} & \MP{} will be used to render braces\\
			& \texttt{font} & The score font will be used to render braces\\
\end{argtable}

\macroname{\textbackslash gre@brace@common}{\#1\#2\#3\#4\#5\#6\#7}{gregoriotex-signs.tex}
Common macro used internally to render braces.

\begin{argtable}
	\#1 & length  & The width of the brace.\\
	\#2 & length  & A vertical shift.\\
	\#3 & length  & A horizontal shift.\\
	\#4 & \texttt{0} & Don’t shift before starting the brace.\\
	& \texttt{1} & Shift back a punctum’s width before starting the brace.\\
	\#5 & \texttt{0} & No accentus above the brace.\\
	& \texttt{1} & Typeset an accentus above the brace.\\
	\#6 & integer & The height number for the brace.\\
	\#7 & csname  & The control sequence name representing the brace.
\end{argtable}

\macroname{\textbackslash gre@render@barbrace}{}{gregoriotex-signs.tex}
Draws a divisio brace.

\macroname{\textbackslash grebracemetapostpreamble}{\#1}{gregoriotex-signs.tex}
Returns the \MP{} preamble for braces.  The control sequence name does
not have the \texttt{@} symbol because this macro is used within \MP{}.

\begin{argtable}
	\#1 & string & the width of the brace; if \texttt{*}, use the bar brace width.
\end{argtable}

\macroname{\textbackslash gre@draw@curlybrace}{\#1}{gregoriotex-signs.tex}
Draws a curly over-brace using \MP{}.

\begin{argtable}
	\#1 & length & the width of the brace.
\end{argtable}

\macroname{\textbackslash gre@draw@brace}{\#1}{gregoriotex-signs.tex}
Draws a round over-brace using \MP{}.

\begin{argtable}
	\#1 & string & the width of the brace; if \texttt{*}, use the bar brace width.
\end{argtable}

\macroname{\textbackslash gre@render@fontbrace}{\#1\#2}{gregoriotex-signs.tex}
Draws a brace using the score font.

\begin{argtable}
	\#1 & string & the width of the brace.\\
	\#2 & \TeX\ code & \TeX\ code that renders the brace using the score font.\\
\end{argtable}

\macroname{\textbackslash gre@draw@underbrace}{\#1}{gregoriotex-signs.tex}
Draws a round under-brace using \MP{}.

\begin{argtable}
	\#1 & length & the width of the brace.
\end{argtable}

\macroname{\textbackslash gre@draw@roundbrace}{\#1\#2\#3}{gregoriotex-signs.tex}
Draws a round over- or under-brace using \MP{}.

\begin{argtable}
	\#1 & length         & the width of the brace.\\
	\#2 & number         & the height of the bounding box in em-relative units.\\
	\#3 & \MP{} commands & \MP{} commands to draw the brace outline.
\end{argtable}

\macroname{\textbackslash gre@draw@slur}{\#1\#2\#3}{gregoriotex-signs.tex}
Draws a slur using \MP{}.

\begin{argtable}
	\#1 & length      & the x-dimension of the slur.\\
	\#2 & length      & the y-dimension of the slur.\\
	\#3 & \texttt{-1} & draw an under-the-notes slur.\\
			& \texttt{1}  & draw an over-the-notes slur.\\
\end{argtable}


\macroname{\textbackslash gre@iflatex}{\#1}{gregoriotex.sty \textup{and} gregoriotex.tex}
Evaluates to \verb=#1= if running under \LaTeX{}.

\begin{argtable}
	\#1 & \TeX{} code & the \TeX{} code to use if running under \LaTeX{}.
\end{argtable}

\macroname{\textbackslash gre@latex@barredsymbol}{\#1\#2\#3\#4}{gregoriotex-signs.tex}
Internal method used by \verb=\grelatexsimpledefarredsymbol= to simplify
the cascading of conditionals used to implement that macro.

\bigskip\textbf{Only available in \LaTeX.}

\begin{argtable}
	\#1 & string      & the value of \verb=\f@series/\f@shape= to match.\\
	\#2 & \TeX{} code & the \TeX{} for the base symbol (\ie, \texttt{A}, \texttt{R}, or \texttt{V}).\\
	\#3 & string      & the control sequence name created by \verb=\gredefsizedsymbol= to use for the bar.\\
	\#4 & dimension   & the amount to shift the bar to the left from the end of the base symbol.
\end{argtable}

\macroname{\textbackslash gre@additionalbottomcustoslineend}{}{gregoriotex-signs.tex}
Macro to place a bottom custos with an additional line (positions \texttt{a} and \texttt{b}) at the end of a line.

\macroname{\textbackslash gre@additionalbottomcustoslinemiddle}{}{gregoriotex-signs.tex}
Macro to place a bottom custos with an additional line (positions \texttt{a} and \texttt{b}) in the middle of a line.

\macroname{\textbackslash gre@additionaltopcustoslineend}{}{gregoriotex-signs.tex}
Macro to place a top custos with an additional line (positions \texttt{l} and \texttt{m}) at the end of a line.

\macroname{\textbackslash gre@additionaltopcustoslinemiddle}{}{gregoriotex-signs.tex}
Macro to place a top custos with an additional line (positions \texttt{l} and \texttt{m}) in the middle of a line.

\macroname{\textbackslash gre@pickcustos}{\#1}{gregoriotex-signs.tex}
Macro to pick the appropriate custos character.

\begin{argtable}
	\#1 & integer & height of the custos character to be placed\\
\end{argtable}

\macroname{\textbackslash gre@nextcustos}{}{gregoriotex-signs.tex}
Macro that saves the next custos height.

\macroname{\textbackslash gre@beginnotes}{}{gregoriotex-main.tex}
Macro to draw the staff lines.  Comes after the initial but before the clef.

\macroname{\textbackslash gre@noinitial}{}{gregoriotex-main.tex}
Macro called when no initial is being set.

\macroname{\textbackslash gre@setbiginitial}{}{gregoriotex-main.tex}
Macro which indicates that a 2-line initial is desired.

\macroname{\textbackslash gre@setinitial}{\#1}{gregoriotex-main.tex}
Macro to set the initial in the score.

\macroname{\textbackslash gre@adjustsecondline}{}{gregoriotex.tex}
Macro to call before first syllable, but after \verb=\GreSetInitialClef=.

\macroname{\textbackslash gre@adjustthirdline}{}{gregoriotex-main.tex}
Macro to call during the second line.

\macroname{\textbackslash gre@adjustlineifnecessary}{}{gregoriotex-main.tex}
Macro that calls \verb=\gre@adjustthirdline= if indicated by \verb=\ifgre@thirdlineadjustmentnecessary=.

\macroname{\textbackslash gre@addspaceabove}{}{gregoriotex-main.tex}
Macro to increase the space above the lines to account for above lines text.

\macroname{\textbackslash gre@removespaceabove}{}{gregoriotex-main.tex}
Macro to decrease the space above the lines as there is no longer any above lines text.

\macroname{\textbackslash gre@alteration}{\#1\#2\#3\#4\#5\#6\#7}{gregoriotex-signs.tex}
Macro to typeset an alteration.

\begin{argtable}
	\#1 & integer & height of the alteration\\
	\#2 & character alias & the alteration\\
	\#3 & character alias & the hole of the alteration\\
	\#4 & \texttt{1} & the alteration is part of the clef\\
	& \texttt{0} & the alteration is not part of the clef\\
	\#5 & \TeX\ code & signs to typeset before the glyph (typically additional bars, as they must be "behind" the glyph)\\
	\#6 & \TeX\ code & signs to typeset after the glyph (almost all signs)\\
	\#7 & string & the line, byte offset, and column address for textedit links when point-and-click is enabled\\
\end{argtable}

\macroname{\textbackslash gre@clef}{}{gregoriotex-signs.tex}
Macro holding the current clef type.

\macroname{\textbackslash gre@clefheight}{}{gregoriotex-signs.tex}
Macro holding the current clef line.

\macroname{\textbackslash gre@clefflatheight}{}{gregoriotex-signs.tex}
Macro to hold the height of the current flat for the clef (\texttt{3} if no flat).

\macroname{\textbackslash gre@cleftwo}{}{gregoriotex-signs.tex}
Macro holding the current secondary clef type.

\macroname{\textbackslash gre@cleftwoheight}{}{gregoriotex-signs.tex}
Macro holding the current secondary clef line (or 0 for no secondary clef).

\macroname{\textbackslash gre@cleftwoflatheight}{}{gregoriotex-signs.tex}
Macro to hold the height of the current flat for the secondary clef (\texttt{3} if no flat).

\macroname{\textbackslash gre@updatelinesclef}{}{gregoriotex-signs.tex}
Macro redrawing a key from \verb=\gre@clefnum=, useful for vertical space changes.

\macroname{\textbackslash gre@currenttextabovelines}{}{gregoriotex-main.tex}
Macro for storing the text which needs to be placed above the lines.

\macroname{\textbackslash gre@typesettextabovelines}{\#1}{gregoriotex-main.tex}
Macro for typesetting the text above the lines.

\macroname{\textbackslash gre@dotranslationcenterend}{}{gregoriotex-main.tex}
Macro to typeset a centered translation.

\macroname{\textbackslash gre@drawfirstlines}{}{gregoriotex-main.tex}
Macro to draw the first set of lines in a score (when shortened by an initial).

\macroname{\textbackslash gre@generatelines}{}{gregoriotex-main.tex}
Macro to (re)populate the box containing the lines.

\macroname{\textbackslash gre@updatelinewidth}{}{gregoriotex-main.tex}
Macro to shorten the lines to account for the presence of the initial.

\macroname{\textbackslash gre@knownline}{}{gregoriotex-main.tex}
A count which keeps track of which line of the score we’re on.

\macroname{\textbackslash gre@lastoflinecount}{}{gregoriotex-main.tex}
Count to track where on the line we are.  Values are \texttt{0} (we are not near the end of a line), \texttt{1} (we’re at the last syllable of the line), and \texttt{2} (we just set the last syllable of the line and so are at the first syllable of a new line).

\macroname{\textbackslash gre@savedlastoflinecount}{}{gregoriotex-main.tex}
A spot to save the \verb=\gre@lastoflinecount= so we can change it temporarily and revert to the saved value later.

\macroname{\textbackslash gre@newlinecommon}{\#1}{gregoriotex-main.tex}
The macro which needs to be called each time a new lines is started.

\begin{argtable}
	\#1 & \texttt{0} & Justifying the line being ended\\
	& \texttt{1} & Do not justify the line being ended\\
\end{argtable}

\macroname{\textbackslash gre@endafterbar}{\#1}{gregoriotex-main.tex}
Macro to call after ending a bar.

\begin{argtable}
	\#1 & \texttt{0} & We are at the end of a line\\
	& \texttt{1} & We not at the end of a line\\
\end{argtable}

\macroname{\textbackslash gre@endofsyllable}{\#1\#2\#3}{gregoriotex-syllable.tex}
Macro called at end of a syllable, adds a penalty and a space.

\begin{argtable}
	\#1 & \texttt{0} & to only add the penalty\\
	& \texttt{1} & adds both penalty and space\\
	\#2 & \texttt{0} & if end of syllable\\
	& \texttt{1} & if end of word\\
	\#3 & \texttt{1} & if next syllable is a bar\\
	& \texttt{0} & otherwise\\
\end{argtable}

\macroname{\textbackslash gre@setfirstsyllabletext}{\#1\#2\#3\#4\#5\#6}{gregoriotex-syllable.tex}
Internal macro to set the first syllable text after all parts are known.

\begin{argtable}
	\#1 & \TeX\ code & First part of the syllable (before the vowel)\\
	\#2 & \TeX\ code & Middle part of the syllable (the vowel)\\
	\#3 & \TeX\ code & Last part of the syllable (after the vowel)\\
	\#4 & \TeX\ code & First letter of the syllable\\
	\#5 & \TeX\ code & Everything after the first letter of the syllable\\
	\#6 & \TeX\ code & Macros to run after the text is emitted\\
\end{argtable}

\macroname{\textbackslash gre@opening@syllabletext}{}{gregoriotex-syllable.tex}
Macro that stores the computed \TeX\ code for rendering the text of the first syllable.

\macroname{\textbackslash gre@exhyphencharsave}{}{gregoriotex-main.tex}
Macro for saving the ex hyphen character so that it can be restored at the end of the score.

\macroname{\textbackslash gre@factor}{}{gregoriotex-main.tex}
Count which stores the current staff size.

Default: 17 (approximately the size found in graduals)

\macroname{\textbackslash gre@fillhole}{\#1}{gregoriotex-signs.tex}
Macro to fill the hole in a glyph so that staff lines do not show through a hole in it.

\begin{argtable}
	\#1 & Gregorio\TeX\ char & character to use to fill the hole\\
\end{argtable}

\macroname{\textbackslash gre@calculate@notesaligncenter}{\#1}{gregoriotex-syllable.tex}
Macro to find the alignment center for a group of notes.  The value is the distance from the left edge of the group to the alignment point and is stored in \verb=\gre@dimen@notesaligncenter=.

\begin{argtable}
	\#1 & Note alignment type & See \ref{notesalign}\\
\end{argtable}

\macroname{\textbackslash gre@calculate@nextnotesaligncenter}{\#1}{gregoriotex-syllable.tex}
Same as previous, but for the next syllable.

\begin{argtable}
	\#1 & Note alignment type & See \ref{notesalign}\\
\end{argtable}

\macroname{\textbackslash gre@calculate@simplenotesaligncenter}{\#1\#2}{gregoriotex-syllable.tex}
Workhorse function behind \verb=\gre@calculate@notesaligncenter= and\\ \verb=\gre@calculate@nextnotesaligncenter=.

\begin{argtable}
	\#1 & Note alignment type & See \ref{notesalign}\\
	\#2 & \texttt{0} & this is for the current syllable\\
	& \texttt{1} & this is for the next syllable\\
\end{argtable}

\macroname{\textbackslash gre@gregoriofontname}{}{gregoriotex-main.tex}
Macro which stores the name of the currently selected font for the neumes.

\macroname{\textbackslash gre@handleclivisspecialalignment}{\#1\#2\#3}{gregoriotex-syllable.tex}
Macro for aligning clivis syllables according to the flag \verb=\gre@clivisalignment=.

\begin{argtable}
	\#1 & Gregorio\TeX\ glyph & Glyph to use when aligning clivis on its center\\
	\#2 & Gregorio\TeX\ glyph & Glyph to use when aligning clivis on the center of the first punctum\\
	\#3 &  \texttt{0} & this is for the current syllable\\
	& \texttt{1} & this is for the next syllable\\
\end{argtable}

\macroname{\textbackslash gre@hepisorline}{\#1\#2\#3\#4\#5}{gregoriotex-signs.tex}
Macro to typeset a horizontal line (either an additional staff line or an episema).

\begin{argtable}
	\#1 & character & The letter of the height of the episema (not the height of the note it corresponds to).\\
	\#2 & \texttt{0} & go back to the beginning of the previous glyph; this starts the episema at the beginning of the previous glyph\\
	& \texttt{1} & stay at the end of the glyph; doesn’t make much sense to use this\\
	& \texttt{2} & go back the width of \#1; this starts the episema at the glyph from the end that starts at \#1’s width from the end\\
	& \texttt{3} & go back to the beginning of the previous glyph and then forward the width of \#1; this starts the episema at the glyph from the start that starts just after \#1’s width from the start\\
	& \texttt{4} & go back to the beginning of the previous glyph and then forward the width of \#1, then back the width of \#2; this ends the episema at the end of \#1\\
	\#3 & integer &the ambitus for a two note episema at the diagonal stroke of a porrectus, porrectus flexus, orculus resupinus, or torculus resupinus flexus\\
	\#4 & \texttt{0} & an horizontal episema\\
	& \texttt{1} & an horizontal episema under a note\\
	& \texttt{2} & a line at the top\\
	& \texttt{3} & a line at the bottom\\
	\#5 & \texttt{f} & a normal episema\\
	& \texttt{l} & a small episema aligned left\\
	& \texttt{c} & a small episema aligned center\\
	& \texttt{r} & for a small episema aligned right\\
\end{argtable}

\macroname{\textbackslash gre@hepisorlineaux}{\#1\#2\#3\#4}{gregoriotex-signs.tex}
Macro that will help in the typesetting of a horizontal episema and additional lines.

\begin{argtable}
	\#1 & Gregorio\TeX\ glyph & an offset glyph (see \#3, below)\\
	\#2 & Gregorio\TeX\ glyph & the episema glyph\\
	\#3 & \texttt{0} & go back to the beginning of the previous glyph; this starts the episema at the beginning of the previous glyph\\
	& \texttt{1} & stay at the end of the glyph; doesn’t make much sense to use this\\
	& \texttt{2} & go back the width of \#1; this starts the episema at the glyph from the end that starts at \#1’s width from the end\\
	& \texttt{3} & go back to the beginning of the previous glyph and then forward the width of \#1; this starts the episema at the glyph from the start that starts just after \#1’s width from the start\\
	\#4 & \texttt{0} & an horizontal episema\\
	& \texttt{1} & an horizontal episema under a note\\
	& \texttt{2} & a line at the top\\
	& \texttt{3} & a line at the bottom\\
\end{argtable}

\macroname{\textbackslash gre@vepisemaorrare}{\#1\#2\#3\#4\#5}{gregoriotex-signs.tex}
Macro to typeset a vertical episema or a rare accent (like accentus, circulus, etc.).  This function must be called after a call to \verb=\GreGlyph=.

\begin{argtable}
	\#1 & character & the letter of the height of the episema (not the height of the note it corresponds to.\\
	\#2 & integer & See \nameref{EpisemaSpecial}\\
	\#3 & Gregorio\TeX\ glyph & the sign glyph\\
	\#4 & \texttt{1} & vertical episema\\
	& \texttt{2} & rare sign\\
	& \texttt{3} & choral sign\\
	& \texttt{4} & brace above the bar\\
	\#5 & string & the choral sign, if relevant\\
\end{argtable}

\macroname{\textbackslash gre@vepisemaorrareaux}{\#1\#2\#3\#4\#5\#6\#7}{gregoriotex-signs.tex}
Macro to help typesetting vertical episema.

\begin{argtable}
	\#1 & Gregorio\TeX\ glyph & is an offset glyph (see \#3 below)\\
	\#2 & Gregorio\TeX glyph & the glyph upon which the sign is to be centered\\
	\#3 & \texttt{0} & go back to the beginning of the previous glyph and then forward half the width of \#2; this puts the sign at the beginning of the previous glyph, whose first note is the size of \#2\\
	& \texttt{1} & go back half the width of \#2; this puts the sign at the end of the previous glyph, whose last note is the size of \#2\\
	& \texttt{2} & go back the width of \#1 and then forward half the width of \#2; this puts the sign at the glyph from the end that starts at \#1’s width from the end\\
	& \texttt{3} & go back to the beginning of the previous glyph and then forward the width of \#1 and then back half the width of \#2; this puts the sign at the glyph from the start that ends at \#1’s width from the start\\
	\#4 & dimension & a shift that we want to get applied, useful for punctum inclinatum for example\\
	\#5 & integer & is the glyph number\\
	\#6 & \texttt{1} & vertical episema\\
	& \texttt{2} & rare sign\\
	& \texttt{3} & choral sign\\
	& \texttt{4} & brace above the bar\\
	\#7 & string & the choral sign if relevant\\
\end{argtable}

\macroname{\textbackslash gre@newglyphcommon}{}{gregoriotex-syllable.tex}
Macro called before each glyph.

\macroname{\textbackslash gre@normalinitial}{}{gregoriotex-main.tex}
Macro called at the end of the score to ensure that a big initial setting doesn’t carry into the next score.

\macroname{\textbackslash greoldcatcode}{}{gregoriotex.tex}
Macro to store the catcode for ``@'' so that we can use said symbol in function names under Plain \TeX\ and then restore the original catcode after the package is done loading.

\macroname{\textbackslash gre@prephepisemaledgerlineheuristics}{}{gregoriotex-spaces.tex}
Prepares the system to accept ledger line heuristics for the horizontal episema.

\macroname{\textbackslash gre@reseteolcustos}{}{gregoriotex-main.tex}
Alias that resets the use of automatic custos to the value set by \verb=\greseteolcustos=.  This macro is aliased to \verb=\gre@useautoeolcustos= or \verb=\gre@usemanualeolcustos= by \verb=\greseteolcustos=.

\macroname{\textbackslash gre@resetledgerlineheuristics}{}{gregoriotex-spaces.tex}
Resets the ledger line heuristic flags.

\macroname{\textbackslash gre@setstylefont}{}{gregoriotex-main.tex}
Macro for opening up greextra font.

\macroname{\textbackslash gre@syllablenotes}{\#1}{gregoriotex-syllable.tex}
Macro for populating \verb=\gre@box@syllablenotes=.

\begin{argtable}
	\#1 & string & The contents to be placed in the box\\
\end{argtable}

\macroname{\textbackslash gre@symbolfontsize}{}{gregoriotex-symbols.tex}
The font size at which symbols are to be loaded.

\macroname{\textbackslash gre@textnormal}{\#1}{gregoriotex-syllable.tex}
Macro which applies the default text format.

\macroname{\textbackslash gre@save@clef}{\#1\#2\#3\#4\#5\#6}{gregoriotex-signs.tex}
Saves clef information for use in \verb=gre@updatelinesclef=.

\begin{argtable}
	\#1 & character & the type of the clef: c or f\\
	\#2 & integer & the line of the clef (1 is the lowest)\\
	\#3 & integer & if \texttt{3}, it means that we must not put a flat after the clef, otherwise it’s the height of the flat\\
	\#4 & character & the type of the secondary clef: c or f\\
	\#5 & integer & the line of the secondary clef (1 is the lowest, 0 for no secondary clef)\\
	\#6 & integer & if \texttt{3}, it means that we must not put a flat after the secondary clef, otherwise it’s the height of the flat\\
\end{argtable}

\macroname{\textbackslash gre@typeclef}{\#1\#2\#3\#4\#5\#6\#7\#8}{gregoriotex-signs.tex}
Macro which typesets the clef.

\begin{argtable}
	\#1 & character & the type of the clef: c or f\\
	\#2 & integer & the line of the clef (1 is the lowest)\\
	\#3 & \texttt{0} & no need to use small clef characters (inside a line)\\
	& \texttt{1} & we must use small clef characters (inside a line)\\
	\#4 & \texttt{0} & no extra space is needed after the clef\\
	& \texttt{}1 & we must type a space after the clef\\
	\#5 & integer & if \texttt{3}, it means that we must not put a flat after the clef, otherwise it’s the height of the flat\\
	\#6 & character & the type of the secondary clef: c or f\\
	\#7 & integer & the line of the secondary clef (1 is the lowest, 0 for no secondary clef)\\
	\#8 & integer & if \texttt{3}, it means that we must not put a flat after the secondary clef, otherwise it’s the height of the flat\\
\end{argtable}

\macroname{\textbackslash gre@typesingleclef}{\#1\#2\#3\#4}{gregoriotex-signs.tex}
Macro which typesets a single clef.

\begin{argtable}
	\#1 & character & the type of the clef: c or f\\
	\#2 & integer & the line of the clef (1 is the lowest)\\
	\#3 & \texttt{0} & no need to use small clef characters (inside a line)\\
	& \texttt{1} & we must use small clef characters (inside a line)\\
	\#4 & integer & if \texttt{3}, it means that we must not put a flat after the clef, otherwise it’s the height of the flat\\
\end{argtable}

\macroname{\textbackslash gre@updateleftbox}{}{gregoriotex-main.tex}
Macro to update the box printed a the left end of every line (the one which holds the staff lines).

\macroname{\textbackslash gre@useautoeolcustos}{}{gregoriotex-main.tex}
Macro which enables automatic custos at the end of lines.

\macroname{\textbackslash gre@usemanualeolcustos}{}{gregoriotex-main.tex}
Macro which disables automatic custos at the end of lines.

\macroname{\textbackslash gre@usestylecommon}{}{gregoriotex-signs.tex}
Macro which specifies the alternate glyphs which are common to all of the styles.

\macroname{\textbackslash gre@widthof}{\#1}{gregoriotex-main.tex}
Macro for calculating the width of its argument and storing it in \verb=\gre@dimen@temp@three=.

\macroname{\textbackslash gre@writebar}{\#1\#2\#3\#4}{gregoriotex-signs.tex}
Macro to write a bar.

\begin{argtable}
	\#1 & \texttt{0} & virgula\\
	& \texttt{1} & minima\\
	& \texttt{2} & minor\\
	& \texttt{3} & major\\
	& \texttt{4} & finalis\\
	& \texttt{5} & the last finalis\\
	\#2 & \texttt{0} & in a syllable containing only this bar\\
	& \texttt{1} & in a syllable containing other notes\\
	\#3 & \texttt{0} & if there is no text underneath the bar\\
	& \texttt{1} & if there is text underneath the bar\\
	\#4 & \TeX\ code & macros that may happen before the skip after the bar (typically GreVEpisema)\\
\end{argtable}

\macroname{\textbackslash gre@@arg}{}{gregoriotex-syllable.tex}
A dummy macro which is used to store a macro which takes an argument so that it can be used in a \verb=\ifx= comparison.  The value of the argument is provided at the time this macro is created.

\macroname{\textbackslash gre@nothing}{}{gregoriotex-main.tex}
A dummy macro which has not contents.  Used for \verb=\ifx= comparisons.

\macroname{\textbackslash gre@annotation}{[\#1]\#2}{gregoriotex-main.tex}
Workhorse function behind \verb=\greannotation=.

\begin{argtable}
	\#1 & \texttt{c} & center align the new line with the existing annotation content\\
	& \texttt{l} & left align the new line with the existing annotation content\\
	& \texttt{r} & right align the new line with the existing annotation content\\
	\#2 & string & the new annotation content\\
\end{argtable}

\macroname{\textbackslash gre@commentary}{[\#1]\#2}{gregoriotex-main.tex}
Workhorse function behind \verb=\grecommentary=.

\begin{argtable}
	\#1 & distance & Additional distance to be placed between the commentary and the top staff line for the next score only.\\
	\#2 & string & the new commentary content\\
\end{argtable}

\macroname{\textbackslash gre@atletter}{}{gregoriotex-main.tex}
A Lua\TeX\ catcode table which makes sure that Lua\TeX\ treats `@‘ corectly.

\macroname{\textbackslash gre@baseunit}{}{gregoriotex-spaces.tex}
The units attached to base dimension in a string distance.  These units are extracted as part of coercing one distance to have the same units as another\\ (\verb=\gre@consistentunits=).

\macroname{\textbackslash gre@stretchunit}{}{gregoriotex-spaces.tex}
The units attached to stretch dimension in a string distance.  These units are extracted as part of coercing one distance to have the same units as another (\verb=\gre@consistentunits=).

\macroname{\textbackslash gre@shrinkunit}{}{gregoriotex-spaces.tex}
The units attached to shrink dimension in a string distance.  These units are extracted as part of coercing one distance to have the same units as another (\verb=\gre@consistentunits=).

\macroname{\textbackslash gre@bug}{\#1}{gregoriotex.sty \textup{and} gregoriotex.tex}
Macro for raising a bug error when some calculation goes awry and comes up with a non-sensical result.  Generally will be found in Lua code, not \TeX\ code.

\macroname{\textbackslash gre@changestyle}{\#1\#2[\#3]}{gregoriotex.sty \textup{and} gregoriotex.tex}
Workhorse function behind \verb=\grechangestyle=.  Necessary because the internals of the definition are slightly different in \LaTeX\ and Plain \TeX.

\macroname{\textbackslash gre@calculate@bolshift}{\#1\#2}{gregoriotex-spaces.tex}
Macro used in \verb=\GreSyllable=. Sets \verb=\gre@skip@bolshift= to the left kern that should appear at the beginning of a line in case of a forced linebreak.  The goal of this left kern is to have all lines aligned on notes.  This shift is applied to the right before every syllable and then to the left after placing an empty box.  At the beginning of the line the shift to the right is ignored by \TeX\ as leading white space, but the shift left is not because of the presence of the “character” of the empty box.

\begin{argtable}
	\#1 & dimension & \verb=begindifference= of the syllable\\
\end{argtable}

\macroname{\textbackslash gre@calculate@eolshift}{\#1}{gregoriotex-spaces.tex}
Macro used in \verb=\GreSyllable=. Sets \verb=\gre@dimen@eolshift= to the left kern that
should appear before an end of line. The improvement is tiny: when
text go further than notes in the last syllable of a line, the idea
is to allow text to go a bit further right, under the custos.  This shift is applied to the left after every syllable and then to the right after setting the line break penalty.  If the line break occurs after this syllable, it will occur between the two shifts, pushing the shift right to the beginning of the next line where \TeX\ will ignore it as leading white space.

\begin{argtable}
	\#1 & dimension & The \verb=enddifference= of the corresponding syllable\\
\end{argtable}

\macroname{\textbackslash gre@calculate@syllablefinalskip}{\#1\#2}{gregoriotex-spaces.tex}
Macro computing the skip at the end of the syllable.

\begin{argtable}
	\#1 & \texttt{0} & if end of syllable\\
	& \texttt{1} & if end of word\\
	\#2 & \texttt{0} & if next syllable is normal\\
	& \texttt{1} & if it’s a bar\\
\end{argtable}

\macroname{\textbackslash gre@convert}{}{gregoriotex-spaces.tex}
Macro to hold the original distance which is to be scaled by \verb=\gre@changeonedimenfactor=

\macroname{\textbackslash gre@debug}{}{gregoriotex.sty \textup{and} gregoriotex.tex}
Macro to hold the list of debug messages which should be designated as printing.

\macroname{\textbackslash gre@debugmsg}{\#1\#2}{gregoriotex-main.tex}
Macro to print debugging messages.

\begin{argtable}
	\#1 & string & The category of the message (used in conjunction with \verb=\gre@debug= to determine whether to print the message or not\\
	\#2 & string & The debug message\\
\end{argtable}

\macroname{\textbackslash gre@declarefileversion}{\#1\#2}{gregoriotex-main.tex}
Macro which checks for version consistency between Gregorio\TeX\ files.

\begin{argtable}
	\#1 & string & name of the current file\\
	\#2 & string & version of the current file\\
\end{argtable}

\macroname{\textbackslash gre@def@char@he}{\#1\#2}{gregoriotex-chars.tex}
Macro for defining the various types of horizontal episema.

\begin{argtable}
	\#1 & string & name of the horizontal episema to be defined\\
	\#2 & string & Camel case name of horizontal episema to be defined\\
\end{argtable}

\macroname{\textbackslash gre@def@char@he@porr}{\#1\#2}{gregoriotex-chars.tex}
Macro for defining the various types of horizontal episema porrectus.

\begin{argtable}
	\#1 & string & name of the horizontal episema porrectus to be defined\\
	\#2 & string & Camel case name of horizontal episema porrectus to be defined\\
\end{argtable}

\macroname{\textbackslash gre@char@cavum}{\#1\#2\#3\#4\#5\#6\#7\#8}{gregoriotex-signs.tex}
Macro to typeset a “cavum” character.

\begin{argtable}
	\#1 & length  & Argument \#2 from \verb=\GreGlyph=. Height to raise the glyph.\\
	\#2 & length  & Argument \#3 from \verb=\GreGlyph=. Height of the next note.\\
	\#3 & integer & Argument \#4 from \verb=\GreGlyph=. The type of glyph.\\
	\#4 & \TeX\ code    & Macros executed before the punctum cavum is written.\\
	\#5 & character & Argument \#5 from \verb=\GreGlyph=. The signs to typeset before the glyph.\\
	\#6 & string & the line, byte offset, and column address for textedit links when point-and-click is enabled.\\
	\#7 & control sequence & The control sequence for the glyph.\\
	\#8 & control sequence & The control sequence for the hole glyph.
\end{argtable}

\macroname{\textbackslash gre@get@spaceskip}{\#1}{gregoriotex-signs.tex}
Loads \verb=\gre@skip@temp@four= with the appropriate rubber length given the
desired case.

\begin{argtable}
	\#1 & \texttt{0} & Default space.\\
	& \texttt{1} & Zero-width space.\\
	& \texttt{2} & Space between flat or natural and a note.\\
	& \texttt{3} & Space between two puncta inclinata, descending.\\
	& \texttt{4} & Space between bivirga or trivirga.\\
	& \texttt{5} & space between bistropha or tristropha.\\
	& \texttt{6} & Space after a punctum mora XXX: not used yet, not so sure it is a good idea\ldots\\
	& \texttt{7} & Space between a punctum inclinatum and a punctum inclinatum debilis, descending.\\
	& \texttt{8} & Space between two puncta inclinata debilis.\\
	& \texttt{9} & Space before a punctum (or something else) and a punctum inclinatum.\\
	& \texttt{10} & Space between puncta inclinata (also debilis for now), larger ambitus (range=3rd), descending.\\
	& \texttt{11} & Space between puncta inclinata (also debilis for now), larger ambitus (range=4th or 5th), descending.\\
	& \texttt{12} & Space between two puncta inclinata, ascending. \\
	& \texttt{13} & Space between a punctum inclinatum and a punctum inclinatum debilis, ascending. \\
	& \texttt{14} & Space between puncta inclinata (also debilis for now), larger ambitus (range=3rd), ascending. \\
	& \texttt{15} & Space between puncta inclinata (also debilis for now), larger ambitus (range=4th or 5th), ascending. \\
	& \texttt{16} & Space between a punctum inclinatum and a ``no-bar'' glyph one pitch below. \\
	& \texttt{17} & Space between a punctum inclinatum and a ``no-bar'' glyph two pitches below. \\
	& \texttt{18} & Space between a punctum inclinatum and a ``no-bar'' glyph three or four pitches below \\
	& \texttt{19} & Space between a punctum inclinatum and a ``no-bar'' glyph one pitch above. \\
	& \texttt{20} & Space between a punctum inclinatum and a ``no-bar'' glyph two pitches above. \\
	& \texttt{21} & Space between a punctum inclinatum and a ``no-bar'' glyph three or four pitches above \\
	& \texttt{22} & Half-space. \\
\end{argtable}

\macroname{\textbackslash gre@nabcfontname}{}{gregoriotex-main.tex}
Macro which stores the name of the currently selected font for \texttt{nabc}.

\macroname{\textbackslash gre@nabcfontsize}{}{gregoriotex-main.tex}
Macro which stores the size of the currently selected font for \texttt{nabc}.

\macroname{\textbackslash gre@endsyllablepart}{}{gregoriotex-syllable.tex}
Macro which stores the end part of the current syllable (that which comes after the alignment part).

\macroname{\textbackslash gre@firstsyllablepart}{}{gregoriotex-syllable.tex}
Macro which stores the first part of the current syllable (that which comes before the alignment part).

\macroname{\textbackslash gre@middlesyllablepart}{}{gregoriotex-syllable.tex}
Macro which stores the middle part of the current syllable (the part which aligns with the notes).

\macroname{\textbackslash gre@nextendsyllablepart}{}{gregoriotex-syllable.tex}
Macro which stores the end part of the next syllable (that which comes after the alignment part).

\macroname{\textbackslash gre@nextfirstsyllablepart}{}{gregoriotex-syllable.tex}
Macro which stores the first part of the next syllable (that which comes before the alignment part).

\macroname{\textbackslash gre@nextmiddlesyllablepart}{}{gregoriotex-syllable.tex}
Macro which stores the middle part of the next syllable (the part which aligns with the notes).

\macroname{\textbackslash gre@fixedtextformat}{\#1}{gregoriotex-syllable.tex}
A macro which applies formatting that needs to apply to the whole syllable (rather than the parts individually) for the current syllable.  Necessary to preserve ligatures across parts within a syllable.

\begin{argtable}
	\#1 & string & The syllable (usually built as \verb=\gre@firstsyllablepart\gre@middlesyllablepart\gre@endsyllablepart=\\
\end{argtable}

\macroname{\textbackslash gre@fixednexttextformat}{\#1}{gregoriotex-syllable.tex}
A macro which applies formatting that needs to apply to the whole syllable (rather than the parts individually) for the next syllable.  Necessary to preserve ligatures across parts within a syllable.

\begin{argtable}
	\#1 & string & The syllable (usually built as \verb=\gre@nextfirstsyllablepart\gre@nextmiddlesyllablepart\gre@nextendsyllablepart=\\
\end{argtable}

\macroname{\textbackslash gre@gabcname}{}{gregoriotex-main.tex}
Macro which holds the point-and-click file name.

\macroname{\textbackslash gre@gregoriotexluaversion}{}{gregoriotex-main.tex}
Macro to hold the version number of \emph{gregoriotex.lua} so that it can be checked for consistency.

\macroname{\textbackslash gre@gregorioversion}{}{gregoriotex-main.tex}
Macro to hold the version number of Gregorio\TeX\ so that it can be checked for consistency.

\macroname{\textbackslash gre@leftfill}{}{gregoriotex-main.tex}
Macro set to \verb=\hfil= or \verb=\relax= depending on alignment choices.

\macroname{\textbackslash gre@lyriccentering}{}{gregoriotex-syllable.tex}
Macro set to \verb=0= for full-syllable centering, \verb=1= for vowel centering (the default), or \verb=2= for first-letter centering.

\macroname{\textbackslash gre@rightfill}{}{gregoriotex-main.tex}
Macro set to \verb=\hfil= or \verb=\relax= depending on alignment choices.

\macroname{\textbackslash gre@mark@abovelinestext}{}{gregoriotex-main.tex}
Macro to set the point-and-click position for above lines text.

\macroname{\textbackslash gre@mark@translation}{}{gregoriotex-main.tex}
Macro to set the point-and-click position for translations.

\macroname{\textbackslash gre@pitch@[a-n,p]}{}{gregoriotex-main.tex}
Macros which map gabc pitch letters (the final part of the macro name) to the numerical value that Gregorio\TeX\ uses in processing note placement.

\macroname{\textbackslash gre@pitch@adjust@top}{}{gregoriotex-main.tex}
If any note appears above this pitch, the space above the lines must be adjusted to account for it.

\macroname{\textbackslash gre@pitch@adjust@bottom}{}{gregoriotex-main.tex}
If any note appears below this pitch, the space below the lines must be adjusted to account for it.

\macroname{\textbackslash gre@pitch@abovestaff}{}{gregoriotex-main.tex}
The pitch above the staff.

\macroname{\textbackslash gre@pitch@belowstaff}{}{gregoriotex-main.tex}
The pitch below the staff.

\macroname{\textbackslash gre@pitch@ledger@above}{}{gregoriotex-main.tex}
The pitch of the ledger line above the staff.

\macroname{\textbackslash gre@pitch@ledger@below}{}{gregoriotex-main.tex}
The pitch of the ledger line below the staff.

\macroname{\textbackslash gre@pitch@barvepisema}{}{gregoriotex-main.tex}
The pitch of the bar episema.

\macroname{\textbackslash gre@pitch@underbrace}{}{gregoriotex-main.tex}
The pitch of the under-the-staff brace.

\macroname{\textbackslash gre@pitch@overbrace}{}{gregoriotex-main.tex}
The pitch of the over-the-staff brace.

\macroname{\textbackslash gre@pitch@overbraceglyph}{}{gregoriotex-main.tex}
The pitch of the over-the-staff brace glyph.

\macroname{\textbackslash gre@pitch@bar}{}{gregoriotex-main.tex}
The pitch of the bar glyph.

\macroname{\textbackslash gre@pitch@raresign}{}{gregoriotex-main.tex}
The pitch of a rare sign (semicirculus, \etc).

\macroname{\textbackslash gre@pitch@dummy}{}{gregoriotex-main.tex}
A meaningless (don't-care) pitch.

\macroname{\textbackslash gre@pointandclick}{\#1\#2}{gregoriotex-main.tex}
Macro to generate the point-and-click links.

\begin{argtable}
	\#1 & \TeX\ code & the entity which is to contain the link\\
	\#2 & link target & line:char:column for the link\\
\end{argtable}

\macroname{\textbackslash gre@prefix}{}{gregoriotex-spaces.tex}
Either ``skip’’ or ``dimen’’ according to the distance being set or changed at the given moment.

\macroname{\textbackslash gre@rubberpermit}{\#1}{gregoriotex-spaces.tex}
Determines whether the given distance is allowed to take a rubber length.

\begin{argtable}
	\#1 & string & the name of the distance to check\\
\end{argtable}

\macroname{\textbackslash gre@setgregoriofont}{[\#1]\#2}{gregoriotex-main.tex}
Workhorse function behind \verb=\gresetgregoriofont=.

\begin{argtable}
	\#1 & \textit{(omitted)} & Use the normal font and rule set (default).\\
	& \texttt{op} & Use the alternate Dominican font/rule set.\\
	\#2 & \texttt{greciliae} & Use the Greciliae font (default).\\
	& \texttt{gregorio} & Use the Gregorio font.\\
	& \texttt{parmesan} & Use the Parmesan font.\\
\end{argtable}

\macroname{\textbackslash gre@syllable@end}{\#1\#2\#3}{gregoriotex-syllable.tex}
Macro to make a few checks and call the right macros between \verb=\endbeforebar, \endofword, \endofsyllable=.

\begin{argtable}
	\#1 & & next syllable type (\#7 of \verb=\GreSyllable=)\\
	\#2 & string & next syllable text\\
	\#3 & \texttt{0} & this syllable is not the end of a word\\
	& \texttt{1} & this syllable is the end of a word\\
\end{argtable}

\macroname{\textbackslash gre@typeout}{\#1}{gregoriotex.sty \textup{and} gregoriotex.tex}
Macro which points to \verb=\typeout= in \LaTeX\ or \verb=\message= in Plain \TeX.

\macroname{\textbackslash gre@unsetfixedtextformat}{}{gregoriotex-syllable.tex}
Macro which changes \verb=\gre@fixedtextformat= back to normal text.

\macroname{\textbackslash gre@unsetfixednexttextformat}{}{gregoriotex-syllable.tex}
Macro which changes \verb=\gre@fixednexttextformat= back to normal text.

\macroname{\textbackslash gregoriotex@symbols@loaded}{}{gregoriotex-symbols.tex}
Empty macro which is used to determine if the symbols have been loaded and prevent loading them again if they have.

\macroname{\textbackslash gre@hskip}{}{gregoriotex-signs.tex}
Alias for \verb=\hskip= or \verb=\kern=.  We use this rather than those functions directly so that the same element can appear in discretionaries where \verb=\kern= is allowed but \verb=\hskip= is not by simply changing the assignment of this macro when we enter one.

\macroname{\textbackslash gre@localleftbox}{}{gregoriotex-main.tex}
Alias for \verb=\luatexlocalleftbox= or \verb=\localleftbox=, depending on \LaTeX\ version.

\macroname{\textbackslash gre@localrightbox}{}{gregoriotex-main.tex}
Alias for \verb=\luatexlocalrightbox= or \verb=\localrightbox=, depending on \LaTeX\ version.

\macroname{\textbackslash gre@resizebox}{}{gregoriotex-main.tex}
Alias for \verb=\resizebox=.

\macroname{\textbackslash gre@dimension}{}{gregoriotex-spaces.tex}
Workhorse function behind \verb=\grecreatedim= and \verb=\grechangedim=.

\macroname{\textbackslash gre@setstafflines}{\#1}{gregoriotex-main.tex}
Sets the number of staff lines.

\begin{argtable}
	\#1 & integer & The number of staff lines\\
\end{argtable}

\macroname{\textbackslash gre@stafflines}{}{gregoriotex-main.tex}
Contains the number of staff lines.

\macroname{\textbackslash gre@romannumeral@majuscule}{\#1}{gregoriotex-main.tex}
Typesets its numeric argument as an upper-case Roman numeral.

\begin{argtable}
	\#1 & integer & The number to typeset\\
\end{argtable}

\macroname{\textbackslash gre@romannumeral@minuscule}{\#1}{gregoriotex-main.tex}
Typesets its numeric argument as a lower-case Roman numeral.

\begin{argtable}
	\#1 & integer & The number to typeset\\
\end{argtable}

\macroname{\textbackslash gre@bar@text}{\#1}{gregoriotex-symbols.tex}
Macro used to switch between spacings where bar has text and those where it doesn't.

\begin{argtable}
	\#1 & \texttt{0} & emits nothing\\
	& \texttt{1} emits \texttt{text}\\
\end{argtable}

\macroname{\textbackslash gre@drawadditionalline}{\#1\#2\#3\#4\#5\#6}{gregoriotex-signs.tex}
Workhorse function behind \verb=\GreDrawAdditionalLine=.

\begin{argtable}
	\#1 & \texttt{0} & Draw an over-the-staff ledger line. \\
			& \texttt{1} & Draw an under-the-staff ledger line. \\
	\#2 & distance   & The length of the line, with TeX units, excluding any left or right distances coming from the rest of the arguments. \\
	\#3 & \texttt{0} & Start the line at this point. \\
			& \texttt{1} & Start the line to the left of this point by \verb=gre@dimen@additionallineswidth=. \\
			& \texttt{2} & Start the line to the left of this point by \#4. \\
	\#4 & distance   & The distance to move left before starting the line if \#3 is \texttt{2}. \\
	\#5 & \texttt{0} & End the line exactly \#2 to the right of this point. \\
			& \texttt{1} & End the line \verb=gre@dimen@additionallineswidth= to the right of \#2 from this point. \\
			& \texttt{2} & End the line \#6 to the right of \#2 from this point. \\
	\#6 & distance   & The distance to end the line after \#2 from this point if \#3 is \texttt{2}. \\
\end{argtable}

\macroname{\textbackslash gre@kern@bar@aftermora}{}{gregoriotex-signs.tex}
Macro which kerns between a punctum mora and a kern according to the setting in \verb=\gre@count@barshiftaftermora=.

\macroname{\textbackslash gre@setgregoriofontscaled}{[\#1]\#2\#3}{gregoriotex-main.tex}
Workhorse behind \verb=\gresetgregoriofontscaled=.

\begin{argtable}
	\#1 & \textit{(omitted)} & Use the normal font and rule set.\\
	& \texttt{op} & Use the alternate Dominican font/rule set.\\
	\#2 & \texttt{greciliae} & Use the Greciliae font (default).\\
	& \texttt{gregorio} & Use the Gregorio font.\\
	& \texttt{parmesan} & Use the Parmesan font.\\
	\#3 & integer & the scaling factor\\
\end{argtable}


\subsection{Auxiliary File}
Gregorio\TeX\ creates its own auxiliary file (extension \texttt{gaux}) which it uses to store information between successive typesetting runs.  This allows for such features as the dynamic interline spacing.  The following functions are used to interact with that auxiliary file.

\macroname{\textbackslash gre@gaux}{}{gregoriotex-main.tex}
The handle for the auxiliary file.

\macroname{\textbackslash gre@open@gaux}{}{gregoriotex-main.tex}
Macro for opening the auxiliary file.

\macroname{\textbackslash gre@close@gaux}{}{gregoriotex-main.tex}
Macro for closing the auxiliary file.

\macroname{\textbackslash gre@write@gaux}{\#1}{gregoriotex-main.tex}
Macro for writing the auxiliary file.

\begin{argtable}
	\#1 & string & contents to be written to the auxiliary file\\
\end{argtable}


\subsection{Fonts}
Gregorio\TeX\ loads a number of fonts which are referred to by the following macros.

\macroname{\textbackslash gre@font@music}{}{gregoriotex-main.tex}
The font for the neumes and other principle score elements.

\macroname{\textbackslash gre@font@style}{}{gregoriotex-main.tex}
The font for some of the extra characters, such as the bar for barred letters.

\macroname{\textbackslash gre@font@nabc}{}{gregoriotex-nabc.tex}
The font for ancient notation.

\macroname{\textbackslash gre@font@initial}{}{gregoriotex.tex}
The font for the default initial format in Plain \TeX.

\macroname{\textbackslash gre@fontfactor@...}{}{gregoriotex-main.tex}
Macro holding the factor at which the font is loaded.  There is one of these macros for each chant font which has been loaded with the ending of the macro name being the name of the chant font (gregorio, parmesan, \etc).


\subsection{Character Reference Aliases}
To make referencing and changing them easier, Gregorio\TeX\ stores reference information for certain characters using the following macros.

\macroname{\textbackslash gre@fontchar@abovebarbrace}{}{gregoriotex-chars.tex}
The above bar brace.

\macroname{\textbackslash gre@fontchar@flat}{}{gregoriotex-chars.tex}
The flat character.

\macroname{\textbackslash gre@fontchar@flathole}{}{gregoriotex-chars.tex}
The flat hole character (\ie, the character needed to prevent lines from showing through the center of the flat).

\macroname{\textbackslash gre@fontchar@natural}{}{gregoriotex-chars.tex}
The natural character.

\macroname{\textbackslash gre@fontchar@naturalhole}{}{gregoriotex-chars.tex}
The natural hole character (\ie, the character needed to prevent lines from showing through the center of the natural).

\macroname{\textbackslash gre@fontchar@sharp}{}{gregoriotex-chars.tex}
The sharp character.

\macroname{\textbackslash gre@fontchar@sharphole}{}{gregoriotex-chars.tex}
The sharp hole character (\ie, the character needed to prevent lines from showing through the center of the sharp).

\macroname{\textbackslash gre@fontchar@punctumcavum}{}{gregoriotex-signs.tex}
The punctum cavum character.

\macroname{\textbackslash gre@fontchar@punctumcavumhole}{}{gregoriotex-signs.tex}
The punctum cavum hole character (\ie, the character needed to prevent lines from showing through the center of the punctum cavum).

\macroname{\textbackslash gre@fontchar@lineapunctumcavum}{}{gregoriotex-signs.tex}
The linea punctum cavum character.

\macroname{\textbackslash gre@fontchar@lineapunctumcavumhole}{}{gregoriotex-signs.tex}
The linea punctum cavum hole character (\ie, the character needed to prevent lines from showing through the center of the linea punctum cavum).

\macroname{\textbackslash gre@fontchar@incclef}{}{gregoriotex-chars.tex}
The \texttt{c}-clef which appears in the middle of a line.

\macroname{\textbackslash gre@fontchar@infclef}{}{gregoriotex-chars.tex}
The \texttt{f}-clef which appears in the middle of a line.

\macroname{\textbackslash gre@fontchar@cclef}{}{gregoriotex-chars.tex}
The \texttt{c}-clef which appears at the beginning of a line.

\macroname{\textbackslash gre@fontchar@fclef}{}{gregoriotex-chars.tex}
The \texttt{f}-clef which appears at the beginning of a line.

\macroname{\textbackslash gre@fontchar@punctum}{}{gregoriotex-chars.tex}
The punctum character.

\macroname{\textbackslash gre@fontchar@punctummora}{}{gregoriotex-chars.tex}
The punctum mora character.

\macroname{\textbackslash gre@fontchar@underbrace}{}{gregoriotex-signs.tex}
The under brace character.

\macroname{\textbackslash gre@fontchar@verticalepisema}{}{gregoriotex-chars.tex}
The vertical episema character.

\macroname{\textbackslash gre@fontchar@brace}{}{gregoriotex-signs.tex}
The (rounded) brace character.

\macroname{\textbackslash gre@fontchar@curlybrace}{}{gregoriotex-signs.tex}
The curly brace character.

\macroname{\textbackslash gre@fontchar@custosbottomlong}{}{gregoriotex-chars.tex}
The custos character with a long upwards directed vigra.

\macroname{\textbackslash gre@fontchar@custosbottommiddle}{}{gregoriotex-chars.tex}
The custos character with a middle upwards directed vigra.

\macroname{\textbackslash gre@fontchar@custosbottomshort}{}{gregoriotex-chars.tex}
The custos character with a short upwards directed vigra.

\macroname{\textbackslash gre@fontchar@custostoplong}{}{gregoriotex-chars.tex}
The custos character with a long downwards directed vigra.

\macroname{\textbackslash gre@fontchar@custostopmiddle}{}{gregoriotex-chars.tex}
The custos character with a middle downwards directed vigra.

\macroname{\textbackslash gre@fontchar@custostopshort}{}{gregoriotex-chars.tex}
The custos character with a short downwards directed vigra.

\macroname{\textbackslash gre@fontchar@divisiofinalis}{}{gregoriotex-signs.tex}
The divisio finalis.

\macroname{\textbackslash gre@fontchar@divisiomaior}{}{gregoriotex-signs.tex}
The divisio maior.

\macroname{\textbackslash gre@char@normalhyphen}{}{gregoriotex-main.tex}
A normal hyphen in the text font.

\macroname{\textbackslash gre@char@fuse@debilis}{}{gregoriotex-chars.tex}
A fused character consisting of a leading punctum initio debilis (of ambitus one) and a regular punctum.

\macroname{\textbackslash gre@char@fuse@oriscus@one}{}{gregoriotex-chars.tex}
A fused character consisting of a leading oriscus (of ambitus one) and a regular punctum.

\macroname{\textbackslash gre@char@fuse@oriscus@two}{}{gregoriotex-chars.tex}
A fused character consisting of a leading oriscus (of ambitus two) and a regular punctum.

\macroname{\textbackslash gre@char@fuse@punctum@one}{}{gregoriotex-chars.tex}
A fused character consisting of a leading punctum (of ambitus one) and a regular punctum.

\macroname{\textbackslash gre@char@fuse@punctum@two}{}{gregoriotex-chars.tex}
A fused character consisting of a leading punctum (of ambitus two) and a regular punctum.

\macroname{\textbackslash gre@char@fuse@quilisma@one}{}{gregoriotex-chars.tex}
A fused character consisting of a leading quilisma (of ambitus one) and a regular punctum.

\macroname{\textbackslash gre@char@fuse@quilisma@two}{}{gregoriotex-chars.tex}
A fused character consisting of a leading quilisma (of ambitus two) and a regular punctum.

\macroname{\textbackslash gre@char@he@...}{\#1}{gregoriotex-chars.tex}
A class of macros for the horizontal episema which populates the \verb=\gre@box@hep= box.



\subsection{Flags}

Flags are either boolean (defined with \verb=\newif=), Lua\TeX\ attributes, or counts (defined with \verb=\newcount=).  They store settings and/or the current state of something so that Gregorio\TeX\ can typeset things in the desired manner.

All distances in \nameref{distances} and \texttt{stafflinefactor} have a boolean associated with them, of the form \verb=\ifgre@scale@*=.  This boolean
indicates if the distance should scale when the staff size changes (true)
or not (false).

\macroname{\textbackslash ifgre@annotationbottomline}{}{gregoriotex-main.tex}
Boolean used to indicate if the bottom line of the annotation should be used as the control line for its initial vertical alignment.

\macroname{\textbackslash gre@count@annotationvalign}{}{gregoriotex-main.tex}
Count used to indicate which part of the annotation control line should be initially aligned with the top line of the staff.  Values: \texttt{0}, top; \texttt{1}, baseline; or \texttt{2}, bottom.

\macroname{\textbackslash ifgre@forcehyphen}{}{gregoriotex-main.tex}
Boolean used to indicate if hyphens should be forced between all syllables in a polysyllabic word.

\macroname{\textbackslash ifgre@checklength}{}{gregoriotex-spaces.tex}
Boolean used in \verb=\gresetdim= to indicate if we are attempting to set a rubber length.

\macroname{\textbackslash ifgre@rubber}{}{gregoriotex-spaces.tex}
Boolean used in \verb=\gre@changeonedimenfactor= to indicate if we are dealing with one of the distances which can accept a rubber length.

\macroname{\textbackslash ifgre@stretch}{}{gregoriotex-spaces.tex}
Boolean used in \verb=\gre@changeonedimenfactor= as we test for the presence of a stretch.

\macroname{\textbackslash ifgre@shrink}{}{gregoriotex-spaces.tex}
Boolean used in \verb=\gre@changeonedimenfactor= as we test for the presence of a shrink.

\macroname{\textbackslash ifgre@translationcentering}{}{gregoriotex-main.tex}
Boolean used to specify whether the translation text should be centered below its respective syllable.

\macroname{\textbackslash ifgre@showlines}{}{gregoriotex-main.tex}
Boolean used to specify whether the staff lines should be shown or not.

\macroname{\textbackslash ifgre@hidepclines}{}{gregoriotex-signs.tex}
Boolean used to specify whether the staff lines behind a punctum cavum should be hidden.

\macroname{\textbackslash ifgre@hidealtlines}{}{gregoriotex-signs.tex}
Boolean used to specify whether the staff lines behind an alteration should be hidden.

\macroname{\textbackslash ifgre@hepisemabridge}{}{gregoriotex-signs.tex}
Boolean used to specify whether adjacent horizontal episemata should be joined together.

\macroname{\textbackslash ifgre@metapost@brace}{}{gregoriotex-signs.tex}
Boolean used to specify whether round over-the-staff braces should be drawn by
\MP{} as opposed to rendered via the score font.

\macroname{\textbackslash ifgre@metapost@underbrace}{}{gregoriotex-signs.tex}
Boolean used to specify whether round under-the-staff braces should be drawn by
\MP{} as opposed to rendered via the score font.

\macroname{\textbackslash ifgre@metapost@curlybrace}{}{gregoriotex-signs.tex}
Boolean used to specify whether curly braces should be drawn by \MP{} as
opposed to rendered via the score font.

\macroname{\textbackslash ifgre@metapost@barbrace}{}{gregoriotex-signs.tex}
Boolean used to specify whether divisio braces should be drawn by \MP{} as
opposed to rendered via the score font.

\macroname{\textbackslash gre@biginitial}{}{gregoriotex-main.tex}
Count to track whether the initial is big (2-lines) or normal (1-line).

\macroname{\textbackslash ifgre@boxing}{}{gregoriotex-syllable.tex}
Boolean to track whether we’re placing the contents of syllable notes into their box or actually printing that box (helps prevent spurious spaces from occurring when the box is being filled but not printed).

\macroname{\textbackslash ifgre@mustdotranslationcenterend}{}{gregoriotex-main.tex}
Boolean to track whether we must do translation centering.

\macroname{\textbackslash ifgre@beginningofscore}{}{gregoriotex-main.tex}
Boolean to mark the first syllable of the score (set to true until we start work on the first syllable, false afterwards).

\macroname{\textbackslash ifgre@endofscore}{}{gregoriotex-syllable.tex}
Boolean to mark the last syllable of the score.

\macroname{\textbackslash ifgre@firstglyph}{}{gregoriotex-syllable.tex}
Boolean that tells us if the current glyph is the first glyph or not.

\macroname{\textbackslash gre@attr@dash}{}{gregoriotex-main.tex}
A Lua\TeX\ attribute which indicates whether a syllable takes a dash if it ends a line.

\macroname{\textbackslash gre@attr@center}{}{gregoriotex-main.tex}
A Lua\TeX\ attribute which indicates the type of translation centering.

\macroname{\textbackslash gre@attr@glyph@id}{}{gregoriotex-main.tex}
A Lua\TeX\ attribute which identifies the glyph we are at.  Used for dynamic line spacing.

\macroname{\textbackslash gre@attr@glyph@top}{}{gregoriotex-main.tex}
A Lua\TeX\ attribute which identifies the high point of the glyph.  Used for dynamic line spacing.

\macroname{\textbackslash gre@attr@glyph@bottom}{}{gregoriotex-main.tex}
A Lua\TeX\ attribute which identifies the low point of the glyph.  Used for dynamic line spacing.

\macroname{\textbackslash gre@clivisalignment}{}{gregoriotex-syllable.tex}
Count to indicate how the clivis is to be aligned with its respective syllable text.  Values: \texttt{0}) always align clivis on its center; \texttt{1}) align clivis on first punctum; \texttt{2}) align clivis on its center, except if notes would go left of text or consonants after vowels are larger than \verb=\gre@dimen@clivisalignmentmin=.

\macroname{\textbackslash gre@insidediscretionary}{}{gregoriotex-signs.tex}
Macro which indicates whether we are currently inside a discretionary (\texttt{1}) or not (\texttt{0}).  Cannot be converted to a \TeX\ boolean because it’s value needs to be passed to Lua.

\macroname{\textbackslash ifgre@isonaline}{}{gregoriotex-syllable.tex}
Boolean which indicates whether the current note is on a line or not (used to adjust the height of some symbols so they won’t print on a line).

\macroname{\textbackslash ifgre@lastendswithmora}{}{gregoriotex-syllable.tex}
Boolean which indicates if the previous syllable ends with a punctum mora (set glyph by glyph, do not rely on it when typesetting glyph).

\macroname{\textbackslash ifgre@thisendswithmora}{}{gregoriotex-syllable.tex}
Same as previous one but for current syllable.

\macroname{\textbackslash ifgre@lastispunctumsave}{}{gregoriotex-signs.tex}
Boolean for storing \verb=\ifgre@lastispunctum= so that it can be restored later.

\macroname{\textbackslash ifgre@ledgerline@above}{}{gregoriotex-spaces.tex}
Boolean which indicates whether the system should act as if there is a ledger line above the staff.

\macroname{\textbackslash ifgre@ledgerline@below}{}{gregoriotex-spaces.tex}
Boolean which indicates whether the system should act as if there is a ledger line below the staff.

\macroname{\textbackslash gre@nlbstate}{}{gregoriotex-main.tex}
Macro which indicates if we are in a no line break area due to translation centering (\texttt{1}), an explicit no line break designation in the gabc (\texttt{2}), or not at all (\texttt{0}).

\macroname{\textbackslash gre@nlbinitialstate}{}{gregoriotex-main.tex}
Macro to store \verb=\gre@nlbstate= as we initialize or end a no line break area so that we can manipulate said flag as part of the process.

\macroname{\textbackslash ifgre@useledgerlineheuristic}{}{gregoriotex-spaces.tex}
Boolean which specifies whether ledger line heuristics will be used or not.

\macroname{\textbackslash ifgre@usestylefont}{}{gregoriotex-main.tex}
Boolean which specifies whether the style font should be loaded or not.

\macroname{\textbackslash ifgre@keeprightbox}{}{gregoriotex-signs.tex}
Boolean which specifies if we have to keep the localrightbox until the end.

\macroname{\textbackslash gre@compilegabc}{}{gregoriotex-main.tex}
Macro which specifies the default compilation behavior: never compile (\texttt{0}), auto compile (\texttt{1}), or always compile (\texttt{2}).

\macroname{\textbackslash ifgre@nabcfontloaded}{}{gregoriotex-nabc.tex}
Boolean which indicates whether the \texttt{nabc} font has been loaded.

\macroname{\textbackslash gre@generate@pointandclick}{}{gregoriotex-syllable.tex}
Count which indicates whether the point-and-click functionality should be implemented (\texttt{1}) or not (\texttt{0}).  Not a boolean because it needs to be readable by Lua.

\macroname{\textbackslash gre@variableheightexpansion}{}{gregoriotex-main.tex}
Count to indicated if the spacing between lines should be variable (\texttt{1}) or fixed (\texttt{0}).  Not a boolean because it needs to be readable by Lua.

\macroname{\textbackslash ifgre@blockeolcustos}{}{gregoriotex-main.tex}
Boolean which indicates whether the custos at the end of the line should be blocked.

\macroname{\textbackslash ifgre@blockeolcustosbeforeeuouae}{}{gregoriotex-main.tex}
Boolean which indicates whether the custos at the end of the line should be blocked if a EUOUAE block immediately follows.

\macroname{\textbackslash ifgre@raggedbreakbeforeeuouae}{}{gregoriotex-main.tex}
Boolean which indicates whether an automatic line break immediately before a EUOUAE block should be ragged.

\macroname{\textbackslash ifgre@breakintranslation}{}{gregoriotex-main.tex}
Boolean which indicates if line breaks are allowed inside a translation.

\macroname{\textbackslash ifgre@bolshiftsenabled}{}{gregoriotex-main.tex}
Boolean which indicates if the left shift for the first syllables of lines is enabled.

\macroname{\textbackslash ifgre@eolshiftsenabled}{}{gregoriotex-main.tex}
Boolean which indicates if the left shift for the last syllables of lines is enabled.

\macroname{\textbackslash ifgre@euouae@implies@nlba}{}{gregoriotex-main.tex}
Boolean which indicates if line breaks are prohibited in an \texttt{euouae} area.

\macroname{\textbackslash ifgre@in@euouae}{}{gregoriotex-main.tex}
Boolean which indicates that we are in an \texttt{euouae} area.

\macroname{\textbackslash ifgre@justifylastline}{}{gregoriotex-main.tex}
Boolean which indicates that the last line of the score should be justified.

\macroname{\textbackslash ifgre@showclef}{}{gregoriotex-main.tex}
Boolean which indicates that the clef should be visible.

\macroname{\textbackslash ifgre@forceemptyfirstsyllablehyphen}{}{gregoriotex-syllable.tex}
Boolean which indicates that a hyphen after an empty first syllable should be forced.

\macroname{\textbackslash ifgre@showhyphenafterthissyllable}{}{gregoriotex-syllable.tex}
Boolean used by \verb=\GreSyllable= to decide if a hyphen should be shown after the syllable.

\macroname{\textbackslash ifgre@thirdlineadjustmentnecessary}{}{gregoriotex-syllable.tex}
Boolean which indicates that a third-line adjustment to staff line width is necessary.

\macroname{\textbackslash ifgre@scale@stafflinefactor}{}{gregoriotex-spaces.tex}
Boolean indicating whether the stafflinefactor should scale with changes of \texttt{grefactor}, or not.

\macroname{\textbackslash ifgre@haslinethree}{}{gregoriotex-spaces.tex}
Boolean indicating whether the staff has a third line.

\macroname{\textbackslash ifgre@haslinefour}{}{gregoriotex-spaces.tex}
Boolean indicating whether the staff has a fourth line.

\macroname{\textbackslash ifgre@haslinefive}{}{gregoriotex-spaces.tex}
Boolean indicating whether the staff has a fifth line.

\macroname{\textbackslash gre@count@barshiftaftermora}{}{gregoriotex-signs.tex}
Count indicating when the presence of a punctum mora immediately before a bar line should affect the spacing.

\macroname{\textbackslash gre@count@lastglyphiscavum}{}{gregoriotex-spaces.tex}
Count indicating if the last glyph has a cavum (this includes flats, naturals, punctum cavum, etc.), when staff line shouldn't appear underneath this empty part. Can be:
\begin{itemize}
\item 0: previous and current glyph are not cavum
\item 1: current glyph is (set when line is hidden, at the end of the glyph)
\item 2: previous glyph is (set at beginning of glyph)
\end{itemize}

\macroname{\textbackslash ifgre@allowdeprecated}{}{gregoriotex.sty \textup{and} gregoriotex.tex}
Boolean which controls whether deprecated functions raise a warning (true) or an error (false).

\macroname{\textbackslash ifgre@newbarspacing}{}{gregoriotex-syllable.tex}
Boolean which controls whether the new bar spacing algorithm is activated.


\subsection{Boxes}

Boxes are used to store elements of the score before they are printed for the purposes of reusing them and/or measuring them in order to determine their appropriate placement.

\macroname{\textbackslash gre@box@hep}{}{gregoriotex-chars.tex}
Box for horizontal episemi.

\macroname{\textbackslash gre@box@temp@width}{}{gregoriotex-main.tex}
Box for holding an element in order to determine its width.

\macroname{\textbackslash gre@box@initial}{}{gregoriotex-main.tex}
Box which holds the initial of the score.

\macroname{\textbackslash gre@box@annotation}{}{gregoriotex-main.tex}
Box holding the annotation which goes above the initial.

\macroname{\textbackslash gre@box@commentary}{}{gregoriotex-main.tex}
Box holding the commentary which goes above the first staff line on the right.

\macroname{\textbackslash gre@box@add}{}{gregoriotex-main.tex}
Box used for the new line to be added to the box being built (used in multi-line commentaries and annotations).

\macroname{\textbackslash gre@box@old}{}{gregoriotex-main.tex}
Box used for the existing lines in the box being build when a new line is being added (used in multi-line commentaries and annotations).

\macroname{\textbackslash gre@box@lines}{}{gregoriotex-main.tex}
Box holding the staff lines.

\macroname{\textbackslash gre@box@temp@sign}{}{gregoriotex-signs.tex}
Box to hold a sign so we can measure it for placement.

\macroname{\textbackslash gre@box@temp@clef}{}{gregoriotex-signs.tex}
Box for holding (and measuring) the clef when stacking non-overlapping clefs.

\macroname{\textbackslash gre@box@temp@cleftwo}{}{gregoriotex-signs.tex}
Box for holding (and measuring) the secondary clef when stacking non-overlapping clefs.

\macroname{\textbackslash gre@box@syllablenotes}{}{gregoriotex-syllable.tex}
Box holding the notes associated with a syllable.

\macroname{\textbackslash gre@box@syllabletext}{}{gregoriotex-syllable.tex}
Box holding the text associated with a syllable.

\macroname{\textbackslash gre@box@hep}{}{gregoriotex-chars.tex}
Box holding the horizontal episema.



\subsection{Distances}
All of the distances listed in \nameref{distances} have an internal associated
with them, of the form of \verb=\gre@space@*@*=, which stores the value of the
distance (in string representation).  The first wildcard is either
\texttt{skip} or \texttt{dimen} according to the distance type, while the
second is the name of the distance.

These additional distances are calculated by Gregorio based on the values for the user customizable distances and what may be going on in the score at the time of their use.

\macroname{\textbackslash gre@dimen@morawidth}{}{gregoriotex-spaces.tex}
Width of a punctum mora (reinitialized at each score, lazily recomputed).

\macroname{\textbackslash gre@dimen@clefwidth}{}{gregoriotex-spaces.tex}
Width of the current clef.

\macroname{\textbackslash gre@dimen@constantglyphraise}{}{gregoriotex-spaces.tex}
Dimension representing the space between the 0 of the gregorian fonts and the effective 0 of the TeX score.

\macroname{\textbackslash gre@dimen@currenttranslationheight}{}{gregoriotex-spaces.tex}
Dimension representing the space for the translation beneath the text.

\macroname{\textbackslash gre@dimen@stafflinewidth}{}{gregoriotex-spaces.tex}
Dimension representing the width of a line of staff.  Can vary, for
example, at the first line.

\macroname{\textbackslash gre@dimen@linewidth}{}{gregoriotex-spaces.tex}
Dimension representing the width of the score (including initial).

\macroname{\textbackslash gre@dimen@additionalbottomspace}{}{gregoriotex-spaces.tex}
Dimension representing extra space below the staff needed for low notes.

\macroname{\textbackslash gre@dimen@additionaltopspace}{}{gregoriotex-spaces.tex}
Dimension representing extra space above the staff needed for high notes.

\macroname{\textbackslash gre@dimen@textlower}{}{gregoriotex-spaces.tex}
Dimension representing the height of the separation between the 0th
line (which is invisible except for notes in the a or b position) and
the bottom of the text.

\macroname{\textbackslash gre@dimen@textaligncenter}{}{gregoriotex-spaces.tex}
Dimension representing the width from the beginning of the letters in
a syllable to the middle of the middle letters.  Used for lining up
neumes and syllables.

\macroname{\textbackslash gre@dimen@additionalleftspace}{}{gregoriotex-spaces.tex}
Dimension representing the additional space that has to be added to
the localleftbox for a big initial (one taking two lines).

\macroname{\textbackslash gre@dimen@initialwidth}{}{gregoriotex-spaces.tex}
Dimension representing the width of the initial (and the space after).

\macroname{\textbackslash gre@dimen@currentabovelinestextheight}{}{gregoriotex-spaces.tex}
Dimension representing the space allocated above the lines for text.

\macroname{\textbackslash gre@dimen@staffheight}{}{gregoriotex-spaces.tex}
The total height of the staff including the width of the lines and the spaces between them.

\macroname{\textbackslash gre@dimen@stafflinediff}{}{gregoriotex-spaces.tex}
Distance representing the difference between the actual size of the staff lines and the ``standard’’ size.

\macroname{\textbackslash gre@dimen@stafflineheight}{}{gregoriotex-spaces.tex}
The height of the staff line.

\macroname{\textbackslash gre@dimen@interstafflinespace}{}{gregoriotex-spaces.tex}
The space between the lines.

\macroname{\textbackslash gre@dimen@glyphraisevalue}{}{gregoriotex-spaces.tex}
The value that a particular glyph must be raised to be set in the correct position.

\macroname{\textbackslash gre@dimen@enddifference}{}{gregoriotex-spaces.tex}
Distance from the end of the notes to the end of the text for the previous syllable.  Positive values when notes go further than text, negative in the other case.

\macroname{\textbackslash gre@dimen@previousenddifference}{}{gregoriotex-spaces.tex}
Stored value of enddifference prior to the current one.

\macroname{\textbackslash gre@skip@nextbegindifference}{}{gregoriotex-spaces.tex}
The difference between the start of the notes and the start of the text for the next syllable.  Positive when when text begins first, negative in other case.

\macroname{\textbackslash gre@dimen@begindifference}{}{gregoriotex-spaces.tex}
The difference between the start of the notes and the start of the text for the current syllable.  Positive when when text begins first, negative in other case.

\macroname{\textbackslash gre@dimen@lastglyphwidth}{}{gregoriotex-spaces.tex}
The width of the last glyph.

\macroname{\textbackslash gre@dimen@notesaligncenter}{}{gregoriotex-spaces.tex}
Distance from beginning of notes to their point of alignment.

\macroname{\textbackslash gre@dimen@temp@...}{}{gregoriotex-spaces.tex}
Temporary dimensions used in calculations.  There are currently five of these.

\macroname{\textbackslash gre@skip@temp@...}{}{gregoriotex-spaces.tex}
Temporary skips used in calculations.  There are currently four of these.

\macroname{\textbackslash gre@dimen@savedglyphraise}{}{gregoriotex-signs.tex}
Macro to hold the value of the glyph raise so that it can be restored after some calculations which may change it are performed.

\macroname{\textbackslash gre@dimen@eolshift}{}{gregoriotex-spaces.tex}
The left kern that should appear before an end of line.

\macroname{\textbackslash gre@dimen@bolshift}{}{gregoriotex-spaces.tex}
The left kern that should appear at the beginning of line in case of a forced line break.

\macroname{\textbackslash gre@dimen@bolextra}{}{gregoriotex-spaces.tex}
An extra space that is added to \verb=\gre@dimen@bolshift= when the first glyph is a flat or a natural.

\macroname{\textbackslash gre@dimen@annotationtrueraise}{}{gregoriotex-spaces.tex}
The distance from the baseline of the initial to the baseline of the annotation.

\macroname{\textbackslash gre@dimen@commentarytrueraise}{}{gregoriotex-spaces.tex}
The distance from the baseline of the initial to the baseline of the commentary.

\macroname{\textbackslash gre@skip@minTextDistance}{}{gregoriotex-spaces.tex}
Minimum distance between text.

\macroname{\textbackslash gre@skip@minNotesDistance}{}{gregoriotex-spaces.tex}
Minimum distance between notes.

\macroname{\textbackslash gre@dimen@curTextDistance}{}{gregoriotex-spaces.tex}
Current distance between text.

\macroname{\textbackslash gre@dimen@curNotesDistance}{}{gregoriotex-spaces.tex}
Current distance between notes.

\macroname{\textbackslash gre@skip@minShiftText}{}{gregoriotex-spaces.tex}
Minimum shift required for the text.

\macroname{\textbackslash gre@skip@minShiftNotes}{}{gregoriotex-spaces.tex}
Minimum shift required for the notes.

\macroname{\textbackslash gre@save@parfillskip}{}{gregoriotex-main.tex}
Macro to store \verb=\parfillskip= value so that we can restore it at the end of the score (needed to force the last line of a score to be justified).

\macroname{\textbackslash gre@scaledist}{}{gregoriotex-spaces.tex}
Working alias for \verb=\gre@skip@temp@one= or \verb=\gre@dimen@temp@one=, as appropriate, used when rescaling a distance due to a change in \verb=\gre@factor=.

\macroname{\textbackslash gre@skip@syllablefinalskip}{}{gregoriotex-spaces.tex}
The final distance to skip at the end of a syllable.

\macroname{\textbackslash greslurheight}{}{gregoriotex-signs.tex}
Stores the computed height of a variable-length slur.  The control sequence name
does not have the \texttt{@} symbol because this dimension is used within \MP{}.

\subsection{Penalties}
These are the macros that Gregorio\TeX\ uses to manipulate the penalties in order to control line and page breaks within a score without affect the surrounding text.

\macroname{\textbackslash gre@penalty}{\#1}{gregoriotex-signs.tex}
Top level function that aliases \verb=\gre@truepenalty= or \verb=\gre@falsepenalty= according to whether penalties should be in play or not.

\begin{argtable}
	\#1 & integer & The penalty to be applied or gobbled\\
\end{argtable}

\macroname{\textbackslash gre@truepenalty}{\#1}{gregoriotex-signs.tex}
Alias for \verb=\penalty=.

\begin{argtable}
	\#1 & integer & The penalty to be applied\\
\end{argtable}

\macroname{\textbackslash gre@falsepenalty}{\#1}{gregoriotex-signs.tex}
Macro to gobble (suppress) its argument.

\begin{argtable}
	\#1 & integer & The penalty to be gobbled\\
\end{argtable}

\macroname{\textbackslash gre@cancelpenalties}{}{gregoriotex-spaces.tex}
Macro called at the beginning of the score to swap text penalties for score penalties.

\macroname{\textbackslash gre@restorepenalties}{}{gregoriotex-spaces.tex}
Macro called at the end of the score to restore the text penalties.

\macroname{\textbackslash gre@brokenpenaltysave}{}{gregoriotex-spaces.tex}
A place to save the current value of the broken penalty so that we can change it temporarily and then restore it later.

\macroname{\textbackslash gre@clubpenaltysave}{}{gregoriotex-spaces.tex}
A place to save the current value of the club penalty so that we can change it temporarily and then restore it later.

\macroname{\textbackslash gre@widowpenaltysave}{}{gregoriotex-spaces.tex}
A place to save the current value of the widow penalty so that we can change it temporarily and then restore it later.

\macroname{\textbackslash gre@emergencystretchsave}{}{gregoriotex-spaces.tex}
A place to save the current value of the emergency stretch so that we can change it temporarily and then restore it later.

\macroname{\textbackslash gre@endafterbarpenaltysave}{}{gregoriotex-main.tex}
A place to save the current value of the end after bar penalty so that we can change it temporarily and then restore it later.

\macroname{\textbackslash gre@endafterbaraltpenaltysave}{}{gregoriotex-main.tex}
A place to save the current value of the alternate end after bar penalty so that we can change it temporarily and then restore it later.

\macroname{\textbackslash gre@endofelementpenaltysave}{}{gregoriotex-main.tex}
A place to save the current value of the end of element penalty so that we can change it temporarily and then restore it later.

\macroname{\textbackslash gre@endofsyllablepenaltysave}{}{gregoriotex-main.tex}
A place to save the current value of the end of syllable penalty so that we can change it temporarily and then restore it later.

\macroname{\textbackslash gre@endofwordpenaltysave}{}{gregoriotex-main.tex}
A place to save the current value of the end of word penalty so that we can change it temporarily and then restore it later.

\macroname{\textbackslash gre@exhyphenpenaltysave}{}{gregoriotex-spaces.tex}
A place to save the current value of the ex hyphen penalty so that we can change it temporarily and then restore it later.

\macroname{\textbackslash gre@hyphenpenaltysave}{}{gregoriotex-main.tex \textup{and} gregoriotex-spaces.tex}
A place to save the current value of the hyphen penalty so that we can change it temporarily and then restore it later.

\macroname{\textbackslash gre@nobreakpenaltysave}{}{gregoriotex-main.tex}
A place to save the current value of the no break penalty so that we can change it temporarily and then restore it later.

\macroname{\textbackslash gre@doublehyphendemeritssave}{}{gregoriotex-spaces.tex}
A place to save the current value of the double hyphen demerits so that we can change it temporarily and then restore it later.

\macroname{\textbackslash gre@finalhyphendemeritssave}{}{gregoriotex-spaces.tex}
A place to save the current value of the final hyphen demerits so that we can change it temporarily and then restore it later.

\macroname{\textbackslash gre@loosenesssave}{}{gregoriotex-spaces.tex}
A place to save the current value of the looseness so that we can change it temporarily and then restore it later.

\macroname{\textbackslash gre@tolerancesave}{}{gregoriotex-spaces.tex}
A place to save the current value of the tolerance so that we can change it temporarily and then restore it later.

\macroname{\textbackslash gre@pretolerancesave}{}{gregoriotex-spaces.tex}
A place to save the current value of the pretolerance so that we can change it temporarily and then restore it later.



\subsection{\LaTeX/Plain \TeX\ compatibility}
While every effort has been made to use only primitives which are compatible with both \LaTeX\ and Plain \TeX, it is sometimes necessary to use primitives which are defined for one but not the other (usually its \LaTeX\ that has what we need built in).  In these cases we have to provide an equivalent macro to the \TeX\ version which is lacking.

\macroname{\textbackslash MessageBreak}{}{gregoriotex.tex}
Creates a line break in typeout, warning, bug, and error messages.  Copied from \LaTeX\ source.

\macroname{\textbackslash protect}{}{gregoriotex.tex}
Prints the name of the macro, rather than its contents in typeout, warning, bug, and error messages.  Copied from \LaTeX\ source.

\macroname{\textbackslash f@size}{}{gregoriotex.tex}
Macro which stores the current font size.


\section{Special arguments}

These arguments are used by multiple functions and take a lot of space
to describe so we describe them once here and refer to this section
rather than have multiple definitions.

\subsection{Note Alignment Type}\label{notesalign}
\rowcolors{1}{lightgray}{lightgray}
\begin{tabulary}{\textwidth}{cL}
	\multicolumn{2}{c}{Integer with the following possibilities:} \\
	\hline
	\texttt{0} & one-note glyph or more than two notes glyph except porrectus : here we must put the aligncenter in the middle of the first note\\
	\texttt{1} & two notes glyph (podatus is considered as a one-note glyph) : here we put the aligncenter in the middle of the glyph\\
	\texttt{2} & porrectus : has a special align center\\
	\texttt{3} & initio-debilis : same as 1 but the first note is much smaller\\
	\texttt{4} & case of a glyph starting with a quilisma\\
	\texttt{5} & case of a glyph starting with an oriscus\\
	\texttt{6} & case of a punctum inclinatum\\
	\texttt{7} & case of a stropha\\
	\texttt{8} & flexus with an ambitus of one\\
	\texttt{9} & flexus deminutus\\
	\texttt{10} & virgula\\
	\texttt{11} & divisio minima, minor and maior\\
	\texttt{12} & divisio finalis
 \end{tabulary}

\subsection{Episema Special}\label{EpisemaSpecial}
\definecolor{shadecolor}{named}{lightgray}%
\begin{shaded*}%
\vspace{-1.4\baselineskip}
\begin{center}String with the following possibilities:\end{center}
\vspace{-0.8\baselineskip}
\hrule
\vspace{-0.8\baselineskip}
\begin{description}
	\item[FinalPunctum] Last note, which is a standard punctum (works with pes).
	\item[FinalDeminutus] Same, but the last note is a deminutus.
	\item[PenultBeforePunctumWide] The note before the last note, which is a standard punctum.
	\item[PenultBeforeDeminutus] Idem, but the note is the note preceding a deminutus.
	\item[AntepenultBeforePunctum] The note before the note before the last note (for porrectus flexus).
	\item[AntepenultBeforeDeminutus] Idem, but when the two last notes are a deminutus.
	\item[InitialPunctum] The first note, if it is a standard punctum.
	\item[InitioDebilis] The first note, if it is an initio debilis.
	\item[PorrNonAuctusInitialWide] first note of a non-auctus porrectus with a second ambitus of at least two.
	\item[PorrNonAuctusInitialOne] first note of a non-auctus porrectus with a second ambitus of one
	\item[PorrAuctusInitialAny] first note of an auctus porrectus, regardless of second ambitus
	\item[FinalInclinatum] punctum inclinatum as last note
	\item[FinalInclinatumDeminutus] punctum inclinatum deminutus as last note
	\item[FinalStropha] stropha as last note
	\item[FinalQuilisma] quilisma as last note
	\item[FinalOriscus] oriscus as last note
	\item[PenultBeforePunctumOne] second-to-last note, with a second ambitus of one, when last note is a standard punctum (like the second note of ghg)
	\item[FinalUpperPunctum] “upper smaller punctum” as last note (concerning simple podatus, podatus, and torculus resupinus)
	\item[InitialOriscus] oriscus as first note, disconnected from next note
	\item[InitialQuilisma] quilisma as first note, disconnected from next note
	\item[TorcResNonAuctusSecondWideWide] second note of a non-auctus torculus resupinus starting with a punctum, with a first and second ambitus of at least two
	\item[TorcResNonAuctusSecondOneWide] second note of a non-auctus torculus resupinus starting with a punctum, with a first ambitus of one and a second ambitus of at least two
	\item[TorcResDebilisNonAuctusSecondAnyWide] second note of a non-auctus torculus resupinus initio debilis with any first ambitus and a second ambitus of at least two
	\item[FinalLineaPunctum] linea punctum (cavum) as last note
	\item[BarStandard] standard bar
	\item[BarVirgula] virgula
	\item[BarDivisioFinalis] divisio finalis
	\item[TorcResQuilismaNonAuctusSecondWideWide] second note of a non-auctus torculus resupinus starting with a quilisma, with a first and second ambitus of at least two
	\item[TorcResOriscusNonAuctusSecondWideWide] second note of a non-auctus torculus resupinus starting with an oriscus, with a first and second ambitus of at least two
	\item[TorcResQuilismaNonAuctusSecondOneWide] second note of a non-auctus torculus resupinus starting with a quilisma, with a first ambitus of one and and second ambitus of at least two
	\item[TorcResOriscusNonAuctusSecondOneWide] second note of a non-auctus torculus resupinus starting with an oriscus, with a first ambitus of one and and second ambitus of at least two
	\item[TorcResNonAuctusSecondWideOne] second note of a non-auctus torculus resupinus starting with a punctum, with a first ambitus of at least two and a second ambitus of one
	\item[TorcResDebilisNonAuctusSecondAnyOne] second note of a non-auctus torculus resupinus initio debilis with any first ambitus and a second ambitus of one
	\item[TorcResQuilismaNonAuctusSecondWideOne] second note of a non-auctus torculus resupinus starting with a quilisma, with a first ambitus of at least two and a second ambitus of one
	\item[TorcResOriscusNonAuctusSecondWideOne] second note of a non-auctus torculus resupinus starting with an oriscus, with a first ambitus of at least two and a second ambitus of one
	\item[TorcResNonAuctusSecondOneOne] second note of a non-auctus torculus resupinus starting with a punctum, with a first and second ambitus of one
	\item[TorcResQuilismaNonAuctusSecondOneOne] second note of a non-auctus torculus resupinus starting with a quilisma, with a first and second ambitus of one
	\item[TorcResOriscusNonAuctusSecondOneOne] second note of a non-auctus torculus resupinus starting with an oriscus, with a first and second ambitus of one
	\item[TorcResAuctusSecondWideAny] second note of an auctus torculus resupinus starting with a punctum, with a first ambitus of at least two and any second ambitus
	\item[TorcResDebilisAuctusSecondAnyAny] second note of an auctus torculus resupinus initio debilis with any first and second ambitus
	\item[TorcResQuilismaAuctusSecondWideAny] second note of an auctus torculus resupinus starting with a quilisma, with a first ambitus of at least two and any second ambitus
	\item[TorcResOriscusAuctusSecondWideAny] second note of an auctus torculus resupinus starting with an oriscus, with a first ambitus of at least two and any second ambitus
	\item[TorcResAuctusSecondOneAny] second note of an auctus torculus resupinus starting with a punctum, with a first ambitus of one and any second ambitus
	\item[TorcResQuilismaAuctusSecondOneAny] second note of an auctus torculus resupinus starting with a quilisma, with a first ambitus of one and any second ambitus
	\item[TorcResOriscusAuctusSecondOneAny] second note of an auctus torculus resupinus starting with an oriscus, with a first ambitus of one and any second ambitus
	\item[ConnectedPenultBeforePunctumWide] second-to-last note connected to prior note, with a second ambitus of at least two, when last note is a standard punctum (like the second note of \textit{gig})
	\item[ConnectedPenultBeforePunctumOne] second-to-last note connected to prior note, with a second ambitus of one, when last note is a standard punctum (like the second note of \textit{gih})
	\item[InitialConnectedPunctum] standard punctum as first note, connected to next higher note
	\item[InitialConnectedVirga] “virga” as first note, connected to next lower note
	\item[InitialConnectedQuilisma] quilisma as first note, connected to next higher note
	\item[InitialConnectedOriscus] oriscus as first note, connected to next higher note
	\item[FinalConnectedPunctum] punctum as last note, connected to prior higher note
	\item[FinalConnectedAuctus] auctus as last note, connected to prior lower note
	\item[FinalVirgaAuctus] virga aucta as last note
	\item[FinalConnectedVirga] “virga” as last note, connected to prior lower note
	\item[InitialVirga] “virga” as first note, disconnected from next note
\end{description}
\end{shaded*}

%%% Local Variables:
%%% mode: latex
%%% TeX-master: “GregorioRef”
%%% End:

% !TEX root = GregorioRef.tex
% !TEX program = LuaLaTeX+se
\section{The GABC File}

gabc is a simple notation based exclusively on ASCII characters that
enables the user to describe Gregorian chant scores. The name
\textit{gabc} was given in reference to the
\href{http://www.walshaw.plus.com/abc/}{ABC} notation for modern
music.

The gabc notation was developed by a monk of the
\href{http://www.barroux.org}{Abbey of Sainte Madeleine du Barroux}
and has been improved by Élie Roux and by other monks of the same
abbey to produce the best possible notation.

This section will cover the elements of a gabc file.

\subsection{File Structure}
Files written in gabc have the extension \texttt{.gabc} and have the
following structure:

\begin{lstlisting}[autogobble]
name: incipit;
gabc-copyright: copyright on this gabc file;
score-copyright: copyright on the source score;
author: if known;
language: latin;
mode: 6;
mode-modifier: t.;
annotation: IN.;
annotation: 6;
%%
(clef) text(notes)
\end{lstlisting}

\subsection{Headers}

The headers, such as \texttt{name: incipit;}, above, each have a name
before the colon and a value, between the colon and the semicolon.  The
header name is composed of ASCII letters and numbers, optionally separated
by dashes.  If you wish to write a value over several lines, omit the
semicolon at the end of the first line, and end the header value with
\texttt{;;} (two semicolons).

Some headers have special meaning to Gregorio:

\begin{description}
\item[name] This is the name of the piece, in almost all cases the
	incipit, the first few words. In the case of the mass ordinary, the
	form as \texttt{Kyrie X Alme Pater} or \texttt{Sanctus XI} is
	recommended where appropriate. \textbf{This field is required.}
\item[gabc-copyright] This license is the copyright notice (in English) of the gabc file, as chosen by the person named in the transcriber field. As well as the notice itself, it may include a brief description of the license, such as public domain, CC-by-sa; for a list of commonly found open source licenses and exceptions, please see \url{https://spdx.org/licenses/}.  A separate text file will be necessary for the complete legal license. For the legal issues about Gregorian chant scores, please see \url{http://gregorio-project.github.io/legalissues}. An example of this field would be:
	\begin{lstlisting}[autogobble]
		gabc-copyright: CC0-1.0 by Elie Roux, 2009 <http://creativecommons.org/publicdomain/zero/1.0/>;
	\end{lstlisting}
\item[score-copyright] This license is the copyright notice (in English) of the score itself from which the gabc was transcribed. Like the \texttt{gabc-copyright}, there may be a brief description of the license too. In unclear or complex cases it may be omitted; it is most suitable for use when the transcriber is the copyright holder and licensor of the score as well. One again, reading the page on legal issues (linked above) is recommended. An example of this field would be:
	\begin{lstlisting}[autogobble]
		score-copyright: (C) Abbaye de Solesmes, 1934;
	\end{lstlisting}
\item[author] The author of the piece, if known; of course, the author of most traditional chant is not known.
\item[language] The language of the lyrics.
\item[oriscus-orientation] If \texttt{legacy}, the orientation of an unconnected oriscus must be set manually.
\item[mode] The mode of the piece. This should normally be an arabic
	number between 1 and 8, but may be any text required for unusual
	cases. The mode number will be converted to roman numerals and
	placed above the initial unless one of the following conditions are
	met:
	\begin{itemize}
	\item There is a \verb=\greannotation= defined immediatly prior to \verb=\gregorioscore=.
	\item The \texttt{annotation} header field is defined.
	\end{itemize}
\item[mode-modifier] The mode ``modifier'' of the piece. This may be any
	\TeX\ code to typeset after the mode, if the mode is typeset.  If the mode
	is not typeset, the mode-modifier will also not be typeset.
\item[mode-differentia] The mode or tone differentia of the piece.  Typically,
	this expresses the variant of the psalm tone to use for the piece.  This may
	be any \TeX\ code to typeset after the mode-modifier, if the mode is typeset.
	If the mode is not typeset, the mode-differentia will also not be typeset.
\item[annotation] The annotation is the text to appear above the
	initial letter. Usually this is an abbreviation of the office-part
	in the upper line, and an indication of the mode (and differentia
	for antiphons) in the lower. Either one or two annotation fields may
	be used; if two are used, the first is the upper line, the second
	the lower. Example:
	\begin{lstlisting}[autogobble]
		annotation:Ad Magnif.;
		annotation:VIII G;
	\end{lstlisting}
	Full \TeX\ markup is accepted:
	\begin{lstlisting}[autogobble]
		annotation:{\color{red}Ad Magnif.};
		annotation:{\color{red}VIII G};
	\end{lstlisting}
	If the user already defined annotation(s) in the main \TeX\ file via
	\verb=\greannotation= then the \texttt{annotation} header field will not
	overwrite that definition.
\end{description}

Although gregorio ascribes no special meaning to them, other suggested headers are:

\begin{description}
\item[office-part] The office-part is the category of chant (in Latin), according to its liturgical rôle. Examples are: antiphona, hymnus, responsorium brevium, responsorium prolixum, introitus, graduale, tractus, offertorium, communio, kyrie, gloria, credo, sanctus, benedictus, agnus dei.
\item[occasion] The occasion is the liturgical occasion, in Latin. For example, Dominica II Adventus, Commune doctorum, Feria secunda.
\item[meter] For hymns and anything else with repetitive stanzas, the meter, the numbers of syllables in each line of a stanza. For example, 8.8.8.8 for typical Ambrosian-style hymns: 4 lines each of 8 syllables.
\item[commentary] This is intended for notes about the source of the text, such as references to the Bible.
\item[arranger] The name of a modern arranger, when a traditional chant melody has been adapted for new words, or when a manuscript is transcribed into square notation. This may be a corporate name, like Solesmes.
\item[date] The date of composition, or the date of earliest attestation. With most traditional chant, this will only be approximate; e.g. XI. s. for eleventh century. The convention is to put it with the latin style, like the previous examples (capital letters, roman numerals, s for seculum and the dots).
\item[manuscript] For transcriptions direct from a manuscript, the text normally used to identify the manuscript, for example Montpellier H.159
\item[manuscript-reference] A unique reference for the piece, according to some well-known system. For example, the reference beginning cao in the Cantus database of office chants. If the reference is unclear as to which system it uses, it should be prefixed by the name of the system. Note that this should be a reference identifying the piece, not the manuscript as a whole; anything identifying the manuscript as a whole should be put in the manuscript field.
\item[manuscript-storage-place] For transcriptions direct from a manuscript, where the manuscript is held; e.g. Bibliothèque Nationale, Paris.
\item[book] For transcriptions from a modern book (such as Solesmes editions; modern goes back at least to the 19th century revival), the name of the book; e.g. Liber Usualis.
\item[transcriber] The name of the transcriber into gabc.
\item[transcription-date] The date the gabc was written, with the following convention yyyymmdd, like 20090129 for January the 29th 2009.
\item[user-notes] This may contain any text in addition to the other headers -- any notes the transcriber may wish. However, it is recommended to use the specific header fields where they are suitable, so that it is easier to find particular information.
\end{description}

\subsubsection{Mode Headers}

The three mode headers described above (\texttt{mode}, \texttt{mode-modifier},
and \texttt{mode-differentia}) will be typeset above the initial if neither
the \texttt{annotation} gabc header nor the \verb=\greannotation= \TeX{}
command is used.

The mode annotation will look like
\writemode{mode}{\thinspace mode-modifier}{\thinspace mode-differentia}.

The \texttt{mode} header is typically a number that will be typeset as a
Roman numeral using the \texttt{modeline} style.  Therefore, if the first
character of \texttt{mode} is a number from one (\texttt{1}) through eight
(\texttt{8}), that number will be converted according to the
\verb=\gresetmodenumbersystem= setting.  However, there are other modes,
so all other parts of \texttt{mode} will be typeset directly.  If the
\texttt{mode} header is omitted, none of the other mode headers will be
typeset.

The \texttt{mode-modifier} header is some text (typeset in the
\texttt{modemodifier} style) that appears after \texttt{mode}, but before
\texttt{mode-differentia}.  This is meant for an extra notation that
indicates something without altering the mode itself.  An example would be
\writemode{}{t.}{} to indicate a transposed mode.  If the
\texttt{mode-modifier} header starts with punctuation, there will be no space
before it, otherwise there will be a \verb=\thinspace= before it.

The \texttt{mode-differentia} header is some text (typeset in the
\texttt{modedifferentia} style) that appears after \texttt{mode-modifier}.
This is meant for indicating the psalm tone ending to use for the paired
psalm tone.  If the \texttt{mode-differentia} header starts with punctuation,
there will be no space before it, otherwise there will be a \verb=\thinspace=
before it.

Some examples:

\begin{tabularx}{\textwidth}{l|l|l|X}
	\texttt{mode} & \texttt{mode-modifier} & \texttt{mode-differentia} & Result \\
	\hline
	\verb=6= & & & \writemode{\romannumeral 6}{}{} \\
	\verb=4A= & & & \writemode{\romannumeral 4\relax A}{}{} \\
	\verb=4a= & & & \writemode{\romannumeral 4\relax a}{}{} \\
	\verb=2*= & \verb=t.= & & \writemode{\romannumeral 2*}{\thinspace t.}{} \\
	\verb=5= & \verb=,\thinspace t.= & & \writemode{\romannumeral 5}{,\thinspace t.}{} \\
	\verb=7= & & {\scriptsize\verb=c\raise0.5ex\hbox{\small2}=} & \writemode{\romannumeral 7}{}{\thinspace c\raise0.5ex\hbox{\small2}} \\
	\verb=8= & \verb=-t.= & \verb=G*= & \writemode{\romannumeral 8}{-t.}{\thinspace G*} \\
	{\scriptsize\verb=t. irregularis=} & & & \writemode{t. irregularis}{}{} \\
\end{tabularx}

\subsection{Neume Fusion}

Neume fusion allows for the composition of new shapes based on a set of
primitive neumes.  These primitives are:

\begin{tabularx}{\textwidth}{l|l|X}
	Gabc & Description & Rules \\
	\hline
	\texttt{g} & punctum & fuses from higher or lower notes, and can fuse to higher or lower notes \\
	\texttt{go} & oriscus & may only fuse in the direction it was fused from \\
	\texttt{gO} & oriscus scapus & at the start only, next note must be higher to fuse \\
	\texttt{gw} & quilisma & does not fuse from anything, and only fuses to a higher note \\
	\texttt{gV} & virga reversa & at the start only, next note must be lower to fuse \\
	\texttt{gf} & flexus & if not at the end, must be followed by a higher note to fuse \\
	\texttt{gh} & pes & at the end only; in non-liquescent form, the previous note must be lower to fuse \\
	\texttt{gfg} & porrectus & at the end only, previous note must be lower to fuse \\
\end{tabularx}

Placing the \texttt{@} character between two notes will attempt to use the above
rules to fuse the notes together.  If a shape that is not fusable is used,
Gregorio will typically fall back on the non-fusable form, but in some cases
will result in a syntax error.

Placing the \texttt{@} character before a primitive that would get a stem will
suppress the stem.  Given the above list of primitives, this means the flexus
and the porrectus.

Here are some examples of fusion:

\gresetinitiallines{0}\gresetlyriccentering{firstletter}%
\gabcsnippet{
(c3) h@iw@ji@j@ih<sp>~</sp>(h@iw@ji@j@ih~)
(;) d@eo@fd(d@eo@fd)
(;) IJ@kh(IJ@kh)
}

As a convenience, a sequence of notes enclosed within \texttt{@[} and
\texttt{]} will be fused automatically based on an algorithm that breaks up
the notes into the above primitives.  Using the same examples as before:

\gabcsnippet{
(c3) @<v>[</v>hiwjijih<sp>~</sp><v>]</v>(@[hiwjijih~])
(;) @<v>[</v>deofd<v>]</v>(@[deofd])
(;) @<v>[</v>IJkh<v>]</v>(@[IJkh])
}

\subsection{Stem length for the bottom lines}

Gregorio will determine the length of the stem for most neumes.
Some manual input might be needed for notes on the bottom staff
line (\textit{d}). Most of the time they will take a short form:

\gabcsnippet{(c3) dv(dv) ed(ed) ed~(ed~) dcd(dcd)}

But when a ledger line is drawn below these notes, they should take a long
form. The problem is that many cases are ambiguous: for instance if a note
is close to a ledger line, one may want to make it long, others may not.
To solve this problem, you can add \texttt{[ll:0]} to the note carrying the stem
to get its short form, or \texttt{[ll:1]} to force its long form.

% This snippet makes LuaTeX segfault!
%\gabcsnippet{
%  (c3) dv<v>[</v>ll:1<v>]</v>(dv[ll:1]) ed<v>[</v>ll:1<v>]</v>(ed[ll:1])
%       ed<sp>~</sp><v>[</v>ll:1<v>]</v>(ed~[ll:1]) dcd<v>[</v>ll:1<v>]</v>(dcd[ll:1] Z)
%       b!dv<v>[</v>ll:0<v>]</v>(b!dv[ll:0]) b!ed<v>[</v>ll:0<v>]</v>(b!ed[ll:0])
%       b!ed<sp>~</sp><v>[</v>ll:0<v>]</v>(b!ed~[ll:0]) dcd<v>[</v>ll:0<v>]</v>!b(dcd[ll:0]!b)
%  }

\subsection{Custom Ledger Lines}

To specify a custom ledger line, use
\texttt{[oll:}\textit{left}\texttt{;}\textit{right}\texttt{]} to create an
over-the-staff ledger line with specified lengths to the left and right of the
point where it is introduced.  If \textit{left} is \texttt{0}, the ledger line
will start at the introduction point.  If \textit{left} is \texttt{1}, the
ledger line will start at the \textit{additionaallineswidth} distance to the
left of the introduction point.  Otherwise, the line will start at the
\textit{left} distance (taken to be an explicit length, with \TeX{} units
required) to the left of the introduction point.  When using this form,
\texttt{right} must be an explicit length to the right of the introduction
point at which to end the line.

Alternately, use
\texttt{[oll:}\textit{left}\texttt{\{}\textit{right}\texttt{]} to specify the
start of an over-the-staff ledger line, followed by \texttt{[oll:\}]} at some
point later to specify its end.  When using this form, \textit{left} has the
same meaning as before.  However, \textit{right} takes on similar values as
\textit{left}, which are instead applied to the right of the specified
endpoint.

Use \texttt{ull} instead of \texttt{oll} (with either form) to create an
under-the-staff ledger line.

When using this feature with fusion, you will not be able to start or end a
ledger line in the middle of two-note primitive shapes.  To work around this,
either adjust the parameters of the ledger line or use manual fusion to break
up those two notes.

\subsection{Simple Slurs}

To specify a simple slur, use
\texttt{[oslur:}\textit{shift}\texttt{;}\textit{width}\texttt{,}\textit{height}\texttt{]}
to create an over-the-notes slur with the specified \textit{width} and
\textit{height}.  If \textit{shift} is \texttt{0}, the slur will start on the
right side of the note to which it is atteched.  If \textit{shift} is
\texttt{1}, the slur will start one punctum's width to the left of the right
side of the note to which it is attached.  If \textit{shift} is \texttt{2},
the slur will start one-half punctum's width to the left of the right side of
the note to which it is attached.

Alternately, use
\texttt{[oslur:}\textit{shift}\texttt{\{]} to specify the start of an
over-the-notes slur, followed by \texttt{[oslur:}\textit{shift}\texttt{\}]} at
some point later to specify its end.  When using this form, \textit{shift} has
the same meaning as before, but applies to both ends of the slur.

Use \texttt{uslur} instead of \texttt{oslur} (with either form) to create an
under-the-staff slur.

\subsection{Horizontal episema placement for very high and low notes}

Gregorio places horizontal episema under c and above k (or the not above upper line when
staff does not have exactly 4 lines) closer to the notes when no ledger line is present.
The heuristics used by Gregorio are not perfect so it may be necessary to make
the presence or absence of ledger line explicit for horizontal episema placement.
This is done in the exact same way as for stem length: place \texttt{[ll:0]} or
\texttt{[ll:1]} on the note carrying the episema, to force gregorio to consider the
absence or presence of a ledger line in episema placement.

\subsection{Horizontal Episema Tuning}

The horizontal episema position within the space can be adjusted should the
defaults not be satisfactory.

There are five tunable dimensions:

\begin{tabularx}{\textwidth}{l|X}
	Dimension & Description \\
	\hline
	\texttt{overhepisemalowshift} & The shift for positioning a horizontal episema that is over a note in a low position in the space\\
	\texttt{overhepisemahighhift} & The shift for positioning a horizontal episema that is over a note in a high position in the space\\
	\texttt{underhepisemalowshift} & The shift for positioning a horizontal episema that is under a note in a low position in the space\\
	\texttt{underhepisemahighhift} & The shift for positioning a horizontal episema that is under a note in a high position in the space\\
	\texttt{hepisemamiddleshift} & The shift for centering the horizontal episema in the middle of a space\\
\end{tabularx}

In addition, gabc allows you to adjust the positioning of a given episema by
appending \texttt{[oh:\textit{p}]} (for the episema over the note) or
\texttt{[uh:\textit{p}]} (for the episema under the note).  Here,
\texttt{\textit{p}} is an optional position specifier followed by an optional
nudge.  However at least one or the other must be specified.

The position specifier allows you to select which of the five tunable
dimensions will be used for the base position:

\begin{tabularx}{\textwidth}{l|X}
	Specifier & Base shift \\
	\hline
	\textit{omitted} & Use the default shift based on the position of the episema relative to the note\\
	\texttt{m} & Use \texttt{hepisemamiddleshift}.\\
	\texttt{l} & Use \texttt{overhepisemalowshift} or \texttt{underhepisemalowshift} depending on whether the episema is over or under the note.\\
	\texttt{h} & Use \texttt{overhepisemahighshift} or \texttt{underhepisemahighshift} depending on whether the episema is over or under the note.\\
	\texttt{ol} & Use \texttt{overhepisemalowshift}.\\
	\texttt{oh} & Use \texttt{overhepisemahighshift}.\\
	\texttt{ul} & Use \texttt{underhepisemalowshift}.\\
	\texttt{uh} & Use \texttt{underhepisemahighshift}.\\
\end{tabularx}

The nudge is a \TeX{} dimension specification (number and units) that starts
with \texttt{+} for a nudge upwards or \texttt{-} for a nudge downwards from
base position selected by the position speciifer.  If omitted, the episema will
be drawn at the base position.

In addition, gabc also allows you to specify that a block of notes---possibly
separated with spaces and in different syllables--should be considered a single
unit when it comes to positioning the horizontal episema.  To do this, put
\texttt{[oh:\textit{p}\{]} (for the episema over the note) or
\texttt{[uh:\textit{p}\{]} (for the episema under the note) before the first
note of the block and the corresponding \texttt{[oh\}]} or \texttt{[uh\}]}
after the last note of the block.  When using this syntax, \texttt{\textit{p}}
is the position specifier as before, but is entirely optional, and when
completely omitted, allows the \texttt{:} to also be omitted.

\subsection{Lyric Centering}

Gregorio centers the text of each syllable around the first note of each
syllable.  There are three basic modes, selected with the command \verb=\gresetlyriccentering{<mode>}=:

\begin{description}
\item[syllable] the entire syllable is centered around the first note
\item[firstletter] the first letter of the syllable is centered around the first note
\item[vowel] the vowel sound of the syllable is centered around the first note
\end{description}

The default is \texttt{vowel}, being common in most Gregorian chant
books with text in Latin.

All modes allow you to force the centering with curly brackets,
for example \verb=a{b}c= will center the notes around \texttt{b}.

\subsubsection{Vowel detection}

The default rules built into Gregorio for \texttt{vowel} mode are for
Ecclesiatical Latin and work fairly well (though not perfectly) for
other languages (especially Romance languages).  However, Gregorio
provides a gabc \texttt{language} header which allows the language of
the lyrics to be set. The default is Latin.

Special characters (input with \texttt{<sp>}) or verbatim text (\texttt{<v>})
count as consonants, so you have to force centering around them, for example
\verb=gr{<sp>'ae</sp>}=. If an elision (input with \texttt{<e>}) is present in
the syllable, Gregorio will consider it as consonant too.

If no vowel is found, the notes are centered around the whole syllable.

\subsubsection{Vowel file}

When run, Gregorio will look for a file named
\texttt{gregorio-vowels.dat} in your working directory or amongst the
GregorioTeX files.  If it finds the language requested by the header (matched in a
\emph{case-sensitive} fashion) in one of these files (henceforth called
vowel files), Gregorio will use the rules contained within for vowel
centering.  If it cannot find the requested language in any of the vowel
files or is unable to parse the rules, Gregorio will fall back on the
Latin rules.  If multiple vowel files have the desired language,
Gregorio will use the first matching language section in the first
matching file, according to Kpathsea order.  You may wish to enable
verbose output (by passing the \texttt{-v} argument to
\texttt{gregorio}), if there is a problem, for more information.

The vowel file is a list of statements, each starting with a keyword and
ending with a semicolon (\texttt{;}).  Multiple statements with the same
keyword are allowed, and all will apply.  Comments start with a hash
symbol (\texttt{\#}) and end at the end of the line.

In general, Gregorio does no case folding, so the keywords and language
names are case-sensitive and both upper- and lower-case characters
should be listed after the keywords if they should both be considered in
their given categories.

The keywords are:

\begin{description}

\item[alias]

The \texttt{alias} keyword indicates that a given name is an alias for a
given language.  The \texttt{alias} keyword must be followed by the name
of the alias (enclosed in square brackets), the \texttt{to} keyword, the
name of the target language (enclosed in square brackets), and a
semicolon.  Since gregorio reads the vowel files sequentially, aliases
should precede the language they are aliasing, for best performance.

\item[language]

The \texttt{language} keyword indicates that the rules which follow are
for the specified language.  It must be followed by the language name,
enclosed in square brackets, and a semicolon.  The language specified
applies until the next language statement.

\item[vowel]

The \texttt{vowel} keyword indicates that the characters which follow,
until the next semicolon, should be considered vowels.

\item[prefix]

The \texttt{prefix} keyword lists strings of characters which end in a
vowel, but when followed by a sequence of vowels, \emph{should not} be
considered part of the vowel sound.  These strings follow the keyword
and must be separated by space and end with a semicolon.  Examples of
prefixes include \emph{i} and \emph{u} in Latin and \emph{qu} in
English.

\item[suffix]

The \texttt{suffix} keyword lists strings of characters which don't
start with a vowel, but when appearing after a sequence of vowels,
\emph{should} be considered part of the vowel sound.  These strings
follow the keyword and must be separated by space and end with a
semicolon.  Examples of suffixes include \emph{w} and \emph{we} in
English and \emph{y} in Spanish.

\item[secondary]

The "secondary" keyword lists strings of characters which do not contain
vowels, but for which, when there are no vowels present in a syllable,
define the center of the syllable.  These strings follow the keyword and
must be separated by space and end with a semicolon.  Examples of
secondary sequences include \emph{w} from Welsh loanwords in English and
the syllabic consonants \emph{l} and \emph{r} in Czech.

\end{description}

By way of example, here is a vowel file that works for English:

\begin{lstlisting}[autogobble]
alias [english] to [English];

language [English];

vowel aàáAÀÁ;
vowel eèéëEÈÉË;
vowel iìíIÌÍ;
vowel oòóOÒÓ;
vowel uùúUÙÚ;
vowel yỳýYỲÝ;
vowel æǽÆǼ;
vowel œŒ;

prefix qu Qu qU QU;
prefix y Y;

suffix w W;
suffix we We wE WE;

secondary w W;
\end{lstlisting}


\begin{appendices}
	%Turn off the \filbreak stuff to prevent it from messing up the long tables in the appendicies
	\let\subsection\oldsubsection
	\let\subsubsection\oldsubsubsection
	\renewcommand\thetable{\thesection\arabic{table}}
	\renewcommand\thefigure{\thesection\arabic{figure}}
	% !TEX root = GregorioRef.tex
% !TEX program = LuaLaTeX+se
%
% Copyright (C) 2006-2017 The Gregorio Project (see CONTRIBUTORS.md)
%
% This file is part of Gregorio.
%
% Gregorio is free software: you can redistribute it and/or modify
% it under the terms of the GNU General Public License as published by
% the Free Software Foundation, either version 3 of the License, or
% (at your option) any later version.
%
% Gregorio is distributed in the hope that it will be useful,
% but WITHOUT ANY WARRANTY; without even the implied warranty of
% MERCHANTABILITY or FITNESS FOR A PARTICULAR PURPOSE.  See the
% GNU General Public License for more details.
%
% You should have received a copy of the GNU General Public License
% along with Gregorio.  If not, see <http://www.gnu.org/licenses/>.
%
\begin{landscape}

\section{Font Glyph Tables}\label{glyphtable}

\subsection{Score Font Glyphs}

The following table lists all of the score glyphs available in the greciliae
font and any variant glyphs contained within.  Some of the glyphs listed are
representative of sets of glyphs differentiated by the ambitus of the component
notes.  These are listed with English words for the numbers in italics, such as
{\itshape TwoTwo}.  The gabc column lists a gabc sequence that uses the given
glyph.  If there are small, slanted characters, such as \excluded{gege} in this
column, they produce glyphs additional to the given glyph, but are necessary
for the given glyph to appear.  Note: glyphs for the horizontal episema
(activated using {\ttfamily\char`_} in gabc) are excluded from this table.

\newcommand\ScoreFontTable[1]{%
	\begin{longtable}{llc|cc|lc|cc}
			\caption{Score Glyphs}\\
			&
			&%
			&%
			\multicolumn{2}{c|}{\bfseries Variants}&
			\multicolumn{2}{c|}{\bfseries Cavum}&
			\multicolumn{2}{c}{\bfseries Cavum Variants}\\
			\hhline{>{\arrayrulecolor{lightgray}}--->{\arrayrulecolor{black}}------}
			{\bfseries Glyph Name}&%
			{\scriptsize\bfseries Sample gabc}&%
			{\scriptsize\bfseries Glyph}&%
			{\scriptsize\bfseries Name}&%
			{\scriptsize\bfseries Glyph}&%
			{\scriptsize\bfseries Sample gabc}&%
			{\scriptsize\bfseries Glyph}&%
			{\scriptsize\bfseries Name}&%
			{\scriptsize\bfseries Glyph}\\
			\hline
		\endfirsthead
			&%
			&%
			&%
			\multicolumn{2}{c|}{\bfseries Variants}&
			\multicolumn{2}{c|}{\bfseries Cavum}&
			\multicolumn{2}{c}{\bfseries Cavum Variants}\\
			\hhline{>{\arrayrulecolor{lightgray}}--->{\arrayrulecolor{black}}------}
			{\bfseries Glyph Name}&%
			{\scriptsize\bfseries Sample gabc}&%
			{\scriptsize\bfseries Glyph}&%
			{\scriptsize\bfseries Name}&%
			{\scriptsize\bfseries Glyph}&%
			{\scriptsize\bfseries Sample gabc}&%
			{\scriptsize\bfseries Glyph}&%
			{\scriptsize\bfseries Name}&%
			{\scriptsize\bfseries Glyph}\\
			\hline
		\endhead
		\directlua{GregorioRef.emit_score_glyphs(#1)}
	\end{longtable}
}%
\ScoreFontTable{'greciliae', 'greciliaeHollow'}

\subsection{Dominican Score Font Glyphs}

The following table lists all of the score glyphs available in the Dominican
versions of the greciliae fonts in the same vein as the prior table.

\ScoreFontTable{'greciliaeOp', 'greciliaeOpHollow'}

\subsection{Extra Glyphs}\label{subsec:greextra}

The following table lists the glyphs available in the greextra font.  There are
score glyphs which may be substituted into the score, text glyphs meant to be
used in the verses or in the \TeX{} document, and miscellaneous glyphs like
decorative lines for more specialized use.

\begin{longtable}{lc|lc}
		\caption{Extra Glyphs}\\
		{\bfseries Glyph Name}&{\bfseries Glyph}&{\bfseries Glyph Name}&{\bfseries Glyph}\\
		\hline
	\endfirsthead
		{\bfseries Glyph Name}&{\bfseries Glyph}&{\bfseries Glyph Name}&{\bfseries Glyph}\\
		\hline
	\endhead
	\directlua{GregorioRef.emit_extra_glyphs('greextra')}
\end{longtable}

\end{landscape}

\end{appendices}

\addcontentsline{toc}{section}{Index}
\printindex

\end{document}
